\documentclass[output=paper,modfonts,nonflat
% colorlinks,
% citecolor=brown,
% newtxmath
]{langsci/langscibook} 
% \bibliography{localbibliography}

\author{
Jacek Witkoś\affiliation{Adam Mickiewicz University in Poznań}\orcid{ 0000-0001-6462-3117}\and 
 Paulina Łęska\affiliation{Adam Mickiewicz University in Poznań}\orcid{0000-0002-8817-9409}\lastand 
 Dominika Dziubała-Szrejbrowska\affiliation{Adam Mickiewicz University in Poznań}}

\title{Dative-marked arguments as binders in Polish}  
\abstract{This paper aims to account for peculiar binding properties of dative arguments in Polish: objects and dative object experiencers. Polish reflexive pronouns are (nominative) subject oriented, they can be bound by dative experiencers \citep{miechowicz2008,witkos2007}. At the same time, object experiencers, unlike nominative subjects, are also proper antecedents for both reflexive and pro\-nominal possessives. This mixed behaviour poses a puzzle for traditional and novel formulations of binding theory, which assume complementarity between anaphors and pronominals and plainly states that the subject is the privileged binder in Slavic. We base our analysis on the concept of index raising, where the undifferentiated ana\-pho\-ric/pro\-nomi\-nal element is (covertly) moved and adjoined to \textit{v} or T \citep{safir2014,nikolaeva2014}. The distribution of the two spellout forms of the anaphoric or pronominal elements is determined by its landing site and the case position of the binder.

\keywords{binding, psych verbs, dative antecedents, dative object experiencers, Polish}
}

% % add all extra packages you need to load to this file  
\usepackage{tabularx} 
\definecolor{lsDOIGray}{cmyk}{0,0,0,0.45}

\usepackage{xassoccnt}
\newcounter{realpage}
\DeclareAssociatedCounters{page}{realpage}
\AtBeginDocument{%
  \stepcounter{realpage}
}

%%%%%%%%%%%%%%%%%%%%%%%%%%%%%%%%%%%%%%%%%%%%%%%%%%%%
%%%           Examples                           %%%
%%%%%%%%%%%%%%%%%%%%%%%%%%%%%%%%%%%%%%%%%%%%%%%%%%%%  
%% if you want the source line of examples to be in italics, uncomment the following line
% \renewcommand{\exfont}{\itshape}
\usepackage{lipsum}
\usepackage{langsci-optional}
\usepackage{./langsci-osl}
\usepackage{langsci-lgr}
\usepackage{langsci-gb4e}
\usepackage{stmaryrd}
\usepackage{pifont} % needed for checkmark \ding{51} and cross \ding{55}
\usepackage[linguistics]{forest}

% ch06
\usepackage[euler]{textgreek}
% ch07
\usepackage{soul}
\usepackage{graphicx}
% ch08, ch14
\usepackage{multicol}
% ch11
\usepackage{fnpct}
% ch13
\usepackage{scrextend}
\usepackage{enumitem}
% ch14
\usepackage{tabto}
\usepackage{multirow}

\usepackage{langsci-cgloss}

% \newcommand{\smiley}{ :) }

% non-italics in examples

\renewcommand{\eachwordone}{\upshape}

% non-italics in examples in footnotes

\renewcommand{\fnexfont}{\footnotesize\upshape}
\renewcommand{\fnglossfont}{\footnotesize\upshape}
\renewcommand{\fntransfont}{\footnotesize\upshape}
\renewcommand{\fnexnrfont}{\fnexfont\upshape}

% chapter03 goncharov

\newcommand{\p}{\textsc{pfv\ }}
\newcommand{\im}{\textsc{ipfv\ }}

\makeatletter
\let\thetitle\@title
\let\theauthor\@author 
\makeatother

\newcommand{\togglepaper}[1][0]{  
  \addbibresource{../localbibliography.bib}  
  \papernote{\scriptsize\normalfont
    \theauthor.
    \thetitle. 
    To appear in: 
    Change Volume Editor \& in localcommands.tex 
    Change volume title in localcommands.tex
    Berlin: Language Science Press. [preliminary page numbering]
  }
  \pagenumbering{roman}
  \setcounter{chapter}{#1}
  \addtocounter{chapter}{-1}
}

\providecommand{\orcid}[1]{}

\IfFileExists{../localcommands.tex}{%hack to check whether this is being compiled as part of a collection or standalone
  % add all extra packages you need to load to this file  
\usepackage{tabularx} 
\definecolor{lsDOIGray}{cmyk}{0,0,0,0.45}

\usepackage{xassoccnt}
\newcounter{realpage}
\DeclareAssociatedCounters{page}{realpage}
\AtBeginDocument{%
  \stepcounter{realpage}
}

%%%%%%%%%%%%%%%%%%%%%%%%%%%%%%%%%%%%%%%%%%%%%%%%%%%%
%%%           Examples                           %%%
%%%%%%%%%%%%%%%%%%%%%%%%%%%%%%%%%%%%%%%%%%%%%%%%%%%%  
%% if you want the source line of examples to be in italics, uncomment the following line
% \renewcommand{\exfont}{\itshape}
\usepackage{lipsum}
\usepackage{langsci-optional}
\usepackage{./langsci-osl}
\usepackage{langsci-lgr}
\usepackage{langsci-gb4e}
\usepackage{stmaryrd}
\usepackage{pifont} % needed for checkmark \ding{51} and cross \ding{55}
\usepackage[linguistics]{forest}

% ch06
\usepackage[euler]{textgreek}
% ch07
\usepackage{soul}
\usepackage{graphicx}
% ch08, ch14
\usepackage{multicol}
% ch11
\usepackage{fnpct}
% ch13
\usepackage{scrextend}
\usepackage{enumitem}
% ch14
\usepackage{tabto}
\usepackage{multirow}

\usepackage{langsci-cgloss}

  \newcommand{\smiley}{ :) }

% non-italics in examples

\renewcommand{\eachwordone}{\upshape}

% non-italics in examples in footnotes

\renewcommand{\fnexfont}{\footnotesize\upshape}
\renewcommand{\fnglossfont}{\footnotesize\upshape}
\renewcommand{\fntransfont}{\footnotesize\upshape}
\renewcommand{\fnexnrfont}{\fnexfont\upshape}

% chapter03 goncharov

\newcommand{\p}{\textsc{pfv\ }}
\newcommand{\im}{\textsc{ipfv\ }}

\makeatletter
\let\thetitle\@title
\let\theauthor\@author 
\makeatother

\newcommand{\togglepaper}[1][0]{  
  \addbibresource{../localbibliography.bib}  
  \papernote{\scriptsize\normalfont
    \theauthor.
    \thetitle. 
    To appear in: 
    Change Volume Editor \& in localcommands.tex 
    Change volume title in localcommands.tex
    Berlin: Language Science Press. [preliminary page numbering]
  }
  \pagenumbering{roman}
  \setcounter{chapter}{#1}
  \addtocounter{chapter}{-1}
}

\providecommand{\orcid}[1]{} 
\togglepaper[15]
}{}

\begin{document}
\maketitle

% The part below is from the generic LangSci Press template for papers in edited volumes. Delete it when you're ready to go.

% SECTION 1
\section{Introduction}\label{sec:witkos:s1}

\sloppy A-binding has been a chief area of research in comparative linguistics since the early 80s (\citealt{chomsky1981,chomsky1986,manzini1987,bellettirizzi1988,rappaport1986,willim1989,burzio1996,hellan1988,progovac1992,progovac1993,avrutin1994}; among others) when the foundations for modern theory of binding were laid. It very soon became transparent that binding phenomena were subject to parametric differences involving such notions as the size of the binding domain, the morphology of the anaphoric element and the choice of the privileged binder. This paper touches upon the last aspect of the parametric difference, namely the strict subject orientation of anaphors in Polish (and other Slavic languages) as well as certain conditions which dative arguments must meet to qualify for binders; it turns out that even when dative arguments happen to be supreme arguments in particular structures, they do not fully mimic the behaviour of nominative subjects as binders.

Polish is a subject-oriented binding language, and objects, either dative or accusative-marked ones, cannot bind anaphors in other object/adjunct positions (\citealt{willim1989,reinders1991,rappaport1986} for almost identical data in Russian), as presented in \REF{ex:witkos:1}.

%%%%%%%%%%%%%%%%%%%%%%%%%%%%%
% example 1
\ea \label{ex:witkos:1}
	\ea{ \label{ex:witkos:1a}
    \gll Jan\textsubscript{1}     pokazał Marii\textsubscript{2} \minsp{[\{} swoje\textsubscript{1,*2} / jej\textsubscript{2} / \minsp{*} jego\textsubscript{1}\} zdjęcie].\\      
         Jan.\textsc{nom} showed Maria.\textsc{dat} {} self.\textsc{poss} {} her {} {} his picture.\textsc{acc}\\
    \glt `Jan showed Maria his/her picture.'
    }
	\ex{ \label{ex:witkos:1b}
    \gll Jan\textsubscript{1}    zawierzył córkę\textsubscript{2}  \minsp{[\{} swojej\textsubscript{1,*2} / jej\textsubscript{2} / \minsp{*} jego\textsubscript{1}\} patronce].\\
         Jan.\textsc{nom} entrusted daughter.\textsc{acc} {} self.\textsc{poss} {} her {} {} his patron.\textsc{dat}\\
    \glt `Jan entrusted his daughter to his/her patron.'
    }
    %  * WHICH BELONGS TO "NIM" KEEPS IN FIRST LINE AND IS SEPARATED
    \ex{ \label{ex:witkos:1c}
    \gll Jan\textsubscript{1}    opowiedział    Marii\textsubscript{2} \minsp{[} o \minsp{\{} sobie\textsubscript{1,*2} / \minsp{?} niej\textsubscript{2}\} \minsp{(} samej) /\hspace{0.5cm} \minsp{*} nim\textsubscript{1}].\\
         Jan.\textsc{nom} told Maria.\textsc{dat} {} about {} self {} {} her {} alone {} {} him\\
    \glt `Jan told Maria about himself/her.'
    }
	\z
\z

%%%%%%%%%%%%%%%%%%%%%%%%%%%%%

\noindent Both the reflexive pronoun and the reflexive possessive seem to be oriented towards the nominative subject, while dative and accusative objects are infelicitous binders in \REF{ex:witkos:1}.

In certain constructions referring to psychological states, dative arguments bind anaphoric pronouns but allow for optionality with pronominal/reflexive possessives.

% example 2
\ea \label{ex:witkos:2}
	\ea[]{ \label{ex:witkos:2a}
    \gll Marii\textsubscript{1} było żal \minsp{\{} siebie\textsubscript{1} / \minsp{*?} jej\textsubscript{1}\} \minsp{(} samej).\\
         Maria.\textsc{dat} was.\textsc{3sg.n} 	sorrow.\textsc{3sg.m} {} self.\textsc{gen} {} {} her.\textsc{gen} {} alone\\
    \glt `Maria felt sorry for herself.'
    }
	\ex[]{ \label{ex:witkos:2b}
    \gll Marii\textsubscript{1} było żal \minsp{\{} swojej\textsubscript{1} / jej\textsubscript{1}\} koleżanki.\\
         Maria.\textsc{dat} was.\textsc{3sg.n} 	sorrow.\textsc{3sg.m}	{} self.\textsc{poss} {} her	friend.\textsc{gen.f}\\
    \glt `Maria felt sorry for her female friend.'
    }
	\z
\z
%%%%%%%%%%%%%%%%%%%%%%%%%%%%
      
\noindent The psychological predicate \textit{podobać się} ‘appeal to’ shows variable behaviour: when bound, the possessive pronoun in the nominative argument is strongly preferred to the possessive reflexive, as in \REF{ex:witkos:3}. However, \cite{witkos2007,witkos2008} shows that a preverbal dative-marked argument can be involved in anaphoric binding into the nominative-marked constituent (cf. \REF{ex:witkos:4}):\footnote{\label{fn1}A detailed analysis of dative object experiencers of both verbal and non-verbal psychological predicates remains beyond the scope of this contribution. Recent analyses are presented in \cite{jimenezfernandez2016} and in \cite{bondaruk2017}.}




% example 3
\ea \label{ex:witkos:3}
    \gll Marii\textsubscript{1} spodobała się \minsp{\{?*} swoja$_1$ / jej\textsubscript{1}\}  nowa sukienka.\\
         Maria.\textsc{dat} liked \textsc{refl} {} self.\textsc{poss} {} her new dress.\textsc{nom}\\
    \glt `Maria liked her new dress.'
\z

% example 4
\ea \label{ex:witkos:4}
    \gll \minsp{[} Nowakom\textsubscript{2}]   spodobała   się   \minsp{[} nowa książka  \minsp{(} Kowalskich\textsubscript{1}) o sobie\textsubscript{1,2}]\\
        {} Nowaks.\textsc{dat} liked \textsc{refl} {} new book.\textsc{nom} {} Kowalskis.\textsc{poss} about self\\
    \glt `The Nowaks liked the new book (by the Kowalskis) about themselves/them.'
\z

\noindent We address these issues by developing and updating an approach to binding based on \citeposst{nikolaeva2014} \textsc{index raising} (IR) and \cite{despic2013,despic2015}. In the view of the data in \REF{ex:witkos:2} and \REF{ex:witkos:4}, our goal is to explain why dative antecedents in constructions with psychological verbs, \REF{ex:witkos:2b}, allow for the option of binding both reflexive and pronominal possessives, while the nominative antecedent allows only for the reflexive possessive variant.\footnote{\label{fn2}We will not take into consideration reciprocal constructions in Polish, whose properties are markedly distinct from reflexives and identical to Russian reciprocals (\citealt{willim1989,reinders1991,rappaport1986} for Russian). For instance, in contrast to reflexives, reciprocals are not subject oriented and can be bound by the object as well. In terms of the IR-based analysis, reciprocal pronouns in Polish do not undergo IR.} We submit that these different binding properties are due to different positions occupied by nominative and dative antecedents, namely SpecTP and Spec\textit{v}P, respectively. We also claim that data such as \REF{ex:witkos:3}, though plentiful, are encumbered with an additional complicating factor in the form of the Anaphor Agreement Effect (AAE, \citealt{rizzi1990,progovac1992,progovac1993,woolford1999,reuland2011}) and they deserve a slightly different treatment. The most straightforward diagnostics for determining the binding potential of the dative argument involves cases when it binds (into) non-nominative elements (so \REF{ex:witkos:2} rather than \REF{ex:witkos:3}).

The paper is organized as follows. In \sectref{sec:witkos:s2}, we present an outline of our theory of binding, with emphasis on our version of the IR theory articulated in \cite{nikolaeva2014}, modified in line with \cite{boskovic2005locality,boskovic2012,boskovic2013,boskovic2014} and \cite{despic2011,despic2013,despic2015}. \sectref{sec:witkos:s3} provides our account of binding in structures with dative arguments. We show why the dative argument of the ditransitive verb cannot bind reflexive elements, we analyse the position and binding option of the dative \textsc{object experiencer} (OE). \sectref{sec:witkos:s4} concludes the paper.

% SECTION 2
\section{Components of the Analysis} \label{sec:witkos:s2}

Our account of anaphoric binding in Polish follows from and draws from a triplet of sources: (A) approaches which stress the need for (covert) anaphor raising to some functional head position, usually Infl or T (\citealt{vikner1985,chomsky1986,pica1987,pica1991,hestvik1992,avrutin1994,nikolaeva2014}), (B) approaches which stress the morphological impoverishment of the anaphoric elements \citep{burzio1991,burzio1996,safir2014}, and (C) approaches that recognize the notion of derived complementarity \citep{hellan1988,safir2004,boeckxetal2008}.

\begin{itemize}
    \item[(A)] The most identity dependent form in \REF{ex:witkos:1}, be it anaphoric, personal or possessive, is overlaid with lexical content late in the derivation, at Spell-Out. It is introduced into initial numeration as an undefined element, the most dependent form, called D-bound in \cite{safir2014}, the index in \cite{nikolaeva2014}, or root-pron in \cite{heinat2008}. \cite[91--92]{safir2014} defines properties of D-bound/index in the following way:
    % example 5
    \ea \label{ex:witkos:5}
    %\begin{itemize}
        \ea{Always a variable: D-bound is the same object in SEM (the syntactic input to semantic interpretation) in all cases; it is interpreted as a bound variable regardless of its φ-features. \label{ex:witkos:5a}}
        \ex{Always A-bound: the binder of D-bound (its antecedent) must c-com\-mand it from an A-position; that is, the D-bound form is A-bound. (We further narrow down the definition of A-position to the position where the antecedent has its case valued). \label{ex:witkos:5b}}
        \ex{ Always feature compatible: D-bound must be feature compatible with its antecedent (informally, this property may be termed antecedent agreement).\label{ex:witkos:5c}}
        \ex{ Spell-Out of the morphological shape of D-bound is potentially sensitive to whether A-binding is phase internal:
        \begin{itemize}
            \item[--] agreement compatible with morphological shape may be determined by phase internal factors locally distinct from antecedent agreement;
            \item[--] D-bound enters the derivation with φ-features arbitrarily assigned to it;
            \item[--] anywhere phase-internal shape is not required, D-bound receives default pronominal shape.
        \end{itemize} \label{ex:witkos:5d}}
    %\end{itemize}
    \z
    \z
    \item[(B)] The D-bound/index is impoverished in its feature composition, very much like the lexical anaphor in Polish, in that it has a [$-$var] feature.\footnote{\label{fn4}The Polish reflexive pronoun and the reflexive possessive inflects for case but not for person, number and gender. The reflexive pronoun \textit{siebie} ‘self’ also has a weak/clitic form \textit{się} but we leave this issue aside in this paper.} The underspecification of this feature forces the index to move to a position where this interpretive impoverishment can be compensated for, in line with a similar procedure for semantically and morphologically deficient pronominal clitics in \cite{bejar2003} and \cite{franks2017,franks2018}.\footnote{\label{fn6}\cite{franks2017,franks2018} claims that clitics are deficient in three respects: prosodically, semantically and syntactically:

% (i)
\ea
The prosodic deficiency: Clitics cannot project prosodic feet. \citep[147]{franks2017}
\z

% (ii)
\ea
The semantic deficiency:\\
\ea{Clitics cannot instantiate lexico-conceptual features.}
\ex{A clitic may not have [+person] features (either entirely or only subcomponents [Participant [Author]] of the 1\textsuperscript{st}/2\textsuperscript{nd} person).}
\z
\z
% (iii)
\ea
The syntactic deficiency: Clitics cannot express syntactic complexity (they are heads).
\z

\noindent In our analysis, the index does not show prosodic deficiency. Following \cite{cardinalettistarke1994} and \cite{bejar2003}, \cite{franks2017} proposes that [+person] must be licensed by entering into an agree relation with a functional category.}

    \item[(C)] The index moves from its thematic/case position to the head \textit{v}/T, but it is not phonologically impoverished the way clitics are. This is why its movement forms a chain in which the copy is pronounced.\footnote{\label{fn7}A similar idea of an element raising (to the edge of the \textit{v}P phase) and having its copy pronounced as reflexive is applied in an analysis of binding in German in \cite{safir2004} and in \cite[291]{leeschoenfeld2008}. According to the latter source the licensing of \textit{sich} ‘self’ co-indexed with \textit{mother} requires covert movement:
% (i)
\ea
\gll Die Mutter$_i$ lässt [\textsubscript{\textit{v}P} die Kleine$_j$ \minsp{\{} sich\textsubscript{?i/j} / ihr\textsubscript{i/*j}\} die Schokolade in den Mund stecken].\\
the mother lets {} the little-one {} self {} her the chocolate in the mouth stick\\
\glt `The mother lets the little girl stick the chocolate in her mouth.’
\z}\textsuperscript{,}\footnote{\label{fn8}A reviewer for this volume expresses doubts as to whether a non-phonological clitic such as our D-bound/index should behave movement-wise like a clitic and pick the same landing site \textit{v}\textsuperscript{0}/T\textsuperscript{0}. This reservation can be addressed in a number of ways. First, let us point out that in terms of their syntax non-clitic elements can be ambiguous between X\textsuperscript{0}/XP status and participate in head movement irrespective of their phonological properties; after all, clitic movement constitutes a subset of head movement. Second, in one of its multiple functions the Polish clitic \textit{się} ‘self’ serves as the clitic form replacement of the reflexive pronoun \textit{siebie} ‘self’. Importantly, the distribution of this type of \textit{się} ‘self’ fully overlaps with the distribution of clitic/weak pronouns and the span of the binding domain in Polish.

We claim that this overlap is not accidental but due to the same underlying operation: movement of D-bound/index and clitic/weak pronoun to the same functional head placed outside VP. Third, there are fruitful analyses of grammatical phenomena in Germanic (scrambling) and Romance languages, which link abstract (covert) clitic elements with overt non-clitic phenomena, such as \cite{sportiche1996} and the concept of ``clitic voices''.}.   
\end{itemize}

In short: we take the relation of binding to hold between the antecedent c-commanding D-bound/ index from its case position. The spell-out form of D-bound/index is determined by its movement to \textit{v}/T.

\cite{nikolaeva2014}, building on \cite{chomsky1986,vikner1985,pica1987,pica1991,hestvik1992} and \cite{avrutin1994}, proposes that the lexicalisation of D-bound/index depends on IR. We modify her original proposal as in \figref{fig:6}.

% FIGURE 6
%(6)		Index Positions: a ditransitive predicate (the DOC type);\footnote{In her original proposal Nikolaeva proposed an intermediate landing site for the index, called position [1], tucked in below the higher object still inside VP. This position was supposed to be phrasal and it caused Anti-Cataphora Effects (ACE, Principle C effects). We decided to follow the structure of the NP in Despić (2013, 2015), see ex. (13), where the possessor c-commands outside the NP, which allows us to account for the ACE. We dispense with the liability of Nikolaeva’s position [1], neither a thematic nor a case position which had to count as an A-position relevant for binding. For ease of presentation we preserve the original numbering of positions [2] and [3] in diagrams (6-7).} 
%%%%%%%%%%%%%%%%%%%%%%%%%%%%%%%%%%%%%%%%%%%%%%%%%%%%%%%%%%%%%%%%%%%%%%%%%%%%%%%%%%%%%%%

\begin{figure}[ht]
\centering
    \begin{forest}
    for tree={s sep=0.5cm, inner sep=0, l=0}
    [TP, s sep=2.2cm
        [Sub\textsubscript{NOM}, name=spec SUB1
        ]
        [T$'$
            [T\textsuperscript{0}
                [\fbox{3}\\\textsc{rfl/prn}, name=spec 3]
                [T\textsuperscript{0}
                ]
            ]
            [\textit{v}P, s sep=2.2cm
                [Sub\textsubscript{NOM}
                ]{
                    \draw[->] () to [out=south west,in=south,looseness=0.8] (spec SUB1);
                }
                [\textit{v}$'$
                    [\textit{v}\textsuperscript{0}
                        [\fbox{2}\\\textsc{rfl/prn},name=spec 2]
                        {
                            \draw[->] () to [out=south west,in=south] (spec 3);
                        }
                        [\textit{v}\textsuperscript{0}
                        ]
                    ]
                    [VP, s sep=12mm
                        [NP$_{2}$
                            [{\ldots} , roof]
                        ]
                        [V$'$, s sep=15mm
                            [V\textsuperscript{0}
                            ]
                            [NP$_{1}$, s=2cm
                                [NP
                                    [\textsc{rfl/prn}, roof]{
                                                    \draw[->] () to [out=south west,in=south] (spec 2);
                                                     }
                                ]
                                [NP$_{1}$
                                    [{\ldots}, roof]
                                ]
                            ]
                        ]
                    ]
                ]
            ]
        ]
    ]
    % \draw[->] (trace-DP) to[out=south west, in=south west, looseness=1.2] (what);
    % \draw[->] (trace-T) to[out=south west, in=south west] (are);
    \end{forest}
    \caption{Index positions: a ditransitive predicate (the DOC type)}
    \label{fig:6}
\end{figure}

%%%%%%%%%%%%%%%%%%%%%%%%%%%%%%%%%%%%%%%%%%%%%%%%%%%%%%%%%%%%%%%%%%%%%%%%%%%%%%%%%%%%%%%%%%%%

% FIGURE 7
% (7)		Index Positions: a psychological predicate with ACC/\\dat Experiencer;

%%%%%%%%%%%%%%%%%%%%%%%%%%%%%%%%%%%%%%%%%%%%%%%%%%%%%%%%%%%%%%%%%%%%%%%%%%%%%

\begin{figure}[ht]
\centering
    \begin{forest}
    for tree={s sep=1cm, inner sep=0, l=0}
    [TP
        [{},inner sep=3mm
        ]
        [T$'$
            [T\textsuperscript{0}
                [\fbox{3}\\\textsc{rfl/prn},name=spec 3]
                [T\textsuperscript{0}]
            ]
                [\textit{v}P, s sep=2cm
                    [NP$_{2}$.Exp
                ]
                [\textit{v}$'$
                    [\textit{v}\textsuperscript{0}
                        [\fbox{2}\\\textsc{rfl/prn}, name=spec 2
                        ]{
                            \draw[->] () to [out=south west,in=south] (spec 3);
                        }
                        [\textit{v}\textsuperscript{0}
                        ]
                    ]
                    [VP
                        [V\textsuperscript{0}
                        ]
                        [NP$_{1}$
                            [NP
                                [\textsc{rfl/prn}
                                , roof]{
                            \draw[->] () to [out=south west,in=south] (spec 2);
                        }
                            ]
                            [NP$_{1}$
                                [{\xspace\hspace{1cm}\xspace}, roof]
                            ]
                        ]
                    ]
                ]
            ]
        ]
    ]
    % \draw[->] (trace-DP) to[out=south west, in=south west, looseness=1.2] (what);
    % \draw[->] (trace-T) to[out=south west, in=south west] (are);
    \end{forest}
    \caption{Index positions: A psychological predicate with \textsc{acc/dat} experiencer}
    \label{fig:7}
\end{figure}




%%%%%%%%%%%%%%%%%%%%%%%%%%%%%%%%%%%%%%%%%%%%%%%%%%%%%%%%%%%%%%%%%%%%%%%%%%%%%%%%%

The diagrams in \figref{fig:6} and \figref{fig:7} show the placement of arguments in both the construction with ditransitive verbs and psychological predicates. In \figref{fig:6} the direct object is the complement to V\textsuperscript{0}, the indirect object occupies SpecVP, from which it c-commands NP\textsubscript{1}. The position of the dative experiencer in Spec\textit{v}P corresponds to the dative in \REF{ex:witkos:2}--\REF{ex:witkos:4} with psychological predicates. As the diagrams above show, we assume two distinct positions for dative goals, bene- and malefactives (NP\textsubscript{2} in SpecVP) and dative OEs (Spec\textit{v}P). We follow \cite{larson1988,larson1990,larson2014} for the placement of the former and \cite{woolford2006} for the placement of the latter. Two positions are reserved for the agentive subject: the bottom of its A-chain in Spec\textit{v}P and the top of its A-chain in SpecTP. The gist of the lexicalisation procedure is as follows \citep[68]{nikolaeva2014}:
\largerpage

% example 8
\ea\label{ex:witkos:8}
\ea\label{ex:witkos:8a} \textit{Movement}: an index (marked as \textsc{rfl/prn} in \figref{fig:6}--\figref{fig:7}) must undergo IR unless it is at a lexicalisation site or movement is no longer possible.
\ex \label{ex:witkos:8b} \textit{Lexicalisation site}: an index is a sister to a node with label D\textsuperscript{0}/\textit{v}\textsuperscript{0}/T\textsuperscript{0} and is c-commanded by a specifier,
\ex \label{ex:witkos:8c} \textit{Co-argumental Lexicalisation}: if an index is at a reflexivization site and is coindexed with a specifier which is its co-argument, the index has to be realized as reflexive.
\ex \label{ex:witkos:8d} \textit{Lexicalisation at spell-out}: when the sentence is sent to spell-out, if an index is coindexed with a specifier of the projection to which it is adjoined, the index has to 	be realized as reflexive.
\ex \label{ex:witkos:8e} \textit{Pronominal is an elsewhere condition}: if an index has not been realized as reflexive, it is realized as pronominal.
\z\z

\largerpage
\noindent As VP is not a lexicalisation site by definition, the overt position of the index (pronoun or anaphor) is mostly ignored in the calculation of its spell-out form.\footnote{\label{fn12}Exceptions include clause \REF{ex:witkos:8c} and co-argumental reflexivisation, where pronouns show not only strong anti-subject orientation but also anti-object orientation: 

\ea
\ea[*]{
\gll Mama$_2$ pokazała 	Marii$_1$ ją\textsubscript{1/2} \minsp{(} w lustrze).\\
mother showed Mary.\textsc{dat} her.\textsc{acc} {} in mirror\\
\glt Intended: `Mother showed her to Maria (in the mirror).’
}
\ex[*]{
\gll Mama$_2$ pokazała Marię$_1$ jej\textsubscript{1/2} \minsp{(} w lustrze).\\
mother 	showed 	Mary.\textsc{acc} her.\textsc{dat} {} in mirror\\
\glt Intended: `Mother showed Maria to her (in the mirror).’
}
\z
\z
        
\noindent This issue remains beyond the scope of the current contribution but see \cite{goglozaetal_toappear} for a detailed analysis couched in the IR framework.} IR is closely linked to ideas concerning clitic movement, see \cite{sportiche1996,kayne1985,kayne1991,roberts1992,roberts1993} in the GB tradition. Clause in \REF{ex:witkos:8e} clearly corresponds to the competition-based approach to binding, see \cite{safir2004}, and the movement-based approach, see \citep{hornstein2001,boeckxetal2008}, where the pronoun is the default ‘elsewhere’ option wherever the reflexive cannot be licensed.

Let us sketch three derivations illustrating the mechanics of the system. \cite{safir2014} proposes the following derivation for an English example:

% example 9
\ea\label{ex:witkos:9}
\ea John\textsubscript{1} praised \minsp{\{} himself\textsubscript{1}/*him\textsubscript{1}/him\textsubscript{2}\}.
\ex $[$\textsubscript{\textit{v}P} John $[$\textsubscript{\textit{v}$'$} \textit{v} $[$\textsubscript{VP} praise D-bound+3sg$]]]$
\ex $[$\textsubscript{TP} John $[$\textsubscript{T$'$} T $[$\textsubscript{\textit{v}P} John $[$\textsubscript{\textit{v}$'$} \textit{v} $[$\textsubscript{VP} praise himself$]]]]]$
\z\z


\noindent D-bound/index is merged in with unvalued φ-features assigned to it. \textit{John} is the antecedent for D-bound/index and because \textit{John} is the phase edge, the D-bound/index spells out in the shape indicating phase-internal dependency (the \textit{-self} form in English). The major difference between Polish reflexive forms and the English ones is that the D-bound/index in Polish is impoverished in its feature composition, very much like the Polish lexical anaphor, in that it has an underspecified slot for [$+$φ] features. We take this underspecification to allow for the copying of the φ-features from the antecedent but not for their expression in situ. The expression of these features takes place only upon the movement of the D-bound/index to \textit{v} and T (see \citealt{bejar2003} and \citealt{franks2017} for a corresponding notion of clitic movement to a compensatory position for $[+$person$]$ expression):

% ARROWS ARE MISSING!!!!!!!!!!!!!!!!!!!!!!!!!!!!!!!!!
% example 10
\ea\label{ex:witkos:10}
\ea \gll Jan\textsubscript{1} zobaczył \minsp{\{} siebie\textsubscript{1} /*\hspace{-2pt} jego\textsubscript{1}\}. \label{ex:witkos:10a}\\
Jan noticed {} self.\textsc{acc} {} him.\textsc{acc}\\
\glt `Jan noticed himself.'

\ex{\textit{Binding}\\ \label{ex:witkos:10b}
$[$\textsubscript{\textit{v}P} Jan\textsubscript{1} [\textsubscript{\textit{v}$'$} [\textsubscript{\textit{v}} $+$3sg\textsubscript{1} -- \textit{v}] [\textsubscript{VP} noticed D-bound$[\#$φ] $]]]$\\
\begin{tikzpicture} 
\draw[step=0.5, white, very thin] (0,0) grid (0.5,0.5);
\draw[->,dashed] (1,0.5) -- (1,0) -- (6.5,0) -- (6.5,0.5);
\end{tikzpicture}
}

\ex {\textit{φ-expression}\\
$[$\textsubscript{\textit{v}P} Jan\textsubscript{1} [\textsubscript{\textit{v}$'$} [\textsubscript{\textit{v}} \textsuperscript{ok}D-bound$+$3sg\textsubscript{1} -- \textit{v}] [\textsubscript{VP} noticed \sout{D-bound}[$\#$φ] $]]]$ \label{ex:witkos:10c}\\
\begin{tikzpicture} 
\draw[step=0.5, white, very thin] (0,0) grid (0.5,0.5);
\draw[<-] (3,0.5) -- (3,0) -- (8,0) -- (8,0.5);
\end{tikzpicture}
}
\ex {\textit{Spell-Out}\\
$[$TP Jan\textsubscript{1} [ T [\textsubscript{\textit{v}P} \sout{Jan} [\textsubscript{\textit{v}$'$} [\textsubscript{\textit{v}} \textsuperscript{ok}D-bound$+$3sg\textsubscript{1} -- \textit{v}] [\textsubscript{VP} noticed self$]]]]]$ \label{ex:witkos:10d}\\
\begin{tikzpicture} 
\draw[step=0.5, white, very thin] (0,0) grid (0.5,0.5);
\draw[<-] (5,0.5) -- (5,0) -- (9.5,0) -- (9.5,0.5);
\end{tikzpicture}
}
\z\z

\noindent In \REF{ex:witkos:10a} the D-bound/index is bound in its base position (it copies the φ-features of its antecedent). In \cite{witkosetal_forth}, we treat A-binding as upward agree for feature [+variable], following \cite{hicks2009} rather than plain phi-feature copying.  Here, the index meets \citeposst{safir2014} condition of local antecedent agreement of \REF{ex:witkos:5c}. The φ-features on D-bound/index cannot be expressed in its base position and this morpho-syntactic deficiency forces the index to move to \textit{v}\textsuperscript{0} in \REF{ex:witkos:10b}, forming a chain. At the point of spell-out of the \textit{v}P D-bound/index is realized as the reflexive form \textit{siebie} ‘self’ on the bottom copy of the chain in \REF{ex:witkos:10c}.

The derivation of \REF{ex:witkos:1} follows a similar path. In \REF{ex:witkos:11b}, D-bound/index is c-commanded by its antecedent and copies its φ-features under local antecedent agreement, yet it cannot express them, so it moves to \textit{v} and forms a chain. The lexicalization of (covert) D-bound/index at \textit{v} is determined by Nikolaeva’s \REF{ex:witkos:8d} the NP content of the local Spec\textit{v}P bears φ-features different from D-bound/index, so it is lexicalized as a pronoun at the bottom of its chain in \REF{ex:witkos:11d}:

% ARROWS ARE MISSING!!!!!!!!!!!!!!!!!!!!!!!!!!!!!!!!!
% example 11
\ea\label{ex:witkos:11}
\ea \label{ex:witkos:11a} \gll Jan pokazał Marii jej zdjęcie.\\
Jan showed Maria her picture\\
\glt `Jan showed Maria her picture.'
\ex $[$\textsubscript{\textit{v}P}  Jan.\textsc{nom}\textsubscript{1} [showed [\textsubscript{VP} Maria.\textsc{dat}\textsubscript{2} [\textsubscript{V$’$} V [her\textsubscript{2} picture$]]]]$

\ex {\textit{Binding}\\ \label{ex:witkos:11b} $[$\textsubscript{\textit{v}P} J\textsubscript{1} $[[$\textsubscript{\textit{v}} $+$3sg.f2-\textit{v}] showed [\textsubscript{VP} M\textsubscript{2} [\textsubscript{V$’$} V [D-bound[$\#$φ]\textsubscript{2} [pic$]]]]$\\
\begin{tikzpicture} 
\draw[step=0.5, white, very thin] (0,0) grid (0.5,0.5);
\draw[->,dashed] (5,0.5) -- (5,0) -- (7,0) -- (7,0.5);
\end{tikzpicture}
}
\ex \textit{φ-expression}\\ \label{ex:witkos:11c} $[$\textsubscript{\textit{v}P} J\textsubscript{1} $[[$\textsubscript{\textit{v}} $+$3sg.f2-\textit{v}] showed [\textsubscript{VP} M\textsubscript{2} [\textsubscript{V$’$} V [D-bound[$\#$φ]\textsubscript{2} [pic$]]]]$\\
\begin{tikzpicture} 
\draw[step=0.5, white, very thin] (0,0) grid (0.5,0.5);
\draw[<-] (2,0.5) -- (2,0) -- (7,0) -- (7,0.5);
\end{tikzpicture}

\ex \textit{Spell-Out}\\ \label{ex:witkos:11d} $[$\textsubscript{\textit{v}P} J\textsubscript{1} $[[$\textsubscript{\textit{v}} $+$3sg.f2-\textit{v}] showed [\textsubscript{VP} M\textsubscript{2} [\textsubscript{V$’$} V [her\textsubscript{2} [pic$]]]]]]$\\
\begin{tikzpicture} 
\draw[step=0.5, white, very thin] (0,0) grid (0.5,0.5);
\draw[<-] (2,0.5) -- (2,0) -- (7,0) -- (7,0.5);
\end{tikzpicture}
\z\z

\noindent Interpretation-wise, the dative goal of a ditransitive verb can function as antecedent for the possessive in the accusative object, see \REF{ex:witkos:11c}, but its case position is placed too low in the structure (it is VP-internal in a broad sense) to serve as a local antecedent for the index at the lexicalization site, see \REF{ex:witkos:11d}.\footnote{\label{fn14}It seems that movement of the D-bound/index to a VP-external position is an inevitable step for any empirically adequate account of subject orientation, as it prevents one object from being antecedent of a possessive reflexive embedded in the other object. Even recent conceptually appealing accounts of binding \citep{reuland2011,zubkov2018} take subject orientation for granted.}

One of the consequences of IR is that the index moved via head movement and adjoined to \textit{v}/T (positions [2] and [3]) should not c-command from the head adjoined position, as this would lead to undesirable principle C violations. While it is commonly believed that an adjunct to a maximal projection does c-command outside its adjunction host (see \citealt{kayne1994} and subsequent work), there is less evidence for c-command following head adjunction. \cite[93--94]{nikolaeva2014}: excludes this option by following the definition of c-command in \cite[574]{hestvik1992}: ``x c-commands y iff every node dominating x includes x and y, and x does not dominate y (where x includes y iff y is dominated by every segment of x, as proposed in \cite{may1985}''. Such a definition leaves the c-command domain of the adjunct undefined, as the node dominating the adjunct at the adjunction site does not include it. \cite{citkoetal2018} invokes the ``word interpretation'' notion from \cite[322]{chomsky1995} to prevent such unwelcome c-command: ``at LF, X\textsuperscript{0} is submitted to independent word interpretation processes WI, where WI ignores principles of the computational system within  X\textsuperscript{0}''. If c-command from within a complex head (a word) leading to a violation of binding principle C is such a ``principle of the computational system'' then it can be ignored.\footnote{\label{fn15}\cite{baker1988} argues extensively that heads incorporated into other heads (where incorporation is a showcase example of head movement) cease to act upon elements they used to c-command before incorporation. So, head movement (incorporation) does not extend their c-domain, quite the contrary. For example, in Mohawk, the incorporated N no longer governs (under c/m-command) its possessor and does not license case on it, the verb as the incorporation host governs the possessor instead.}
Furthermore, \cite{roberts2009} develops a minimalist analysis of clitic climbing, to which IR corresponds, and observes that if clitics are taken to minimally constitute only the bundle of φ-features, moving them via excorporation from one head to another is very close to agree for φ-features. 

We propose a particular structure for NPs including possessives, which captures \textsc{anti-}\textsc{cataphora} \textsc{effects} (ACEs):

% example 16
\ea[*]{\label{ex:witkos:16}
    \gll Jan\textsubscript{1} pokazał jej\textsubscript{2} dyplom koleżance Marii\textsubscript{2}.\\
         Jan.\textsc{nom} showed her.\textsc{acc} diploma.\textsc{acc} friend.\textsc{dat} Maria’s.\textsc{dat}\\
    \glt Intended: `Jan showed her diploma to Marta’s friend.'}
\z
                
\noindent R-expressions in Polish cannot be placed in positions following co-indexed pronouns, even if these pronouns apparently do not c-command them in an obvious manner. The grammar of Polish (as well as other Slavic languages) does not tolerate cataphoric relations. \cite{despic2011,despic2013,despic2015} develops an account of binding in Serbo-Croatian (SC) which relies to a large degree on the idea that adjectival possessives are adjuncts and therefore c-command outside the NP they are part of. In SC, the possessive c-commands from its adjoined position, on a theory of adjunction as in \cite{kayne1994}, and thus causes a principle B effect, \REF{ex:witkos:18b}, and a principle C effect, \REF{ex:witkos:18a}, which does not occur in English examples, e.g. \REF{ex:witkos:17}. Significantly, Polish shares with SC the fact that possessive pronouns trigger off the ACEs \REF{ex:witkos:19a}, although nominal possessives do not, see \REF{ex:witkos:19b}, as discussed in \cite{witkos2015}:

% example 17
\ea\label{ex:witkos:17}
\ea His\textsubscript{i} latest movie really disappointed Kusturica\textsubscript{i}.
\ex Kusturica\textsubscript{i}’s latest movie really disappointed him\textsubscript{i}.
\z\z

% example 18
\ea \label{ex:witkos:18}
\ea[*]{ \label{ex:witkos:18a}
\gll Njegov\textsubscript{i} najnoviji film je zaista razočarao Kusturicu\textsubscript{i}.\\
     his latest movie is really disappointed Kusturica.\\
     \glt Intended: `His new movie really dissapointed Kusturica.'}
\ex[*]{ \label{ex:witkos:18b}
\gll Kusturicin\textsubscript{i} najnoviji film ga\textsubscript{i} je zaista razočarao.\\
     Kusturica’s latest 	movie him is really disappointed\\ 
     \glt Intended: `Kusturica's new movie really dissapointed him.'}
\z
\z

\ea \label{ex:witkos:19}
	\ea[*]{ \label{ex:witkos:19a}
    \gll Jego\textsubscript{i} siostra bardzo pocieszyła Janka\textsubscript{i}.\\
         his sister.\textsc{nom} very comfort.\textsc{past} Janek.\textsc{acc}\\
    \glt Intended: `His sister comforted John very much.'
    }
	\ex[]{ \label{ex:witkos:19b}
    \gll Siostra Janka\textsubscript{i} bardzo go\textsubscript{i} pocieszyła.\\
         sister.\textsc{nom} Janek.\textsc{gen} very him.\textsc{acc} comfort.\textsc{past}\\
    \glt `Janek’s sister comforted him very much.'
    }
	\z
\z

\largerpage
\noindent These authors conclude that Polish seems to employ two structures to represent nominal with possessives: the simpler bare NP-structure is used with pronominal and reflexive possessives, while the more complex structure involving possessive phrase and another functional projection (FP) on top of it is used with nominal possessives:

\ea \label{ex:witkos:20}
	\ea[]{ \label{ex:witkos:20a}
    \gll [\textsubscript{NP} jego [\textsubscript{NP} siostra$]]$\\
         {} his.\textsc{gen} {} sister.\textsc{nom}\\
    \glt `his sister'
    }
	\ex[]{ \label{ex:witkos:20b}
    \gll [\textsubscript{FP}[\textsubscript{NP} siostra][\textsubscript{F$'$} F\textsuperscript{0} [\textsubscript{PossP}[\textsubscript{NP} Janka][\textsubscript{Poss$'$} Poss\textsuperscript{0} [\textsubscript{NP} siostra]]]]]\\
         {} sister.\textsc{nom} {} {} Janek.\textsc{gen} {} {} sister.\textsc{nom}\\
    \glt `Janek’s sister'
    }
	\z
\z
                        
\noindent Nominals with pronominal possessors appear to be smaller, truncated versions of structures with nominal possessors.\footnote{\label{fn16}In his analysis of English possessive constructions, \cite{despic2015} proposes a similar solution in that the pronominal possessor is placed at a lower level of the DP structure than the nominal possessor or the reciprocal possessor:

% (i)
\ea
$[$\textsubscript{DP} Mary/each other [\textsubscript{D$’$} [\textsubscript{D} 's] [\textsubscript{PossP} my/their/her [\textsubscript{Poss$’$} Poss [\textsubscript{NP} friends$]]]]]$
\z} The result is that only the pronominal possessives are expected to c-command outside the NP they modify, while nominal possessives do not. Significantly, the structure in \REF{ex:witkos:20a} has the following advantage: the pronominal c-commands outside its NP from its base position, the position where it has both its thematic role and case licensed, thus its A-position.\footnote{\label{fn17}A reviewer raises the question of the origin of the thematic role and case for the possessor, an adjunct in syntax which functions like an argument in LF. We follow \cite{boskovic2005locality,boskovic2012} and \cite{despic2011,despic2013,despic2015} in this regard and assume that the thematic role for the possessive as adjunct is determined compositionally at LF upon the transfer of the nominal phase (NP). Its genitive case is inherent, determined straightforwardly by the thematic relation. A plausible alternative leading to identical consequences for c-command relations, which we do not consider here, would be to posit movement of the pronominal possessor from within an extended projection of NP and adjunction to its outer edge, cf. \cite{ceglowski2017} for a recent analysis of the internal composition of the Polish NP.}

% SECTION 3
\section{Index Raising in action}\label{sec:witkos:s3}

This section serves as an illustration of an application of the notion of IR to constructions with datives in Polish.   

% SUB-SECTION 3.1
\subsection{The VP-internal dative antecedents}\label{s3.1}

The study of ditransitive structures has gained a lot of prominence in Slavic linguistics and the discussion has typically involved two problem areas. Initially, the assumption was that there was one underlying structure for all ditransitive constructions and much of the debate  centred around the issue of the basic order between the \textsc{acc} and \textsc{dat} objects: e.g. \citet{willim1989}, \citet{witkos1998,witkos2007,witkos2008}, \citet{tajsner2008}, and \cite{citko2011} for Polish, and \citet{franks1995}, \citet{dyakonova2007,dyakonova2009} for Russian, claimed that the \textsc{dat}--\textsc{acc} was the basic order, while \citet{bailyn1996,bailyn2010,bailyn2012} and \cite{antonyuk2015} argued for \textsc{acc}--\textsc{dat} as the basic order. The argumentation was based on such tests as genitive of negation, distributive \textit{po} constructions, binding of reciprocals, licensing of secondary predicates, idiom formation, focus propagation, and VP topicalization.\footnote{\label{fn18}We refer the Reader to the above-mentioned sources for a detailed discussion of these tests.}

\largerpage[2]
The second general approach was funded on the conviction that ditransitive verb constructions are derived from two basic underlying structures, one corresponding to English DOCs (V-/\textsc{acc}), as in \REF{ex:witkos:21} and the other to the so called \textit{to}-dative construction (V-\textsc{acc}- (to) \textsc{dat}), in (\REF{ex:witkos:22}, after \cite{dvorak2010}): 

% example 21
\ea  \label{ex:witkos:21} \minsp{[} \textsubscript{\textit{v}P} Jan showed [\textsubscript{ApplP} Maria.\textsc{dat} Appl\textsuperscript{0} [\textsubscript{VP} V\textsuperscript{0} her picture.\textsc{acc}$]]]$\\
\z

% example 22
\ea  \label{ex:witkos:22} \minsp{[} \textsubscript{\textit{v}P} Jan subordinated [\textsubscript{VP} a page.\textsc{acc} V\textsuperscript{0} [\textsubscript{PP} P\textsuperscript{0}(to) his knight.\textsc{dat}$]]]$\\
\z

\noindent In this regard, two positions can be outlined. One holds that particular verbs project only one of the two underlying structures and further alternative word order permutations operate on them \citep{dvorak2010}. Other authors argue that all benefactive/recipient verbs can appear with any underlying structures (\citealt{gracanin2006,marvinstegovec2012}). The criteria used in distinguishing between the two construction types involve the obligatory presence of the dative argument \REF{ex:witkos:24}, causative reading \REF{ex:witkos:25}, VP-topicalization, nominalisation, quantifier scope, and the two-goal construction. For lack of space, we illustrate only a few of these tests. Our Polish examples are based on the examples given in \cite{gracanin2006}, \cite{dvorak2010}, and \cite{marvinstegovec2012} for other languages.

% example 24
\ea \label{ex:witkos:24}
	\ea[]{ \label{ex:witkos:24a}
    \gll Jan wysłał \minsp{(} Marii) paczkę już wczoraj.\\
         Jan sent {} Maria.\textsc{dat} package.\textsc{acc} already yesterday\\
    \glt `Jan sent Maria a package yesterady.'
    }
	\ex[]{ \label{ex:witkos:24b}
    \gll Jan powierzył 	Marię \minsp{*} \minsp{(} jej patronowi) już 	wczoraj.\\
         Jan entrusted Maria.\textsc{acc} {} {} her patron.\textsc{dat} already yesterday\\
    \glt `Jan entrusted Maria to her patron yesterday'
    }
	\z
\z

% example 25
\ea \label{ex:witkos:25}
	\ea[]{ \label{ex:witkos:25a}
    \gll Beethoven dał światu Czwartą Symfonię.\\
         Beethoven gave world Fourth Symphony\\
    \glt `Beethoven gave the world the Fourth Symphony.'
    }
	\ex[\#]{ \label{ex:witkos:25b}
    \gll Beethoven 	dał Czwartą Symfonię światu.\\
         Beethoven 		gave 	Fourth 		Symphony 	world\\
    \glt `Beethoven gave the Fourth Symphony to the world.'
    }
	\z
\z

\noindent From our perspective, all the above mentioned ditransitive constructions show a crucial property, namely the superior object cannot function as an antecedent for the reflexive possessive in the other object; it can only antecede a pronominal possessive:\footnote{\label{fn20}It must be noted that this conclusion does not hold for all Slavic languages. For example, \cite{marvinstegovec2012} show that in Slovenian, a quantifier in the higher dative object can bind a reflexive possessive in the lower object, as in (i). 

\ea
\gll Tat$_2$ je vrnil [\hspace{-2pt} vsakemu oškodovancu]\textsubscript{1} [\hspace{-2pt} svoj\textsubscript{1,2} avto].\\
thief$_j$ \textsc{aux} return.\textsc{past} {} each.\textsc{dat} victim.\textsc{dat} {} his.\textsc{acc} car.\textsc{acc}\\
\glt `The thief returned every victim his car.'/ `The thief returned every victim his (the thief's) car.'
\z}

% example 28
\ea \label{ex:witkos:28}
	\ea \label{ex:witkos:28a}
    \gll Jan\textsubscript{1} pokazał Marii\textsubscript{2} \minsp{\{} swoje\textsubscript{1/*2} / jej\textsubscript{2}\} zdjęcie.\\
         Jan showed Maria.\textsc{dat} {} self.\textsc{poss} {} her picture.\textsc{acc}\\
    \glt `Jan showed Maria her picture.'
	\ex \label{ex:witkos:28b}
    \gll Jan\textsubscript{1} pokazał Marię\textsubscript{2} \minsp{\{} swojej\textsubscript{1/*2} / jej\textsubscript{2}\} przełożonej.\\
         Jan showed Maria.\textsc{acc} {} self.\textsc{poss} {} her supervisor.\textsc{dat}\\
    \glt `Jan showed Maria to her supervisor.'
    \ex \label{ex:witkos:28c}
    \gll Król\textsubscript{1} podporządkował giermka\textsubscript{2} \minsp{\{} swojemu\textsubscript{1/*2} / jego\textsubscript{2}\} rycerzowi.\\
         king subordinated page.\textsc{acc} {} self.\textsc{poss} {} his knight.\textsc{dat}\\
    \glt `The king subordinated the page to his knight.'
    \ex\label{ex:witkos:28d}
    \gll Król\textsubscript{1} podporządkował rycerzowi\textsubscript{2} \minsp{\{} swojego\textsubscript{1/*2} / jego\textsubscript{2}\} giermka.\\
         king 	subordinated knight.\textsc{dat} {} self.\textsc{poss} {} his page.\textsc{acc}\\
    \glt `The king subordinated his page to the knight.'
	\z
\z

\noindent This leads us to propose that in both patterns singled out, the Spell-Out form of the index in both constructions is determined by the fact that VP is not a reflexivization domain/site. So, both the accusative and the dative object of a regular ditransitive verb is placed too low in the structure to serve as a co-indexed antecedent for the index at the reflexivization site, defined as \textit{v}P or TP, but not VP, see positions [2] and [3] in \figref{fig:6} and \figref{fig:7}. As soon as the pronoun is not a co-argument of the object, IR applies and carries the index to the domain of \textit{v}P/TP, out of the c-domain of the object, so despite their coindexation, no Condition B violation occurs:

% example 29
\ea \label{ex:witkos:29}
$[$\textsubscript{\textit{v}P} Maria.\textsc{nom}\textsubscript{1} [index\textsubscript{2}-\textit{v} showed [\textsubscript{VP} Jan.\textsc{dat}\textsubscript{2} [\textsubscript{V’} V [index\textsubscript{2} [pictures$]]]]$\\
\z

\noindent Interestingly, the dative of possession seems to behave like a regular VP-internal dative object. Polish has a construction where the dative-marked nominal represents the thematic role of possessor, correctly captured in the English translation:

% example 30
\ea \label{ex:witkos:30}
	\ea{ \label{ex:witkos:30a}
    \gll Jan\textsubscript{1} złamał Tomkowi\textsubscript{2} \minsp{\{*} swoją\textsubscript{2} / jego\textsubscript{2}\} ulubioną kredkę.\\
         Jan.\textsc{nom} broke Tomek.\textsc{dat} {} self.\textsc{poss} {} his favourite colour-pencil.\textsc{acc}\\
    \glt `Jan broke Tomek’s favourite colour pencil.'
    }
	\ex{ \label{ex:witkos:30b}
    \gll Maria\textsubscript{1} wybiła {} Tomkowi\textsubscript{2} \minsp{\{*} swoją\textsubscript{2} / jego\textsubscript{2}\} nową złotą plombę.\\
         Maria.\textsc{nom} knocked out Tomek.\textsc{dat} {} self.\textsc{poss} {}  his new golden filling.\textsc{acc}\\
    \glt `Maria knocked out Tomek’s new golden filling.'
    }
	\z
\z
                        
\noindent The fact that only the possessive pronoun is the correct co-indexed bound form indicates that dative of possession is placed only as high as SpecVP, as NP\textsubscript{2} in \figref{fig:6}, and the index is raised to attach to \textit{v}/T, outside of its c-command domain:

% example 31
\ea \label{ex:witkos:31}
$[$\textsubscript{\textit{v}P} Jan\textsubscript{1} index\textsubscript{2}-\textit{v} broke [\textsubscript{VP} Tomek.\textsc{dat}\textsubscript{2} [\textsubscript{V$’$} V [\textsubscript{NP} index\textsubscript{2} [favourite pen$]]]]]$\\
\z

\noindent In general, if accepted, our analysis can be used as a detector for the position in which a given antecedent is placed with respect to heads \textit{v}/T; any antecedents placed below v, so within VP, are predicted not to be able to bind reflexive pronouns/reflexive possessives. 

In this context, consider an example of the impersonal passive construction with a dative argument:


\ea \label{ex:witkos:32}
\gll Marii pokazano \minsp{\{*} swoją / jej\} nową koleżankę.\\
     Maria.\textsc{dat} shown.\textsc{imprs} {} self.\textsc{poss} {} her new friend.\textsc{acc}\\
\glt `Maria was shown her new friend.'
\z
                
\noindent Despite the fact that the dative argument is placed in the left peripheral position in the clause and on many analyses, it occupies SpecTP, it can only function as antecedent to a pronominal possessive. We take this fact to indicate that the case position of this argument is really low, probably SpecVP, as any ordinary dative object of a ditransitive construction and its movement to T does not extend its binding domain, see \REF{ex:witkos:5b}.\footnote{\label{fn21}The same conclusion is reached in \cite{mooreperlmutter2000} for Russian impersonal passives.} At the same time, the word order in \REF{ex:witkos:32} does not convey any information-structure related message and it can be used as an answer to a general ‘what has happened?’ question. So a position in Spec\textsc{Top}P or Spec\textsc{Foc}P is not an option. We assume that the dative NP in \REF{ex:witkos:32} is either in SpecTP on account of checking only the [+EPP] property of T, which is not sufficient to extend its binding domain, or it is moved to a position that is technically an A-position but, crucially, not a case position, as proposed in \cite{germain2015} and \cite{citkoetal2018}:\footnote{\label{fn22}\cite{germain2015} proposes that conflicting characteristics of this position find a natural explanation if feature inheritance is split and the phase head C (Fin in her account where \citeposst{rizzi1997} split CP architecture is assumed \REF{split_cp_rus}. The head Fin passes on only φ-features to T but retains the [+EPP] property. Hence the nominative case can be valued under agree on the postverbal DP, while the non-nominative DP can move up to Spec\textsc{Fin}P to satisfy the EPP-property.

\ea\label{split_cp_rus}
{[\textsubscript{ForceP} Force [\textsubscript{TopP} Top [\textsubscript{FocP} Foc [\textsubscript{FinP} Fin]]]]}\\
\xspace\hfill (Russian left periphery; \citealt[428]{germain2015})
\z
}

% example 33
\ea \label{ex:witkos:33}
{[\textsubscript{TP/FinP} Maria.\textsc{dat} (Fin) [\textsubscript{TP} index-T shown [\textsubscript{VP} \sout{Maria}.\textsc{dat} [\textsubscript{V’} V [\textsubscript{NP} index [new friend]]]]]]}\\
\z

% SUB-SECTION 3.2
\subsection{The medial domain: Dative OEs in Spec\textit{v}P}\label{s3.2}

In this section we investigate (both verbal and non-verbal) dative OEs which bind anaphoric pronouns as co-arguments and optionally possessive reflexives as non-co-arguments. The successful antecedents to anaphoric pronouns are all placed in a clausal position higher than VP.

Psychological predicates with dative experiencers fall into two classes: non-verbal predicates and verbal ones. The chief source of differences between them in terms of binding properties of their dative arguments stems from the fact that only the latter allow for nominative T/SM (Target/Subject Matter) arguments and binding into these shows considerable speaker variation. 


\subsubsection{OEs in non-verbal psychological predicates}\label{s3.2.1}

We start with non-verbal psychological predicates such as \textit{było żal} ‘was sorrow’ or \textit{było wstyd} ‘was shame’. In \REF{ex:witkos:34}, the anaphoric/pronominal object (the index) is the object of the predicate \textit{żal}  ‘pity’, so a co-argument of the dative experiencer. IR carries it to the v-adjoined position and no further, see \REF{ex:witkos:8c}. This position is c-commanded by the dative NP:


\ea \label{ex:witkos:34}
	\ea{ \label{ex:witkos:34a}
    \gll Marii\textsubscript{1} było żal \minsp{\{} siebie\textsubscript{1} / \minsp{*?} jej\textsubscript{1}\} \minsp{(} samej).\\
         Maria.\textsc{dat} was.\textsc{3.sg.n} sorrow.\textsc{3.sg.m} {} self {} {} her {} alone\\
    \glt `Maria felt sorry for herself.'
    }
	\ex{ \label{ex:witkos:34b}
    $[$\textsubscript{\textit{v}P} Maria.\textsc{dat} [\textsubscript{\textit{v}’} index-\textit{v} was [ sorrow index$]]]$\\
    }
	\z
\z
                
\noindent In \REF{ex:witkos:35} the index is free to either head-adjoin to \textit{v} or move on to head-adjoin to T, as it is not a co-argument to Maria. In the former case, clause \REF{ex:witkos:8d} forces lexicalisation as reflexive, in the latter, clause \REF{ex:witkos:8e} forces lexicalisation as pronominal:

% example 35
\ea \label{ex:witkos:35}
	\ea{ \label{ex:witkos:35a}
    \gll Marii\textsubscript{1} było żal \minsp{\{} swojej\textsubscript{1} / jej\textsubscript{1}\} koleżanki.\\
         Maria.\textsc{dat} was.\textsc{3.sg.n} sorrow.\textsc{3.sg.m} {} self.\textsc{poss} {} her friend.\textsc{sg.f.gen}\\
    \glt `Maria felt sorry for her female friend.'
    }
	\ex{ \label{ex:witkos:35b}
    $[$\textsubscript{TP} index-T [\textsubscript{\textit{v}P} Maria.\textsc{dat} [\textsubscript{\textit{v}’} index-\textit{v} was [ sorrow [index friend].\textsc{gen}]]]]\\
    }
	\z
\z

\noindent The two derivations above markedly differ from equivalent constructions with nominative subjects and a corresponding verbal predicate \textit{żałować} ‘feel pity’ in a predictable manner:

% example 36
\ea \label{ex:witkos:36}
	\ea{ \label{ex:witkos:36a}
    \gll Maria\textsubscript{1} żałuje 	siebie\textsubscript{1} \minsp{(} samej).\\
         Maria.\textsc{nom} feels-pity self.\textsc{gen} {} alone\\
    \glt `Maria feels pity for herself.'
    }
	\ex{ \label{ex:witkos:36b}
    \gll Maria\textsubscript{1} żałuje \minsp{\{} swojej\textsubscript{1} / \minsp{*} jej\textsubscript{1}\} koleżanki.\\
         Maria.\textsc{nom} feels-pity {} self.\textsc{poss} {} {} her friend\\
    \glt `Maria feels pity for her friend.'
    }
	\z
\z
                        
\noindent In \REF{ex:witkos:36a}--\REF{ex:witkos:36b}, the subject occupies SpecTP, the highest A-position in the clause, so in both corresponding derivations the index must be spelled-out as a reflexive: 

% example 37
\ea \label{ex:witkos:37}
	\ea\label{ex:witkos:37a}
    {[\textsubscript{TP} Maria.\textsc{nom} index-T [\textsubscript{\textit{v}P} Maria.\textsc{nom} [\textsubscript{\textit{v}’} index-\textit{v} [ feels-pity index]]]}\\
	\ex \label{ex:witkos:37b}
    {[\textsubscript{TP} Maria.\textsc{nom} index-T [\textsubscript{\textit{v}P} Maria.\textsc{nom} [\textsubscript{\textit{v}’} index-\textit{v} [ feels-pity [index friend]]]}\\
	\z
\z
                
\noindent Furthermore, other verbs with dative OEs can function as antecedents to both reflexive pronouns and reflexive possessives, for instance in selected PPs:

% example 38
\ea \label{ex:witkos:38}{
\gll Marii nudziło się w \minsp{\{} swoim / jej\} domu.\\
     Maria.\textsc{dat} bored.\textsc{3.sg.neut} \textsc{refl} in {} self’s {} her home.\textsc{loc}\\
\glt `Maria was bored at home.'
}
\z
                
\noindent Additionally, other factors support the idea of a higher placement in the clausal structure of dative OEs in comparison with dative goals/benefactives, for instance the applicative characteristics that the predicates with dative OEs display. \cite{cuervo2003} argues that the dative argument of \textit{gustar}, the Spanish equivalent to \textit{podobać się}  ‘appeal to’ seems to be licensed by a high applicative in the sense of \cite{pylkkanen2002}. This is because the nominative argument is not involved with it in any relation of possession or location which are typical of ‘low’ applicatives in \REF{ex:witkos:39a}--\REF{ex:witkos:39b}, where Maria’s habitual possession of the pen is implied, so \REF{ex:witkos:39b} can only mean that the pen was Mark’s and it was in Maria’s  possession only temporarily. No possession or location is implied in \REF{ex:witkos:40}, where the fancy of the pen (whosever it is) has overcome Maria:

% example 39
\ea \label{ex:witkos:39}
	\ea[]{ \label{ex:witkos:39a}
    \gll Marii złamał się długopis.\\
         Maria.\textsc{dat} broke \textsc{refl} pen.\textsc{nom}\\
    \glt `Maria broke a pen.'
    }
	\ex[\#]{ \label{ex:witkos:39b}
    \gll Marii złamał się długopis Marka.\\
         Maria.\textsc{dat} broke \textsc{refl} pen.\textsc{nom} Mark.\textsc{gen}\\
    \glt `Maria broke Mark’s pen.'
    }
	\z
\z

% example 40
\ea \label{ex:witkos:40}
	\ea{ \label{ex:witkos:40a}
    \gll Marii podoba się długopis Marka.\\
         Maria.\textsc{dat} appeal \textsc{refl} pen.\textsc{nom} Mark.\textsc{gen}\\
    \glt `Mark’s pen appeals to Maria.'
    }
	\ex{ \label{ex:witkos:40b}
    \gll Marii było żal długopisu Marka.\\
         Maria.\textsc{dat} was.\textsc{3.sg.neut} sorrow pen Marek.\textsc{gen}\\
    \glt `Maria was sorry about Mark’s pen.'
    }
	\z
\z
                        
\noindent In the context of the solution proposed here, the dative argument corresponding to the ‘high’ applicative is placed in Spec\textit{v}P, while the one in the ‘low’ applicative in SpecVP. It appears that  \REF{ex:witkos:40} stands apart from \REF{ex:witkos:39}.

It is therefore more worrying that at first glance the binding capacity of verbal psychological predicates runs counter to what has been presented thus far.

% SUB-SUB-SECTION 3.2.2
\subsubsection{Verbal psychological predicates and the idiosyncrasy of \textit{podobać się} ’appeal to’} \label{s3.2.2}

The most frequently researched verbal psychological predicate in Polish is \textit{podobać się} ‘appeal to’ (see \citealt{miechowicz2008,bondaruk2007,zychlinski2013,jimenezfernandez2016,bondaruketal2017}). Its distinctive property is the fact that it selects for the experiencer in dative and the cause/target of emotion in nominative. It has also been noticed that the binding potential of its dative-marked argument differs from the dative of non-verbal psych-predicates:\footnote{\label{fn23}\cite[253]{franks1995} observes this contrast for Russian:
\ea[]{
\gll Mne zal’ sebja.\\
me.\textsc{dat} sorry self.\textsc{acc}\\
\glt Intended: `I feel sorry for myself'}
\z
\ea[*]{
\gll Mne nadoedaet svoj ucebnik\\
me.\textsc{dat} bore.\textsc{3.sg} self’s textbook.\textsc{nom}\\
\glt Intended: `My textbook bores me.'}
\z
\ea[*]{
\gll Mne dosazdaet svoj brat\\
me.\textsc{dat} vex.\textsc{3.sg} self’s brother\\
\glt `My brother vexes me.'}
\z}

% example 41
\ea \label{ex:witkos:41}
\gll Maria\textsubscript{1} podobała się sobie\textsubscript{1} w lustrze.\\
 Maria.\textsc{nom} appealed \textsc{refl} self.\textsc{dat} in mirror\\
\glt `Maria appealed to herself in the mirror.'
\z

% example 42
\ea \label{ex:witkos:42}
\gll Maria\textsubscript{1} podobała się \minsp{\{} swojemu\textsubscript{1} / \minsp{*} jej\textsubscript{1}\} koledze z ławki.\\
 Maria.\textsc{nom} appealed \textsc{refl} {} self.\textsc{poss} {} {} her friend.\textsc{dat} from school.desk\\
\glt `Maria appealed to her school desk friend.'
\z

% example 43
\ea \label{ex:witkos:43}
\gll Marii\textsubscript{1} spodobała się nawet \minsp{?\%} ona\textsubscript{1} \minsp{\{(} sama) / \minsp{*} sobie\textsubscript{1}\} (sama) w lustrze.\\
     Maria.\textsc{dat} appealed \textsc{refl} even {} she.\textsc{nom} {} alone {} {} self.\textsc{nom} alone in mirror\\
\glt `Even herself in the mirror appealed to Maria.'
\z

% example 44
\ea \label{ex:witkos:44}
\gll Marii\textsubscript{1} podobał się \minsp{\{*} swój\textsubscript{1} / jej\textsubscript{1}\} kolega z ławki.\\
     Maria.\textsc{dat} liked \textsc{refl} {} self’s {} her friend.\textsc{nom} from school.desk\\
\glt `Her school desk friend appealed to Maria.'
\z

% example 45
\ea \label{ex:witkos:45}
\gll Janowi\textsubscript{1} spodobały się listy od \minsp{\{} swoich\textsubscript{1} / jego\textsubscript{1}\} fanek\\
     Jan.\textsc{dat} liked \textsc{refl} letters from {} self’s {} his fans\\
\glt `Letters from his fans appealed to Jan.'
\z

% example 46
\ea \label{ex:witkos:46}
\gll Mi\textsubscript{1} się swój\textsubscript{1} głos podoba.\\
     Me \textsc{refl} my voice like\\
\glt `I like my voice' \\ 
\xspace\hfill(\citealt[107, ex. (62)]{miechowicz2008}; corpus search)
\z

\noindent In general, if the dative Experiencer of a psychological verb is in Spec\textit{v}P, we expect to see optionality with the pronominal vs reflexive possessive, similar to that with non-verbal psych-predicates, but this is not the case.\footnote{\label{fn24}Other similar verbs in Polish are \textit{dokuczać} ‘tease, vex’, \textit{nudzić} ‘bore’, and \textit{szkodzić} ‘harm’:

\ea
\gll Marii$_1$ dokuczał \minsp{\{*} swój$_1$ / jej$_1$\} kolega z ławki.\\
Maria.\textsc{dat} teased {} self’s  {} her friend.\textsc{nom} from school.desk\\
\glt `Her school desk friend teased to Maria.’
\z

\ea
\gll Janowi$_1$ szkodziły listy od \minsp{\{?} swoich$_1$ / jego$_1$\} fanek.\\
Jan.\textsc{dat} harmed letters from {} self’s {} his fans\\
\glt `Letters from his fans harmed Jan.’
\z} The data pie can be partitioned into three uneven sections. Most native speakers asked for judgements on the data prefer for the dative Experiencer to bind pronominal possessives, see \REF{ex:witkos:44}. Quite a few allow the dative Experiencer to bind a possessive reflexive but only in contexts where the reflexive is embedded deep in the nominative constituent and bears a different case. 

These inconclusive data lead to conflicting views on the position of the dative OE and the position of the nominative argument. \cite{jimenezfernandez2016} assume that its A-position is in SpecVP (and the preverbal position is in an articulated CP area). \citet{bondaruk2007}, \citet{tajsner2008}, and \citet{bondaruketal2017} propose that the dative experiencer is in Spec\textit{v}P, as its binding scope is different from dative goals. \cite{miechowicz2008} claim that the dative Experiencer reaches as high as SpecTP:

For the sake of concreteness, we assume that the dative Experiencer occupies the position of Spec\textit{v}P, though this view is not uncontroversial:\footnote{\label{fn25}For instance, \cite{cuervo2003} argues strongly for the view that the dative OE is placed in a higher position, as the nominative Theme occupies Spec\textit{v}P:

\ea
$[$\textsubscript{TP} index$_i$-T $[$\textsubscript{ApplP} \textsc{dat} $[$\textsubscript{Appl$'$}  \textsc{Appl}\textsuperscript{0} $[$\textsubscript{\textit{v}P} index $[$\textsubscript{\textit{v}$’$} \textit{v}\textsuperscript{0} $[$ \textsc{nom} index$][$\textsubscript{\textit{v}’} \textit{v}-\textsc{be} $[$\textsubscript{VP} \textsc{psych verb}$]]]]]]]$
\z

\noindent We cannot discuss this issue in full for lack of space.}  

% example 49
\ea \label{ex:witkos:49}
$[$\textsubscript{TP} index\textsubscript{i}-T $[$\textsubscript{\textit{v}P}  \textsc{dat}\textsubscript{i} $[$\textsubscript{\textit{v}$'$} index\textsubscript{i}-\textit{v} $[$\textsubscript{VP} V $[$\textsc{nom} index\textsubscript{i} N\textsubscript{k}$]]]]]$\\
\z

\noindent Let us tentatively assume that the structure in \REF{ex:witkos:49} is correct for the Polish \textit{podobać się} `appeal to', with the nominative theme argument optionally raised to its case position in SpecTP in overt syntax.\footnote{\label{fn26}An analogous structure is proposed in \cite{klimek2004}.} The index is c-commanded by the dative OE in the \textit{v}-adjoined position (corresponding to position [2] in \figref{fig:7}) but it is not so c-commanded when placed in the T-adjoined position (corresponding to position [3] in \figref{fig:7}). 

With this general idea in mind, the derivations of \REF{ex:witkos:41}--\REF{ex:witkos:45} look as follows, with the dative experiencer fronted to an A-position in line with \cite{germain2015}, see \REF{ex:witkos:44}:

% example 50
\ea \label{ex:witkos:50}
$[$\textsubscript{FinP} Maria.\textsc{dat} Fin [\textsubscript{TP} index-T [\textsubscript{\textit{v}P} \sout{Maria}.\textsc{dat} [\textsubscript{\textit{v}’} index-\textit{v} [\textsubscript{VP} appealed \textsc{refl}-V [even she.\textsc{nom}/*self\textsubscript{*case} (self/alone)] in mirror$]]]]]$\\
\z

% example 51
\ea \label{ex:witkos:51}
$[$\textsubscript{FinP} Maria.\textsc{dat} Fin [\textsubscript{TP} index-T [\textsubscript{\textit{v}P} \sout{Maria}.\textsc{dat} [\textit{v}’ index-\textit{v} [\textsubscript{VP} appealed \textsc{refl}-V [?*self’s/her school friend$]]]]]$\\
\z

% example 52
\ea \label{ex:witkos:52}
$[$\textsubscript{FinP} Jan.\textsc{dat} Fin [\textsubscript{TP} index-T [\textsubscript{\textit{v}P} \sout{Jan}.\textsc{dat} index-\textit{v} [\textsubscript{VP} appealed \textsc{refl}-V [\textsubscript{NP} [\textsubscript{NP} letters] [\textsubscript{PP} from [\textsubscript{NP} self’s/his fans$]]]]]$.\\
\z

\largerpage[-1]
\noindent \REF{ex:witkos:36} is easy to deal with, as Polish has no nominative reflexive pronouns. Due to lack of this form in the morphological paradigm, its closest equivalent is selected, in line with \citeposst{safir2004} \textsc{form to interpretation principle} (FTIP).\footnote{\label{fn27}FTIP, \cite{safir2004}: If: 
\begin{itemize}
    \item[a.] X c-commands position Y,
    \item[b.] z is the lexical form or string that fills Y,
    \item[c.] w is a single form more dependent than z,
    \item[d.] both w and z could support the same identity-dependent interpretation if Y were exhaustively dependent on X, then (the referential value for) Y cannot be interpreted as identity dependent on X.\end{itemize}}

\REF{ex:witkos:51} seems to be a problem indeed, but an unacceptable status of the reflexive possessive can be credited to what \cite[26]{rizzi1990} calls the \textsc{anaphor agreement effect} (AAE):\footnote{\cite[32--33]{rizzi1990} reports the following contrast in Italian: a dative experiencer can bind an anaphor as long as it is not nominative, so since \textit{importare} ‘matter’ takes a genitive theme, this theme can be bound, while a nominative argument of \textit{interessare} ‘matter’ cannot:

% example 54
%\ea \label{ex:witkos:54}
\ea[]{ \label{ex:witkos:54a}
\gll A loro importa solo di {se stessi}.\\
     to them matters only of themselves\\
\glt `They matter only to themselves.'}
\z

\ea[*]{ \label{ex:witkos:54b}
\gll A loro interessano solo {se stessi}.\\
    to them interest only themselves\\
\glt Intended: `They have interest only in themselves.'}
\z
%\z
                        
\noindent Significantly, however, dative experiencers can function as binders once the AAE is controlled for, as in \REF{ex:witkos:54a}. The same picture obtains with Polish dative experiencers above.}

% example 53
\ea \label{ex:witkos:53}
Anaphors do not occur in syntactic positions construed with agreement.\\
\z


Our discussion of Polish data reveals that consequences of the AAE are subject to considerable language variation: nominative reflexive possessives are typically avoided by most speakers, although they are construed with agreement only indirectly: they agree (in case and φ-features) with NP they modify and this NP agrees with the auxiliary/verb. Yet, the structure we propose for pronominal possessive NPs is shown in \REF{ex:witkos:57}. It is only natural to extend it to cases of reflexive possessives:\footnote{\label{fn29}Note that the structure in \REF{ex:witkos:57} with a pronominal possessive is much less ambiguous than the one with the reflexive in \REF{ex:witkos:58} on account of the pronominal possessive bearing a different case (genitive) from the nominative of the NP it modifies.} 

% example 57
\ea \label{ex:witkos:57}{
\gll $[$\textsubscript{NP} jego [\textsubscript{NP}  siostra$]]$\\
     {} his.\textsc{gen} {} sister.\textsc{nom}\\
\glt `his sister'
}
\z

% example 58
\ea \label{ex:witkos:58}{
\gll {[}\textsubscript{NP} swoja [\textsubscript{NP} siostra$]]$\\
     {} self.\textsc{poss} {} sister.\textsc{nom}\\
\glt `self's sister'
}
\z
    
\noindent This structure may be quite ambiguous when the AAE applies, as the possessive element is equidistant to T with the NP it modifies.

% example 59
\ea \label{ex:witkos:59}{
\gll Janowi nie spodobała się \minsp{\{*} swoja / jego\} siostra.\\
     Jan.\textsc{dat} not appleal \textsc{refl} {} self.\textsc{poss} {} his sister.\textsc{nom}\\
\glt `His sister did not appeal to Jan.'
}
\z

% example 60
\ea \label{ex:witkos:60}
T.\textsc{agr}\textsubscript{2/1} {\dots} Jan.\textsc{dat}\textsubscript{1} {\dots} [\textsubscript{NP} self.\textsc{nom}\textsubscript{1} [\textsubscript{NP} sister.\textsc{nom}\textsubscript{2}$]]$\\
\z

\noindent The equidistance relationship in question may cause confusion as to what really agrees with Infl/T here, the modified NP (with no consequence for the AAE) or the possessive reflexive (violating the AAE in \REF{ex:witkos:58} and \REF{ex:witkos:59} above). In the latter case, from the perspective of the Binding Principles, the possessive forces its referential subscript to represent the subscript of the entire NP.\footnote{\label{fn31}In languages where possessives are genuine specifiers rather than adjuncts, possessive reflexives are allowed, as shown in \cite[273--274]{woolford1999}.} Now, this is quite similar to what \cite[109--111]{landau2000} observes for cases of Obligatory Control, where the controller (unexpectedly) does not c-command PRO but constitutes the specifier of a c-commanding DP:

% example 61
\ea \label{ex:witkos:61}
It would help Bill’s\textsubscript{1} development [PRO\textsubscript{1} to behave himself\textsubscript{1} in public]\\
\z

\noindent Landau proposes that a well-defined class of nouns denoting abstract notions reflecting the individuality of the controller ([X’s NP]) allows for what he calls the logophoric extension of X:

% example 62
\ea \label{ex:witkos:62}
For the purpose of control, a logophoric extension [X’s NP] is non-distinct from X: [X’s\textsubscript{1} NP] → [X’s NP]\textsubscript{1}.\\
\z

\noindent Thus, logophoric extension is a selective process that affects only one module of grammar and one aspect of interpretation: Control Theory. We would like to submit that an analogous process of reanalysis affects the adjunct/specifier structure:

% example 63
\ea \label{ex:witkos:63}
\textit{Extended AAE:}\\
Anaphors do not occur in syntactic positions construed with agreement directly (a) or indirectly (b):\\
\begin{itemize}
    \item[a.] Nominative anaphors do not exist in languages showing subject/verb agreement;\\
    \item[b.] For the purpose of binding, an indexical extension [X’s NP] is non-distinct from X:\\
    $[$\textsubscript{NP2} swój.\textsc{nom}\textsubscript{1} [\textsubscript{NP2} name.\textsc{nom}\textsubscript{2}$]]$ → [\textsubscript{NP1} swój.\textsc{nom}\textsubscript{1} [\textsubscript{NP1} name.\textsc{nom}\textsubscript{2}$]]$\\
\end{itemize}
\z

\noindent Our notion of indexical extension differs from Landau’s original on two counts: first, it is not limited by the semantic (sub)class of N and second, it depends on the structural position of X, which we have shown to act as an adjunct, following \cite{despic2011,despic2013,despic2015}.

But this does not seem to be enough to cover the whole spectrum of the data. First, notice that index extension may be less local in the cases discussed by Landau. For instance, the controller for Obligatory Control PRO can also be placed in a position embedded in a measure NP selecting for the ‘logophoric NP’:
    
% example 64
\ea[?]{
It considerably helped [\textsubscript{NP\textsuperscript{1}} first stages of [\textsubscript{NP\textsuperscript{2}} her\textsubscript{1} music career]] [PRO\textsubscript{1} to have an uncle in a record company]\\ \label{ex:witkos:64}}
\z

\noindent So, it seems that (at least for some speakers) X from \REF{ex:witkos:64} need not be very close to the edge of the NP to propagate its index to the maximal NP (here NP\textsuperscript{1}).

Once we allow for the less local propagation of the index in the cases of indexical extension in definition \REF{ex:witkos:63} above, we can account for \REF{ex:witkos:4} above on the assumption that the rule of the Extended AAE is subject to graded speaker variation:
    
% example 65
\ea[*]{
antecedent\textsubscript{i} {\dots} [\textsubscript{AgrP} anaphor\textsubscript{i} agreement\textsubscript{i}{\dots}]\\ \label{ex:witkos:65}}
\z

% example 66
\ea \label{ex:witkos:66}
For the purpose of binding, an indexical extension [X’s NP] is non-distinct from X:\\
$[$\textsubscript{NP2} self.\textsc{nom}\textsubscript{1} [\textsubscript{NP2} name.\textsc{nom}\textsubscript{2}$]]$ → [\textsubscript{NP1} self.\textsc{nom}\textsubscript{1} [\textsubscript{NP1} name.\textsc{nom}\textsubscript{2}$]]$\\
\z

% example 67
\ea \label{ex:witkos:67}
For the purpose of binding, an indexical extension [X’s NP] is non-distinct from X:\\
$[$\textsubscript{NP3} N\textsubscript{3}{\dots} [\textsubscript{NP2} self.\textsc{nom}\textsubscript{1} [\textsubscript{NP2} name.\textsc{nom}\textsubscript{2}$]]]$ → [\textsubscript{NP3} N1{\dots} [\textsubscript{NP1/2} swój.\textsc{nom}\textsubscript{1} [\textsubscript{NP1/2} name.\textsc{nom}\textsubscript{2}$]]]$\\
\z

\noindent All speakers of Polish have \REF{ex:witkos:65} in their grammars, most speakers have \REF{ex:witkos:66} as a part of their grammars and exclude nominative reflexive possessives as a result of indexical extension, while the most conservative ones allow for non-local indexical extension and disallow reflexive possessives in cases other than nominative if they are embedded in nominative NPs, see \REF{ex:witkos:67}.\footnote{\label{fn34}A similar effect arises for the ACE. \cite[82]{willim1989} reports that the following example is problematic, though many native speakers accept it as only mildly deviant:

\ea [\%]{
\gll Ta recenzja książki mojego brata$_1$ zupełnie go$_1$ załamała.\\
this review.\textsc{nom} of-book my brother’s completely him.\textsc{acc} devastated\\
\glt `This review of my brother’s book devastated him completely.’
}
\z

\noindent A reviewer for this volume raises the issue of how the propagation of the index can be constrained. We presume that it is a matter of speaker variation but the extent of the propagation is difficult to gauge on accout of processing difficulties. Certainly, this issue deserves further empirical study.}

% SECTION 4
\section{Concluding remarks}\label{sec:witkos:s4}

In the process of our investigation, we have raised a number of questions with respect to the data in \REF{ex:witkos:1}--\REF{ex:witkos:4}. We conclude that, universally, there is one D-bound/index which is the most dependent form bound locally and lexicalized in two different forms: reflexive and pronominal, determined by IR. When the co-agreeing NP locally c-commands D-bound/ index in its landing site at \textit{v}/T, it is spelled out as a reflexive form, otherwise it is spelled out as a pronoun. The chain of Index Raising exhibits copy pronunciation, i.e. the tail of the chain is pronounced. In Polish (Slavic), IR is driven by the need to compensate for impoverished [+person] feature on the D-bound/index. The subject orientation of reflexive pronouns and possessive reflexives comes out rather naturally in this account. As IR places the index in these positions, it is not surprising that pronouns and anaphors show complementary distribution only with respect to the subject but not the object. The picture becomes even more transparent when we take into account the distinction between co-argument and non-co-argument reflexivization, see \REF{ex:witkos:8c} vs. \REF{ex:witkos:8d}--\REF{ex:witkos:8e}. The non-co-argument index covertly raised beyond the c-domain of the object is predicted to be spelled out as a pronominal possessive, although it is co-indexed with the object c-commanding it in overt syntax. We have shown that successful binders of reflexives and reflexive possessives in Polish need not occupy the position of SpecTP, which is reserved for nominative subjects only. Dative OEs occupy a lower position of Spec\textit{v}P.\footnote{\label{fn36}For arguments to this effect also see \cite{citkoetal2018}.} In view of the scope and reach of IR, these elements can bind and be co-indexed with reflexives and reflexive possessives adjoined to \textit{v} but they can also be co-indexed with pronominal possessives adjoined to T. The optionality of possessive forms co-indexed with them is thus explained.\footnote{\label{fn37}Introduction of more structural content which does not block IR, specifically PRO and infinitive T with raising and control constructions, multiplies reflexivization sites and provides for more spell-out options for the index:

\ea
\gll Maria$_1$ kazała Piotrowi$_2$ patrzeć na \minsp{\{} siebie$_1$$_,$$_2$ / \minsp{*} niego$_2$ / nią$_1$\}.\\
Maria.\textsc{nom} told Piotr.\textsc{dat} look.\textsc{inf} at {} self {} {} him / her\\
\glt `Maria told Piotr to look at himself/her.’
\z
        
\noindent The infinitive complement in (i) constitutes a binding domain for a co-argument index of PRO. Hence the co-argument must stop moving at the \textit{v}P level and is spelled-out as a reflexive (this is interpretation with index$_2$, as PRO is controlled by the object). The pronoun co-indexed with PRO is clearly impossible here (*niego$_2$). However, a considerable number of cliticization sites implies that the index co-indexed with the subject of the control predicate has a few options and can be spelled out as either a reflexive (siebie$_1$’self) when it is raised to matrix \textit{v} or T, or as a pronoun (nią$_1$’her’) when it cliticizes to embedded \textit{v} or embedded T:


\ea
$[$\textsubscript{TP} Maria$_1$ index$_1$-T [\textsubscript{\textit{v}P} index$_1$-v-told [\textsubscript{VP} Piotr.\textsc{dat}\textsubscript{2} [\textsubscript{V’} V [\textsubscript{CP} [\textsubscript{TP} PRO$_2$ index$_1$-T [\textsubscript{\textit{v}P} PRO index\textsubscript{1,2}-v-look [\textsubscript{VP} V [\textsubscript{PP} at index$]]]]]]$
\z

\noindent For a more thorough discussion of the issues discussed in this contribution, see \cite{witkosetal_forth}.} We have also credited imperfect results of dative OEs binding into nominative themes to Extended AAE of \cite{rizzi1990} caused by a specific placement of possessives as adjuncts at the edge of the nominative NP. Such placement leads to ambiguity of representation and the probe/goal relations involving T and NP.\textsc{nom}.

% Just uncomment the input below when you're ready to go.

%\input{example-osl.tex}

\section*{Abbreviations}

\begin{tabularx}{.55\textwidth}{@{}lX@{}}
\textsc{1/2/3}&first/second/third person\\
AAE&anaphor agreement effect\\
\textsc{acc}&{accusative}\\
ACE&anti-cataphora effects\\
\textsc{aux}&auxiliary\\
\textsc{cl}&{clitic}\\
\textsc{cop}&{copula}\\
\textsc{dat}&{dative}\\
DOC&double object construction\\
\textsc{f}&feminine\\
FTIP&form to interpretation principle\\
\textsc{gen}&{genitive}\\
\textsc{imprs}&impersonal\\
\end{tabularx}%
\begin{tabularx}{.40\textwidth}{@{}lX@{}}
\textsc{inf}&{infinitive}\\
IR&index raising\\
\textsc{loc}&{locative}\\
\textsc{neut}&{neuter}\\
\textsc{nom}&{nominative}\\
OE&object experiencer\\
\textsc{past}&past tense\\
\textsc{poss}&possessive\\
\textsc{pres}&{present tense}\\
\textsc{refl}&{reflexive clitic}\\
\textsc{sg}&singular\\
SC&Serbo-Croatian\\
&\\
\end{tabularx}

\section*{Acknowledgements}
This publication is funded by grant no 2014/15/G/HS2/04715 of the Polish National Science Centre.

\sloppy
\printbibliography[heading=subbibliography,notkeyword=this]

\end{document}
