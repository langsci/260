\documentclass[output=paper,colorlinks,citecolor=brown]{langsci/langscibook}
\ChapterDOI{10.5281/zenodo.3764861}
%\bibliography{localbibliography}
%% add all extra packages you need to load to this file  
\usepackage{tabularx} 
\definecolor{lsDOIGray}{cmyk}{0,0,0,0.45}

\usepackage{xassoccnt}
\newcounter{realpage}
\DeclareAssociatedCounters{page}{realpage}
\AtBeginDocument{%
  \stepcounter{realpage}
}

%%%%%%%%%%%%%%%%%%%%%%%%%%%%%%%%%%%%%%%%%%%%%%%%%%%%
%%%           Examples                           %%%
%%%%%%%%%%%%%%%%%%%%%%%%%%%%%%%%%%%%%%%%%%%%%%%%%%%%  
%% if you want the source line of examples to be in italics, uncomment the following line
% \renewcommand{\exfont}{\itshape}
\usepackage{lipsum}
\usepackage{langsci-optional}
\usepackage{./langsci-osl}
\usepackage{langsci-lgr}
\usepackage{langsci-gb4e}
\usepackage{stmaryrd}
\usepackage{pifont} % needed for checkmark \ding{51} and cross \ding{55}
\usepackage[linguistics]{forest}

% ch06
\usepackage[euler]{textgreek}
% ch07
\usepackage{soul}
\usepackage{graphicx}
% ch08, ch14
\usepackage{multicol}
% ch11
\usepackage{fnpct}
% ch13
\usepackage{scrextend}
\usepackage{enumitem}
% ch14
\usepackage{tabto}
\usepackage{multirow}

\usepackage{langsci-cgloss}

%\newcommand{\smiley}{ :) }

% non-italics in examples

\renewcommand{\eachwordone}{\upshape}

% non-italics in examples in footnotes

\renewcommand{\fnexfont}{\footnotesize\upshape}
\renewcommand{\fnglossfont}{\footnotesize\upshape}
\renewcommand{\fntransfont}{\footnotesize\upshape}
\renewcommand{\fnexnrfont}{\fnexfont\upshape}

% chapter03 goncharov

\newcommand{\p}{\textsc{pfv\ }}
\newcommand{\im}{\textsc{ipfv\ }}

\makeatletter
\let\thetitle\@title
\let\theauthor\@author 
\makeatother

\newcommand{\togglepaper}[1][0]{  
  \addbibresource{../localbibliography.bib}  
  \papernote{\scriptsize\normalfont
    \theauthor.
    \thetitle. 
    To appear in: 
    Change Volume Editor \& in localcommands.tex 
    Change volume title in localcommands.tex
    Berlin: Language Science Press. [preliminary page numbering]
  }
  \pagenumbering{roman}
  \setcounter{chapter}{#1}
  \addtocounter{chapter}{-1}
}

\providecommand{\orcid}[1]{}
\IfFileExists{../localcommands.tex}{
  % add all extra packages you need to load to this file  
\usepackage{tabularx} 
\definecolor{lsDOIGray}{cmyk}{0,0,0,0.45}

\usepackage{xassoccnt}
\newcounter{realpage}
\DeclareAssociatedCounters{page}{realpage}
\AtBeginDocument{%
  \stepcounter{realpage}
}

%%%%%%%%%%%%%%%%%%%%%%%%%%%%%%%%%%%%%%%%%%%%%%%%%%%%
%%%           Examples                           %%%
%%%%%%%%%%%%%%%%%%%%%%%%%%%%%%%%%%%%%%%%%%%%%%%%%%%%  
%% if you want the source line of examples to be in italics, uncomment the following line
% \renewcommand{\exfont}{\itshape}
\usepackage{lipsum}
\usepackage{langsci-optional}
\usepackage{./langsci-osl}
\usepackage{langsci-lgr}
\usepackage{langsci-gb4e}
\usepackage{stmaryrd}
\usepackage{pifont} % needed for checkmark \ding{51} and cross \ding{55}
\usepackage[linguistics]{forest}

% ch06
\usepackage[euler]{textgreek}
% ch07
\usepackage{soul}
\usepackage{graphicx}
% ch08, ch14
\usepackage{multicol}
% ch11
\usepackage{fnpct}
% ch13
\usepackage{scrextend}
\usepackage{enumitem}
% ch14
\usepackage{tabto}
\usepackage{multirow}

\usepackage{langsci-cgloss}

  \newcommand{\smiley}{ :) }

% non-italics in examples

\renewcommand{\eachwordone}{\upshape}

% non-italics in examples in footnotes

\renewcommand{\fnexfont}{\footnotesize\upshape}
\renewcommand{\fnglossfont}{\footnotesize\upshape}
\renewcommand{\fntransfont}{\footnotesize\upshape}
\renewcommand{\fnexnrfont}{\fnexfont\upshape}

% chapter03 goncharov

\newcommand{\p}{\textsc{pfv\ }}
\newcommand{\im}{\textsc{ipfv\ }}

\makeatletter
\let\thetitle\@title
\let\theauthor\@author 
\makeatother

\newcommand{\togglepaper}[1][0]{  
  \addbibresource{../localbibliography.bib}  
  \papernote{\scriptsize\normalfont
    \theauthor.
    \thetitle. 
    To appear in: 
    Change Volume Editor \& in localcommands.tex 
    Change volume title in localcommands.tex
    Berlin: Language Science Press. [preliminary page numbering]
  }
  \pagenumbering{roman}
  \setcounter{chapter}{#1}
  \addtocounter{chapter}{-1}
}

\providecommand{\orcid}[1]{}
  %% hyphenation points for line breaks
%% Normally, automatic hyphenation in LaTeX is very good
%% If a word is mis-hyphenated, add it to this file
%%
%% add information to TeX file before \begin{document} with:
%% %% hyphenation points for line breaks
%% Normally, automatic hyphenation in LaTeX is very good
%% If a word is mis-hyphenated, add it to this file
%%
%% add information to TeX file before \begin{document} with:
%% %% hyphenation points for line breaks
%% Normally, automatic hyphenation in LaTeX is very good
%% If a word is mis-hyphenated, add it to this file
%%
%% add information to TeX file before \begin{document} with:
%% \include{localhyphenation}
\hyphenation{
Ro-ma-no-va
Isa-čen-ko
}

\hyphenation{
Ro-ma-no-va
Isa-čen-ko
}

\hyphenation{
Ro-ma-no-va
Isa-čen-ko
}

  \togglepaper[10]%%chapternumber
}{}
%\togglepaper[10]


\author{Vesna Plesničar\affiliation{University of Nova Gorica}}
\title{Complementizer doubling in Slovenian subordinate clauses}
\abstract{The focus of the present paper is on complementizer doubling constructions in subordinate clauses in Slovenian. The main goal is to show that complementizer doubling in Slovenian is a syntactic phenomenon comparable to complementizer doubling in other, mainly dialectal variants of Romance languages (e.g. \citealt{Paoli2003, ledgeway2005, dagnac2012, villa2012, GiP2014, munaro2016}). The Slovenian complementizer doubling data strongly suggests that the syntactic analysis of such constructions is possible only under the assumption that the complementizer field is split into several functional projections, as was first proposed by \cite{rizzi1997}. Since it seems that the doubling complementizer in Slovenian is always the closing element of the complementizer system, it is reasonable to assume that at least in Slovenian, this element occupies the head of finiteness projection, while the first complementizer in complementizer doubling constructions, which functions as the complement clause introducer, sits in the head of the highest projection of the split CP field, i.e. the force projection. The suitability of force projection as the host of the first complementizer in Slovenian complementizer doubling constructions is justified by the fact that topicalized and focalized phrases necessarily follow it, which is the exact same pattern that was observed also for complementizer doubling constructions in Romance languages (e.g. \citealt{ledgeway2005, dagnac2012, munaro2016}, among others).


\keywords{subordinate complementizer, complementizer phrase, subordinate clause, complementizer doubling, split CP hypothesis}
}
%kjer je Villa-Garcia 2012 2013 - v lit je samo 2012, a je tu mišljeno kaj drugega



\begin{document}
\maketitle
\il{Slovenian|(}
\section{Introduction}
\isi{Complementizer doubling} is a phenomenon in which a clause contains two complementizers, of which the first is the top-most element of the \isi{subordinate clause}, while the second, \isi{doubling} \isi{complementizer} is positioned after the element that occurs right after the first \isi{complementizer}. The phenomenon was first observed in mainly dialectal variants of \ili{Romance} languages such as \ili{Italian} (\citealt{Paoli2003, ledgeway2005, munaro2016}), \ili{French} \citep{dagnac2012}, \ili{Spanish} (\citealt{villa2012, GiP2014}) and \ili{Portuguese} \citep{mascarenhas2007}.

Examples of \ili{Slovenian} \isi{complementizer doubling} are shown in \REF{ex:plesnicar:oneb} and \REF{ex:plesnicar:two}.\footnote{A subset of our data were collected in a controlled acceptability-judgment task; results of that task are reported in \cite{plesnicar2016}. The judgments in \cite{plesnicar2016} and the judgments reported here are by \ili{Slovenian} speakers from Goriška region. Regional variation with respect to this type of sentences is possible.}\textsuperscript{,}\footnote{Phrases in examples typeset in small capitals are focused. The use of commas in this paper corresponds to \ili{Slovenian} writing conventions and is not intended to reflect either the syntactic status of constituents that are located between the two complementizers or \isi{prosody}.}


\begin{exe}
\ex \begin{xlist}\label{ex:plesnicar:1}
\ex \label{ex:plesnicar:onea}\gll Mislim,	da	ker		pošteno	dela, 	mu	pripada plačilo.\\
	think that because	honest		work	he	belong payment \\
\trans `I think that because he works hard, he should get paid.'
\ex \label{ex:plesnicar:oneb}
\gll	Mislim,	da	ker		pošteno	dela,	\textbf{da}	mu 	pripada plačilo. \\
	think that	because	honest	work	that	he	belong payment \\
\trans `I think that because he works hard, he should get paid.'
\end{xlist}\end{exe}

\begin{exe}
\ex \label{ex:plesnicar:two}
\gll Rekel	je,	da	\textsc{Petrovim}	\textsc{prijateljem}	da	ne	zaupa. \\
	said	\textsc{aux}	that	Peter’s		friends	that	not	trust\\
\trans `He said that he doesn’t trust Peter’s friends.'
\end{exe}

\noindent\largerpage
As is evident from the examples in \REF{ex:plesnicar:oneb} and \REF{ex:plesnicar:two}, the \isi{complementizer} can only be doubled if there is some phrase that splits the two complementizers in the \isi{left periphery} of the \isi{embedded clause}.\footnote{\cite{lenertova2001} and \cite{Veselovská2008} argue that \ili{Czech} \isi{second position} clitics occupy the lowest head in the CP field, i.e. Fin\textsuperscript{0}. If \ili{Slovenian} clitics occupied the same position we could say that the second \isi{complementizer} in \ili{Slovenian} is also in Fin\textsuperscript{0}, especially in view of the contrast between \REF{ex:plesnicar:oneb} and \REF{ex:plesnicar:embone}, as the \isi{clitic} cannot follow the first \isi{complementizer}.

\ea [*] {\label{ex:plesnicar:embone}
\gll Mislim, 	da 	mu, 	ker 	pošteno 	dela, 	da 	pripada 	plačilo.\\
     think	 	that	he	because	honest	work	that	belong	payment\\
\glt }
\z

\noindent But as pointed out in \cite{marusic2008} and as is evident by comparing \REF{ex:plesnicar:onea} and \REF{ex:plesnicar:embtwo}, \ili{Slovenian} clitics do not occupy a unique \isi{syntactic position}, so they cannot be located in Fin\textsuperscript{0}.

\begin{exe}
\ex \label{ex:plesnicar:embtwo}
\gll Mislim, 	da 	mu, 	ker 	pošteno 	dela, 	pripada 	plačilo. \\
	think	that	he	because	honest	work	belong	payement\\
\trans `I think that because he works hard, he should get paid.'
\end{exe}

\noindent Due to the different status of \ili{Slovenian} and \ili{Czech} clitics, an argumentation built on the \isi{clitic} status will most likely not be on the right track.} If there is no such element, \isi{complementizer doubling} cannot occur. The intervening element in \isi{complementizer doubling} constructions must be a constituent which was not base-generated in its left-peripheral position between the two complementizers, but is rather located there as a result of movement (the arguments for this claim will be presented in \sectref{s2} below, where the properties of \isi{complementizer doubling} phenomena in \ili{Slovenian} are discussed in more detail). Possible landing sites for such moved constituents are specifiers of the projections that are positioned between the first and the second \isi{complementizer}, arguably the specifiers of Topic or Focus phrases. The moved constituent can be an \isi{adverbial} phrase or \isi{adverbial clause}, as shown in \REF{ex:plesnicar:oneb}, but it can also be a PP- or an NP-argument of the \isi{embedded clause}, as is the case in \REF{ex:plesnicar:two}.

\isi{Complementizer doubling} can thus be understood as an inherent property of complement clauses that can be realized only if some additional element or some additional structure is inserted into the specifier position of one of the available projections in the CP field, i.e., a projection between force and \isi{finiteness phrase}. The \isi{focus} of the remaining part of the paper will be on examples such as \REF{ex:plesnicar:oneb} and \REF{ex:plesnicar:two}, in which a second \isi{complementizer} is present in the sentence structure.

Before proceeding with a detailed description and analysis, two clarifications are in order. In \isi{complementizer  doubling} constructions, both complementizers – the first one, which introduces the \isi{complement clause}, and the second, \isi{doubling} one – occur in the \isi{complement clause}. At this point, there is no clear picture about what the full structure of the \ili{Slovenian} CP-domain looks like. In this paper, we are going to adopt one (of course not exhaustive) relevant proposal for the structure of the \ili{Slovenian} CP domain – that of \cite{mismas2015} – which is based on her investigation of \ili{Slovenian} multiple \isi{wh-fronting}. A graphical representation of the relevant structure is presented in \figref{three} below.\footnote{According to \cite{mismas2015}, the starred projections are in the CP field only when needed, and their positions are interchangeable.}

\begin{figure}
 \begin{forest}for tree={s sep=.6cm,l=0,inner sep=0}
  [ForceP
   [SpecForceP]
   [Force$'$
    [Force\textsuperscript{0}]
    [InterP
     [SpecInterP]
     [Inter$'$
      [Inter\textsuperscript{0}]
      [TopicP*
       [SpecTopicP]
       [Topic$'$
        [Topic\textsuperscript{0}]
        [FocusP*
         [SpecFocusP]
         [Focus$'$
          [Focus\textsuperscript{0}]
          [WhP*
           [SpecWhP]
           [Wh$'$
            [Wh\textsuperscript{0}]
            [FinP
             [SpecFinP]
             [Fin$'$
              [Fin\textsuperscript{0}]
              [IP]
             ]
            ]
           ]
          ]
         ]
        ]
       ]
      ]
     ]
    ]
   ]
  ]
\end{forest}
\caption{The structure of the CP field in Slovenian \citep{mismas2015}}
\label{three}
\end{figure}

\largerpage
Secondly, as becomes clear from  \REF{ex:plesnicar:onea} and  \REF{ex:plesnicar:oneb} above, \ili{Slovenian} \isi{complementizer doubling} is optional, and the only obligatory \isi{complementizer} in potential \isi{complementizer doubling} constructions is the highest one, as in  \REF{ex:plesnicar:onea} above. The optionality of the \isi{doubling} \isi{complementizer} is characteristic of \isi{complementizer doubling} phenomena in general (e.g. \citealt{ledgeway2005, dagnac2012, munaro2016}). At this point we will not try to answer the question about the motivation for the appearance of the \isi{doubling} \isi{complementizer} in the syntactic structure. We will, however, assume – in accordance with the cartographic analyses of the \isi{left periphery} (e.g. \citealt{rizzi2004}; though contrary to \citealt{mismas2015}) – that the position for the \isi{doubling} \isi{complementizer} is present in the sentence structure regardless of whether the \isi{doubling} \isi{complementizer} is phonetically realized or not.

In the next section, we will describe the key properties of \isi{complementizer doubling} constructions: first, the movement of the intervening element; second, the absence of interpretative differences between \isi{complementizer doubling} constructions and their counterparts without an overt \isi{doubling} \isi{complementizer}; and third, the restriction of \isi{complementizer doubling} to subordinate clauses, and more specifically, to argument clauses/clauses in the \isi{syntactic position} of argument.

\section{Properties of Slovenian complementizer doubling constructions}\label{sec:plesnicar:s2}
This section describes and exemplifies three key properties of \isi{complementizer doubling} constructions in \ili{Slovenian} linked to their structure and interpretation. We will first show that phrases that occur between the two complementizers must have moved to that position from a lower structural position. Then we will argue that there is no interpretative difference between the \isi{complementizer doubling} construction and its counterpart without a \isi{doubling} \isi{complementizer}. And thirdly, we will claim that \ili{Slovenian} \isi{complementizer doubling} is not just an example of speech disfluency, but rather a \isi{syntactic phenomenon} available only in argument clauses.

\subsection{Movement of the intervening element} \label{s21}

As shown in \REF{ex:plesnicar:four}, the phrase \textit{svojo mamo} `one’s own mother', which sits in the left edge between the two complementizers and contains the bound \isi{reflexive} \isi{adjective} \textit{svojo} `one’s own', is bound by \textit{vsak} `everyone', whose surface position would appear to suggests that it is located lower in the hierarchical structure of the \isi{embedded clause}. By \isi{Principle A} of the traditional binding theory, reflexives must have a local antecedent, which essentially means that the phrase that contains the bound \isi{reflexive} in \REF{ex:plesnicar:four} must have been originally located in the \isi{embedded clause} \citep{chomsky1981}.

\begin{exe}
\ex \label{ex:plesnicar:four}
\gll Rekel	je,	da	svojo$_i$	mamo		da	ima	vsak$_i$		rad. \\
	said	\textsc{aux}	that	one’s	mother		that	has	everyone	like\\
\trans `He said that everyone likes their own mother.'
\end{exe}

\noindent More direct evidence for the claim that the surface position of the phrase \textit{svojo mamo} `one’s own mother' in \REF{ex:plesnicar:four} is a derived position is shown in \REF{ex:plesnicar:five}, where movement of the \isi{reflexive} is illustrated step by step. \REF{ex:plesnicar:fivea} is the example with the most salient or preferable word order of the structurally incorporated numeration elements from sentence \REF{ex:plesnicar:four}, in which nothing has moved into the CP field, and \REF{ex:plesnicar:fiveb} is the example with movement of the phrase that contains the bound \isi{reflexive} from the original structural position into the next available structural position, though not yet as high as the CP field. Another available position for the phrase with the bound \isi{reflexive} is shown in  \REF{ex:plesnicar:fivec}: since only a copy is left after the movement of this phrase through the CP field of the \isi{embedded clause}, \isi{complementizer doubling} is not available in \REF{ex:plesnicar:fivec}.

\begin{exe}
\ex \begin{xlist} \label{ex:plesnicar:five}
\ex \label{ex:plesnicar:fivea}\gll Rekel	je,	da	ima	vsak$_i$		rad	svojo	mamo$_i$.\\
	said	\textsc{aux}	that	has	everyone	like	one’s mother \\
\trans `He said that everyone likes their own mother.'
\ex \label{ex:plesnicar:fiveb}
\gll	Rekel	je,	da	ima	svojo	mamo$_i$	vsak$_i$		rad. 	\\
	said	\textsc{aux} that	has	one’s	mother	everyone	like \\
\trans `He said that everyone likes their own mother.'
\ex \label{ex:plesnicar:fivec}
\gll	Svojo	mamo$_i$	je	rekel,	da	ima	vsak$_i$		rad. 	\\
	one’s	mother	\textsc{aux} 	said	that	has	everyone	like \\
\trans `He said that everyone likes their own mother.'
\end{xlist}\end{exe}

\noindent Of course, if the movement explanation from above is accepted for cases like \REF{ex:plesnicar:four}, then it is reasonable to try and pursue the approach more generally, among others also for cases like \REF{ex:plesnicar:six}, in which the intervening element is not an argument phrase but an adjunct \isi{adverbial clause}. In other words, we would expect that all intervening constituents found between the two complementizers in \isi{complementizer doubling} constructions, including \isi{adverbial} clauses, have moved to the intervening position from their original position, which is lower in the sentence structure. \REF{ex:plesnicar:six} below confirms the correctness of this approach for an intervening \isi{adverbial clause} adjunct. %manjkajo črke pri referencah

\begin{exe}
\ex \begin{xlist} \label{ex:plesnicar:six}
\ex \label{ex:plesnicar:sixa}
\gll Rekel 	je, 	da 	vsak$_i$ 		žaluje, 	če 	izgubi 	svojo 	mamo$_i$.\\
	said 	\textsc{aux}	that	everyone	grieves	if	loses	one’s	mother \\
\trans `He said that everyone grieves if they lose their mother.'
\ex \label{ex:plesnicar:sixb}
\gll	Rekel 	je, 	da 	če 	izgubi 	svojo 	mamo$_i$, da 	vsak$_i$ 		žaluje. 	\\
	said	\textsc{aux}	that	if	loses	one’s	mother	 that	everyone	grieves \\
\trans `He said that everyone grieves if they lose their mother.'
\end{xlist}\end{exe}

\noindent The examples in \REF{ex:plesnicar:six} are direct parallels to the examples in \REF{ex:plesnicar:five}. A comparison of \REF{ex:plesnicar:sixa} and \REF{ex:plesnicar:sixb} shows that the \isi{reflexive} \isi{adjective} \textit{svojo} `one’s own', located within an adjunct \isi{adverbial clause}, must have originated, like the one in \REF{ex:plesnicar:five}, lower in the sentence structure or else it could not have satisfied the conditions set by the binding theory’s Principle A.


\subsection{Interpretation of the complementizer doubling construction and its counterpart without an overt doubling complementizer} \label{s22}

Crucially, there is no interpretative difference between cases with the \isi{doubling} \isi{complementizer} and their minimal-pair counterparts without the second \isi{complementizer}, such as \REF{ex:plesnicar:onea} and \REF{ex:plesnicar:oneb} above. In the \isi{complementizer doubling} construction the second, \isi{doubling} \textit{da} `that' does not seem to function as anything other than a \isi{doubling} \isi{complementizer}, clearly not having the echo- or discourse-marking role that in certain other cases, such as \REF{ex:plesnicar:seven}--\REF{ex:plesnicar:eight}, \textit{da} `that' does also have.

\begin{exe}
 \ex \label{ex:plesnicar:seven}
 \begin{xlist}
 \exi{A:}[]{\gll Petra	pride.\\
	Petra	come\\
    \trans `Petra is coming.'}
  \exi{B:}[]{\gll \textsc{Kdo}	da 	pride?\\
  who	that		come\\
 \trans `Who’s coming, again?'}
 \end{xlist}
\end{exe}

\begin{exe}
\ex \label{ex:plesnicar:eight}
\gll Misliš	da? \\
	think	that\\
\trans `Do you think so?'
\end{exe}

\noindent As can be seen from the contrasts between examples \REF{ex:plesnicar:onea}, \REF{ex:plesnicar:oneb}, \REF{ex:plesnicar:seven} and \REF{ex:plesnicar:eight} the \ili{Slovenian} element \textit{da} `that' can be used to serve at least four different functions. It can function as a \isi{complement clause} \isi{introducer}, as in \REF{ex:plesnicar:onea}, as a \isi{doubling} \isi{complementizer} in \isi{complementizer doubling} constructions, as in \REF{ex:plesnicar:oneb}, as an echo marker, as shown in \REF{ex:plesnicar:seven}, or as a discourse marker, as \REF{ex:plesnicar:eight} shows. The last two functions will be left aside in the remaining part of the paper and only \textit{da} `that' elements in the function of primary \isi{complementizer} and \textit{da} `that' elements in function of \isi{doubling} \isi{complementizer} will be the focus of our analysis, since these are the only available candidates that can fill the two edge projections of the \isi{complementizer system}.

\subsection{Restriction of complementizer doubling to argument clauses} \label{s23}
\largerpage[1.5]
In \ili{Slovenian}, \isi{complementizer doubling} is possible only in sentences with true complement clauses (clauses in the \isi{syntactic position} of argument) introduced by the \isi{complementizer} \textit{da} `that', as in \REF{ex:plesnicar:nine}, marginally also in clauses introduced by the \isi{complementizer} \textit{če} `if', as illustrated in \REF{ex:plesnicar:ten}, but not in \isi{adverbial} or \isi{adjectival} (\isi{relative}) subordinate clauses, as shown, respectively, in \REF{ex:plesnicar:eleven}--\REF{ex:plesnicar:thirteen}.\footnote{As will be shown in \sectref{sec:plesnicar:s3}, the \isi{complementizer} \textit{če} `if' is possible, but marked, as an \isi{introducer} of a subject clause in copula sentences or in certain cases as an \isi{introducer} of a \isi{complement clause}. In this type of contexts \textit{če} can be used as an alternative to the \isi{complementizer} \textit{da}. Although the use of this declarative \textit{če}, when compared to \textit{da}, does seem to bring a certain semantic difference, this difference is not directly relevant for the purposes of this paper.}\textsuperscript{,}\footnote{In \REF{ex:plesnicar:eleven}--\REF{ex:plesnicar:thirteen}, the elements \textit{ki} `who', \textit{ker} `because' and \textit{ko} `when' are complementizers that introduce a \isi{subordinate clause} and are therefore, in this function, more like the complement-clause introducing \textit{da} `that' from \REF{ex:plesnicar:nine} or the subject-clause introducing \textit{če} `if' from \REF{ex:plesnicar:ten} than like a wh-question word (as one might incorrectly conclude especially on the basis of the glosses in examples \REF{ex:plesnicar:eleven} and \REF{ex:plesnicar:thirteen}).}

\begin{exe}
\ex \label{ex:plesnicar:nine}
\gll  Sem	pa	za	to,	da	ker		imamo	zadevo	na dnevnem	redu,	da	jo	čimprej	tudi	zaključimo {\ldots}\\
 am	\textsc{ptcl}	for	this	that	because	have	matter	on daily		agenda	that	it	as.soon.as.possible		also	finish	\\
\trans `I think that since the matter is already on the agenda, the only reasonable thing is to conclude it as soon as possible {\ldots}'\hfill (Gigafida corpus)
\end{exe}

\begin{exe}
\ex \label{ex:plesnicar:ten}
\gll Najslabše 	je,	če	ker		te	ne	razume,	(\textsuperscript{??}\hspace{-2pt} če) kričiš. \\
	worst		is	if	because	you	not	understand	{} if scream\\
\trans `The worst is if you scream because he does not understand you.'
\end{exe}

\begin{exe}
\ex \label{ex:plesnicar:eleven}
\gll  Vse	preveč	je	tistih,	ki 	ko	naredijo	izpit,	(*\hspace{-2pt} ki) mislijo,	da	znajo	vozit.\\
	all	too.many of them,	who	when	pass		exam	{} who think		that	know	drive\\
\trans `There are too many of those people who, when they receive the driver's license, think that already know how to drive a car.' %gloss poglej, zbrisamo intended tudi v spodnjih dveh
\end{exe}
\begin{exe}

\ex \label{ex:plesnicar:twelve}
\gll Ne	sme	se	premaknit,	ker		če	se	premakne, (*\hspace{-2pt} ker)		mu	lahko	počijo	kosti.\\
	not	may	\textsc{aux}	move	because	if	refl.	move {} because	him	may	crack	bones\\
\trans `He should not move, because if he moves his bones may crack.'
\end{exe}


\begin{exe}
\ex \label{ex:plesnicar:thirteen}
\gll Ni	mu	všeč,	ko	ker		je	jezna,	(*\hspace{-2pt} ko)	mu	ne	skuha kosila. \\
	not	him	like	when	because	\textsc{aux}	angry	{} when	him	not	cook lunch \\
\trans `He is not happy when she does not make him lunch because she is angry.'
\end{exe}

\noindent What examples \REF{ex:plesnicar:nine} to \REF{ex:plesnicar:thirteen} above show is that the discussed \ili{Slovenian} \isi{complementizer doubling} is a \isi{syntactic phenomenon}, not just an example of speech disfluency.\footnote{Note that when asked for a judgment, speakers emphasised that these examples seem a bit odd, but certainly possible. The speakers judged the examples as clumsy – clumsy with respect to the standard (normative) \ili{Slovenian}. However, \isi{complementizer doubling} can easily be found in \ili{Slovenian} corpora, e.g. Gigafida (the largest corpus of contemporary written \ili{Slovenian}).} If it was not a \isi{syntactic phenomenon} but just a disfluency-type of repetition, one would expect it to be available in any type of \isi{subordinate clause} introduced by a subordinate \isi{complementizer}, including \isi{adverbial} and \isi{adjectival} subordinate clauses, rather than being restricted by syntactic context. Clearly, the way the \isi{subordinate clause} is incorporated into the sentence structure is one of the key characteristics of the observed \isi{complementizer} \isi{doubling} phenomenon. What we conclude from the data introduced so far, then, is that the \ili{Slovenian} \isi{complementizer doubling} under discussion can occur in clauses in the \isi{syntactic position} of argument, but not in embedded adjunct clauses.

Summing up \sectref{s21} through \sectref{s23}, we showed that phrases positioned between the two complementizers of the \isi{embedded clause} must have moved to that position from a lower position of a sentence structure. We demonstrated that the presence of a \isi{doubling} \isi{complementizer} in \isi{complementizer doubling} constructions does not result in a difference in the meaning of the sentence, i.e., sentences with a second, \isi{doubling} \isi{complementizer} have the same meaning as their counterparts without a \isi{doubling} \isi{complementizer}. We also showed that \isi{complementizer} \isi{doubling} is only available in argument clauses, from which we concluded that \isi{complementizer} \isi{doubling} is a \isi{syntactic phenomenon}.

\section{Doubly-filled \textsc{comp} filter vs. complementizer doubling} \label{sec:plesnicar:s3}

In this section we will briefly introduce another \ili{Slovenian} construction recently discussed by \cite{Bacskai2016}, which was argued there to represent a violation of and thus a counterexample to the \isi{doubly-filled \textsc{comp} filter}. At first sight, this construction seems very close to our \isi{complementizer doubling} constructions, so the question arises whether Bacskai-Atkari’s analysis can be used to capture our \isi{complementizer doubling} constructions as well. We will establish that these constructions are not one and the same phenomenon (even though in principle, there could be an indirect relation between these two phenomena). We will argue that there is a structural and interpretative difference between \ili{Slovenian} sentences that allow \isi{complementizer doubling} and \citeposst{Bacskai2016}  sentences. While the second \isi{complementizer} in the doubly-filled \textsc{comp} constructions always contributes some additional meaning to the sentence interpretation, no such interpretative difference is contributed by the second \isi{complementizer} in the \isi{complementizer doubling} constructions.

An example of  \citeposst{Bacskai2016} \isi{doubly-filled \textsc{comp} filter} construction (her data is from \citealt{hladnik2010restrictive}) is in \REF{ex:plesnicar:fourteen}.\footnote{In \citet{Bacskai2016}, the sentence is marked with a question mark. In \citet[(15)]{hladnik2010restrictive}, it has no such marking.}

\begin{exe}
\ex \label{ex:plesnicar:fourteen}
\gll Vprašal 	je, 	če 		da 	pride. \\
	asked 		\textsc{aux}	whether	that 	comes \\
\trans `He asked whether it was true that he was coming.'
\end{exe}

\noindent According to Bacskai-Atkari, such sentences represent structures with a simultaneously realized specifier and head position of the same CP. Crucially, the presence of the second \isi{complementizer} (\textit{da}) in \REF{ex:plesnicar:fourteen} above is responsible for the `it-was-true-that' part of the interpretation: without \textit{da}, the interpretation of the sentence would be just `He asked whether he was coming'. This means that \REF{ex:plesnicar:fourteen} is felicitous only when used as a response to a statement such as `He is coming' \citep{Bacskai2016}. In our \isi{complementizer-doubling} \ili{Slovenian} examples from above, on the other hand, the second \isi{complementizer} does not seem to contribute any special interpretational difference, and there seems to be no special meaning-related requirement for the use of \isi{complementizer doubling} construction in \ili{Slovenian}. Recall from the minimal-pair sentences in \REF{ex:plesnicar:1} and \sectref{s22} that the structure with and without the second \isi{complementizer} both have the same interpretation, which would suggest that the realization of the \isi{doubling} \isi{complementizer} in \ili{Slovenian} \isi{complementizer doubling} constructions is optional.

The lack of semantic effects thus clearly separates our \isi{complementizer doubling} constructions from \citeposst{Bacskai2016} \isi{doubly-filled \textsc{comp} filter} construction. In addition, our \isi{complementizer} \isi{doubling} sentences contain not just two complementizers but also a phrase that has moved into the CP layer. For these constructions Bacskai-Atkari’s approach therefore cannot work, because there must be at least three positions available in the \isi{complementizer system} structure of the embedded argument clause to accommodate this CP material. This is sketched in \figref{ex-fifteen}, where, as will become clear below, the topmost \isi{functional projection} (responsible for typing the clause) hosts the first \isi{complementizer}, the specifier of the intermediate \isi{functional projection} (TopicP or FocusP) hosts the moved constituent, and the lowest projection hosts the \isi{doubling} \isi{complementizer}.



\begin{figure}
 \centering
    \begin{forest}
    for tree={inner sep=0, l=0}
  [CP, s sep=15mm
    [XP]
    [C$'$, s sep=2mm
      [C\textsuperscript{0} [that/if]]
      [CP, s sep=2mm
        [ZP [moved\\ constituent]]
        [C$'$, s sep=15mm
         [C\textsuperscript{0}]
         [CP, s sep=15mm
          [YP]
          [C$'$, s sep=15mm
           [C\textsuperscript{0} [that/if]]
           [...]
          ]
         ]
        ]
      ]
    ]
  ]
\end{forest}

\caption{The structure of complementizer doubling construction}\label{ex-fifteen}
\end{figure}

It was shown above that the \isi{complementizer doubling} and the doubly-filled \textsc{comp} constructions are not the same phenomenon, contrary to what one might assume on the basis of the superficial similarity between the complementizers involved in these two construction types. Now that we have established the difference between these two phenomena, we can \isi{focus} on \isi{complementizer doubling} more closely. This closer examination of \isi{complementizer doubling} in the next section will then form the basis for the analysis provided in \sectref{sec:plesnicar:s5}, which will focus on determining the syntactic positions that the complementizers occupy in the \isi{doubling construction}.


\section{Further characteristics of Slovenian complementizer doubling}\label{sec:plesnicar:s4}

As we already saw in \sectref{s23} above, \textit{če} `if' is another \isi{complementizer} that can, in addition to \textit{da} `that', function as a \isi{doubling} \isi{complementizer} (albeit with some degree of degradation). These two complementizers, however, are marked by a clear difference in terms of their \isi{doubling} positions. The declarative \isi{complementizer} \textit{če} `if' can be doubled in the subject clause of copula sentences with predicates such as \textit{pametno je} `smart is' or \textit{najslabše je} `the worst is', as shown in examples \REF{ex:plesnicar:sixteen} and \REF{ex:plesnicar:seventeen} below, or in complement clauses with verbs such as \textit{prositi} `ask/request', as in \REF{ex:plesnicar:eighteen}.

\begin{exe}
\ex [\textsuperscript{??}] {\label{ex:plesnicar:sixteen}
 \gll Pametno	bi	bilo,	če	da	se	izogneš	dezinterpretaciji, če	mu	svoje	stališče		jasno	predstaviš.\\
 smart		would	be	if.\textsc{decl}	that	you	avoid		misinterpretation if	him	your	opinion	clearly	present \\
\trans `It would be good if you state your position clearly, so that you avoid misinterpretation.'}
\end{exe}

\begin{exe}
\ex [\textsuperscript{??}] {\label{ex:plesnicar:seventeen}
 \gll Najslabše 	je,	če	ker		te	ne	razume,	če kričiš. \\
 worst		\textsc{aux}	if.\textsc{decl}	because	you	not	understand	if scream \\
\trans `The worst is if you scream because he does not understand you.'}
\end{exe}

\begin{exe}
\ex [\textsuperscript{??}] {\label{ex:plesnicar:eighteen}
\gll Prosil 	je, 	če 	ko 	konča		z 	delom, 	če pospravi 	za 	sabo.\\
 request \textsc{aux} if.\textsc{decl}	when	finish		with	work		if clean.up	after	yourself \\
\trans `He asked if he could clean up after himself when he finishes with his work.'}
\end{exe}

\noindent On the other hand, the \isi{doubling construction} is possible with \textit{da} used in complement clauses introduced by the verbs such as \textit{misliti} `think' or \textit{reči} `say', as shown in \REF{ex:plesnicar:oneb} and \REF{ex:plesnicar:two} above, as well as with \textit{da} used in any other environment where \textit{če} is possible; compare examples \REF{ex:plesnicar:seventeen} and \REF{ex:plesnicar:eighteen} above with examples \REF{ex:plesnicar:nineteen} and \REF{ex:plesnicar:twenty} below.

\begin{exe}
\ex \label{ex:plesnicar:nineteen}
\gll Najslabše	je, 	da 	ker 		te	ne 	razume, 	da 	kričiš.\\
 worst		\textsc{aux}	that	because	you	not	understand	that     scream \\
\trans `The worst is if you scream because he does not understand you.'
\end{exe}

\begin{exe}
\ex \label{ex:plesnicar:twenty}
\gll Prosil 		je, 	da 	ko 	konča	 z 	delom, da 	pospravi za 	sabo.\\
request		\textsc{aux}	that	when	finish	with	work	that 	clean.up after	yourself \\
\trans `He asked if he could clean up after himself when he finishes with his work.'
\end{exe}

\noindent More accurately, it is not just \textit{possible} for \textit{da} to be used in positions available for \textit{če}: according to our informants, the use of \textit{da} rather than \textit{če} actually improves the acceptability of such \isi{doubling} constructions, compare \REF{ex:plesnicar:sixteen}, \REF{ex:plesnicar:seventeen} and \REF{ex:plesnicar:eighteen} above with \REF{ex:plesnicar:twentyone}, \REF{ex:plesnicar:twentytwo} and \REF{ex:plesnicar:twentythree} below. Moreover, the use of \textit{da} seems to be more natural in the function of complement subordinator than the use of \textit{če}; compare the contrast between examples \REF{ex:plesnicar:seventeen} or \REF{ex:plesnicar:twentytwo} and \REF{ex:plesnicar:nineteen}.

\begin{exe}
\ex \label{ex:plesnicar:twentyone}
\gll Pametno 	bi 	bilo, 	če 	da 	se 	izogneš 	dezinterpretaciji, da	mu	svoje	stališče		jasno	predstaviš.\\
smart	would	be if.\textsc{decl} that you	avoid 	misinterpretation that	him	your	opinion	clearly	present \\
\trans `It would be good if you state your position clearly, so that you avoid misinterpretation.'
\end{exe}

\begin{exe}
\ex [\textsuperscript{?}] {\label{ex:plesnicar:twentytwo}
 \gll	Najslabše 	je, 	če 	ker 		te 	ne	razume, 	da      kričiš. \\
 worst	\textsc{aux}	if.\textsc{decl}	because	you	not	understand	that     scream \\
\trans `The worst is if you scream because he does not understand you.'}
\end{exe}

\begin{exe}
\ex [\textsuperscript{?}] {\label{ex:plesnicar:twentythree}
\gll Prosil 		je, 	če 	ko 	konča 	z	delom,	da 	pospravi za sabo.\\
request		\textsc{aux}	if.\textsc{decl}	when	finish	with	work	that 	clean.up after	yourself\\
\trans `He asked if he could clean up after himself when he finishes with his work.'}
\end{exe}

\noindent The acceptability of a particular \isi{complementizer} in the \isi{doubling construction} thus appears to depend on the \isi{matrix predicate}, which can also be confirmed with the availability of all four different combinations in cases where the matrix \isi{verb} is such that it accepts either \textit{da} or \textit{če}, as attested through the set of examples in \REF{ex:plesnicar:seventeen}, \REF{ex:plesnicar:nineteen} and \REF{ex:plesnicar:twentytwo} above and \REF{ex:plesnicar:twentyfour} below.

\begin{exe}
\ex [\textsuperscript{??}] {\label{ex:plesnicar:twentyfour}
\gll Najslabše 	je, 	da 	ker 		te 	ne	 razume, 	če    kričiš.\\
 worst		\textsc{aux}	that	because	you	not	understand	if     scream \\
\trans `The worst is if you scream because he does not understand you.'}
\end{exe}

\noindent Given that the acceptability of \textit{da} or \textit{če} in the \isi{complementizer doubling} construction depends on the requirements of the \isi{matrix predicate}, it is not surprising that the same type of matrix-predicate dependence actually holds for the use of \textit{da}/\textit{če} outside the \isi{doubling construction}. On the one hand, if the \isi{matrix predicate} allows the use of either one of these complementizers outside the \isi{doubling construction}, as in \REF{ex:plesnicar:twentyfive} and \REF{ex:plesnicar:twentysix}, then both are also acceptable in the \isi{doubling construction}.

\begin{exe}
\ex \label{ex:plesnicar:twentyfive}
\gll Najslabše 	je, 	če / \hspace{-2pt} da 		kričiš.\\
  worst		\textsc{aux}	if.\textsc{decl} {} {} that	scream\\
\trans `The worst is if you scream.'
\end{exe}


\begin{exe}
\ex \label{ex:plesnicar:twentysix}
\gll Prosil 		je, 	če / \hspace{-2pt} da 		pospravi 	za 	sabo.\\
 request		\textsc{aux}	if.\textsc{decl} {} {} that	clean.up 	after	yourself \\
\trans `He asked if he could clean up after himself.'
\end{exe}

\noindent On the other hand, in sentences with \textit{misliti} `think' or \textit{reči} `say' as the matrix \isi{verb}, where the \isi{introducer} of the \isi{complement clause} can only be \textit{da}, the use of \textit{če} is impossible, as shown by \REF{ex:plesnicar:twentyseven}--\REF{ex:plesnicar:twentyeight}.

\begin{exe}
\ex  [*]{\label{ex:plesnicar:twentyseven}
\gll Mislim,	če 	mu 	pripada 	plačilo.\\
  think		if.\textsc{decl}	he	belong		payement\\
\trans Intended: `I think that he should get paid.'}
\end{exe}


\begin{exe}
\ex [\#] {\label{ex:plesnicar:twentyeight}
\gll Rekel 	je, 	če 	ne 	zaupa 	Petrovim 	prijateljem. \\
 said	\textsc{aux}	if.\textsc{decl} not	trust	Peter’s 	friends \\
\trans Intended: `He said that he doesn’t trust Peter’s friends.'}
\end{exe}

\noindent In addition to its declarative use from above, \textit{če} `if' can also function as an \isi{introducer} of an embedded yes/no question. Like its declarative use, \textit{če}’s interrogative use also allows \isi{doubling}, as shown in \REF{ex:plesnicar:twentynine} below.

\begin{exe}
\ex \label{ex:plesnicar:twentynine}
\gll Sprašuje se,	če	ker	ga	nihče		ne	posluša,	če	naj		še	kar	naprej	govori. \\
   ask  self 	if.\textsc{int} because him nobody	not	listen if should more	still	on	talk\\
\trans `He wonders if he should keep talking, given that no one is listening to him.'
\end{exe}

\noindent Note that examples very similar to our \ili{Slovenian} \isi{doubling} examples from above have also been observed in non-standard varieties of \ili{English}. Specifically, \cite{mccloskey2006}  reports \isi{complementizer doubling} in declarative and interrogative contexts of the type shown in \REF{ex:plesnicar:thirty} and \REF{ex:plesnicar:thirtyone}, which he analyzes with two CPs, one stacked on top of the other.

\ea \label{ex:plesnicar:thirty}
He thinks that if you are in a bilingual classroom that you will not be encouraged to learn \ili{English}.
\hfill\citep[23,~(69b)]{mccloskey2006}
\z

\ea \label{ex:plesnicar:thirtyone}
John was asking me if, when the house was sold, would they move back to Derry.
\hfill\citep[24,~(72c)]{mccloskey2006}
\z

\noindent In perfect parallel to \ili{Slovenian}, \isi{doubling} of the \ili{English} declarative \isi{complementizer} \textit{that} is realized in complement clauses introduced by verbs like \textit{think}, as shown in \REF{ex:plesnicar:thirty} above. On the other hand, the parallel between \ili{Slovenian} interrogative \textit{če} \isi{doubling} and the \ili{English} \REF{ex:plesnicar:thirtyone} is less straightforward; \REF{ex:plesnicar:thirtyone} is a less transparent case of \isi{complementizer doubling}. \cite{mccloskey2006} argues, however,  that \REF{ex:plesnicar:thirtyone} nevertheless shows clear evidence for the presence of two CPs through the simoultaneous presence of both \textit{if} and the auxiliary-subject inversion. More specifically, all the CP material in \REF{ex:plesnicar:thirtyone} is evidence for the presence of three distinct CP-field positions; one for the yes/no clause \isi{introducer} \textit{if}, one for the topicalized constituent \textit{when the house was sold} and another one for the inverted auxiliary \textit{would}. Even though \REF{ex:plesnicar:thirty} and \REF{ex:plesnicar:thirtyone} are thus both analyzed as cases of \isi{complementizer doubling}, they are also marked by a difference, namely, only \REF{ex:plesnicar:thirty} shows actual \isi{doubling} of the lexical \isi{complementizer}.

Similarly, \ili{Slovenian} also shows a difference between the cases of declarative \isi{complementizer doubling} and interrogative \isi{complementizer doubling}. Doubling of \textit{če} in its declarative use is somewhat degraded, and the acceptability of such \isi{complementizer doubling} examples improves if \textit{da} is used instead, as was shown above. In contexts of embedded yes/no questions, however, only \textit{če} can occupy the position of the \isi{doubling} \isi{complementizer}, as can be seen from the comparison between \REF{ex:plesnicar:twentynine} above and \REF{ex:plesnicar:thirtytwo} below.


\begin{exe}
\ex[*]{\label{ex:plesnicar:thirtytwo}
\gll Sprašuje	se,	če	ker		ga	nihče		ne	posluša, da	naj	še	kar	naprej	govori.\\
  ask		self	if.\textsc{int}	because	him	nobody	not	listen that	should	more	still	on	talk\\
\trans Intended: `He wonders if he should keep talking, given that no one is listening to him.'}
\end{exe}

\noindent A comparison between \isi{doubling} in the two contexts, i.\,e., complement clauses and embedded yes/no questions, implies that although \isi{complementizer doubling} is possible in both types of constructions, there is an additional restriction in the case of the latter. This points to the fact that the \isi{complement clause} \isi{introducer} \textit{da} `that' and the embedded yes/no clause \isi{introducer} \textit{če} `if' are not the same element with respect to their function in the sentence structure. This is further illustrated through the contrast between \REF{ex:plesnicar:thirtythree} below and \REF{ex:plesnicar:twentynine} above. \REF{ex:plesnicar:thirtythree} shows that the substitution of \textit{če} from \REF{ex:plesnicar:twentynine} above with \textit{da} is not possible in the case of a yes/no embedded question, which is required by the matrix \isi{verb} \textit{spraševati} `ask'.

\begin{exe}
\ex  [*] {\label{ex:plesnicar:thirtythree}
\gll Sprašuje	se,	da	ker	ga	nihče	ne	posluša,	da naj	še	kar	naprej	govori.\\
  ask self that	because him	nobody not	listen that should	more still	on	talk\\
\trans Intended: `He wonders if he should keep talking, given that no one is listening to him.'}
\end{exe}

\noindent We take this to suggest that, on the one hand, \textit{da} and the declarative \textit{če} occupy the same structural position -- the head of the highest CP projection, \isi{force projection} -- since in certain cases they can be used interchangeably, as shown in \REF{ex:plesnicar:seventeen} and \REF{ex:plesnicar:nineteen}. On the other hand, despite the fact that it also allows \isi{doubling}, the embedded yes/no question \isi{introducer} must be assigned a different position, which makes it impossible for \textit{da} to freely take this position. The fact that it is only in indirect questions that the \isi{doubling} \isi{complementizer} must be identical in form to the doubled one does not suggest that we are dealing with two different types of \isi{complementizer doubling} constructions; rather, it only confirms the idea that \ili{Slovenian} \isi{complementizer  doubling} is restricted by requirements of the matrix \isi{verb}, i.e., both in declarative and in interrogative complement clauses. In fact, \ili{Slovenian} also seems to allow the \isi{doubling} of the yes/no-question operator in embedded questions, as shown in \REF{ex:plesnicar:thirtyfour}, further suggesting that we are dealing with a single system of C-\isi{doubling}.

\begin{exe}
\ex \label{ex:plesnicar:thirtyfour}
\gll Vprašal je, a 	ker ga 	ne 	mara, 	a 	naj 	kar 	gre?\\
   ask \textsc{aux} Q because him	not	like Q	should	just	leave\\
\trans `He asked if he should leave because s(he) don’t like him..'
\end{exe}

\noindent To recapitulate, \sectref{sec:plesnicar:s3} showed that \citeposst{Bacskai2016} doubly-filled \textsc{comp} phenomenon and our \isi{complementizer} \isi{doubling} realize two different types of CP configurations. For the realization of the doubly-filled \textsc{comp} construction, it suffices to have one CP projection, whereas at least three positions are needed to derive our \isi{complementizer} \isi{doubling}. The second \isi{complementizer} of the doubly-filled \textsc{comp} construction contributes additional meaning to the sentence, while our \isi{doubling} \isi{complementizer} does not affect the interpretation of the sentence. Furthermore, as was shown in \sectref{sec:plesnicar:s4}, our \isi{complementizer doubling} is not exhibited only in declarative complement clauses but also in embedded yes/no questions, and in both of these, the use of the \isi{complementizer} is governed by the matrix \isi{verb}.

In the next section we will lay out our analysis of the \ili{Slovenian} \isi{complementizer doubling} construction, framing this also in the context of a comparison with \isi{complementizer doubling}, and its analysis, in \ili{Romance}.


\section{Analysis}\label{sec:plesnicar:s5}
We will claim that the \ili{Slovenian} \isi{complementizer} \isi{doubling construction} can best be explained with \citeauthor{rizzi1997}'s (\citeyear{rizzi1997, rizzi2001}) split CP model, and more specifically with the model of \ili{Slovenian} \isi{complementizer} field proposed by \cite{mismas2015}, which is sketched in \figref{three} above. We will argue that in declarative complement clauses, the first \isi{complementizer} sits in the highest part of the CP field, namely in the head of the \isi{force projection}, and the second \isi{complementizer} in the head of \isi{finiteness} projection. (For the doubled embedded yes/no clause \isi{introducer} \textit{če} `if', we will simply assume that, like in \ili{Italian} (see \citealt{rizzi2001}), it occupies the interrogative projection (InterP) in \figref{three} above.)

As has been observed for the \ili{Oïl} dialect of \ili{French} \citep{dagnac2012}, \ili{Slovenian} \isi{complementizer doubling} is closely related to \isi{finiteness} of the \isi{subordinate clause}, with non-finite subordinate clauses not allowing it. The existence of this restriction, however, is not surprising since \ili{Slovenian} non-finite subordinate clauses are never introduced by a subordinator, so that the use of \textit{da} is ungrammatical regardless of whether the latter is doubled or not, as shown in \REF{ex:plesnicar:thirtyfive}.

\begin{exe}
\ex  [*] {\label{ex:plesnicar:thirtyfive}
\gll Peter je hotel, da	ko	ga	ne	bo	nihče videl, \minsp{(} da) prositi Metko	za	pomoč.\\
  Peter	\textsc{aux} want that when	he	not	\textsc{aux} nobody	see	{} that ask	Metka	for	help\\
\trans Intended: `Peter wanted to ask Metka for help when no one sees him.'}
\end{exe}

\noindent In other words, whenever \textit{da} can appear in a structure as the subordinator, \isi{complementizer doubling} can also occur; but when the use of \textit{da} is ungrammatical, \isi{doubling} cannot occur. But even though \ili{Slovenian} finite embedded clauses are always introduced by a \isi{complementizer} and non-finite ones are always without a \isi{complementizer}, we still expect the information about \isi{finiteness}/non-\isi{finiteness} to be present in both types of sentences; as argued by \cite{rizzi1997}, the information carried by FinP expresses a distinction related to \isi{tense}.

Positing a link between complementizers and the \isi{finiteness} projection is no novelty either. That complementizers can express distinctions related to \isi{tense} has been established on the basis of \ili{Irish}, where sensitivity of the \isi{complementizer} to the \isi{tense} of the \isi{embedded clause} is reflected in the form of the \isi{complementizer} \citep{Koppen2017}. As shown by \REF{ex:plesnicar:36}, the past-\isi{tense} form of the \isi{complementizer} differs from the form used for all other tenses. If the \isi{embedded clause} shows future \isi{tense}, as in \REF{ex:plesnicar:thirtysixa}, the \isi{complementizer} is go, but when it shows past \isi{tense}, the \isi{complementizer} gets a past-\isi{tense} marker \textit{-r}, surfacing as \textit{gur}, as in \REF{ex:plesnicar:thirtysixb}.


\begin{exe}
\ex \label{ex:plesnicar:36}
\begin{xlist}
\ex \label{ex:plesnicar:thirtysixa}
\gll Deir 	sé 	go 	dtógfaidh 	sé 	an 	peann.\\
	say.\textsc{pres} 	he 	that 	take.\textsc{fut} he 	the 	pen\\
\trans  `He says that he will take the pen.'
\ex \label{ex:plesnicar:thirtysixb}
\gll Deir 	sé 	gur 		thóg 		sé 	an 	peann.\\
say.\textsc{pres} 	he 	that.\textsc{past} 	take.\textsc{past} 	he 	the 	pen\\
\trans  `He says that he took the pen.'
\end{xlist}
\exi{}[]{\hfill(\citealt{cottell1995}, as cited in \citealt{Koppen2017}: (3))}
\end{exe}

\noindent Based especially on evidence from dialectal variants of \ili{Romance} languages (e.g. \citealt{Paoli2003, mascarenhas2007, villa2012, GiP2014, munaro2016}), it has been proposed that the \isi{doubling} \isi{complementizer} in those languages sits in the head of the topic projection. The main evidence for this position is the fact that the \isi{doubling} \isi{complementizer} is never found after a \isi{focused phrase}, as pointed out for \ili{Spanish}  by \cite{villa2012}, consider the contrast between \REF{ex:plesnicar:37a} and \REF{ex:plesnicar:37b} (see also \citealt{ledgeway2005}  for comparable data from older varieties of \ili{Italian}).

\begin{exe}
\ex \begin{xlist} \label{ex:plesnicar:37}
\ex[]{\label{ex:plesnicar:37a}
\gll Me	dijeron	que	{a tu} primo,	que	\textsc{solo} \textsc{dos} \textsc{portátiles}	le	robaron\hspace{0.3cm}	\minsp{(} no	tres).\\
\textsc{cl}	said that your	cousin	that only two laptops	\textsc{cl}	stole {} not    three\\
\trans  `They told me that it was only two laptops that your cousin got stolen, not three.'}
\ex [*] {\label{ex:plesnicar:37b}
Me dijeron que \textsc{solo} \textsc{dos} \textsc{portátiles}, que le robaron a tu primo (no tres).
\glt \xspace\hfill(\citealt{villa2012}: 30, (24a) and (24b))}

\end{xlist}
\end{exe}

\noindent This analysis cannot be adopted for our \ili{Slovenian} data, however, because it wrongly predicts that \REF{ex:plesnicar:38} below will be unacceptable, since the \isi{doubling} \isi{complementizer} in it is found after a \isi{focused phrase}, specifically, after the contrastively focused \textit{dve pivi} `two beers'.


\begin{exe}
\ex \label{ex:plesnicar:38}
\gll Rekel je,	da	\textsc{dve}	\textsc{pivi} da	je včeraj zvečer spil \minsp{(} in	ne	treh).\\
  said	\textsc{aux} that two beers	that \textsc{aux} yesterday evening drink {} and not three\\
\trans `He said that he only drank two beers yesterday evening (and not three).'
\end{exe}

\noindent In fact, \REF{ex:plesnicar:38} could be taken as suggesting the \isi{head of finiteness phrase} as the site of the \isi{doubling} \isi{complementizer}, because the \isi{left periphery} of the \isi{embedded clause} is usually seen as not featuring a topic phrase located after the contrastive \isi{focus} phrase to potentially host the \isi{doubling} \isi{complementizer} \citep{rizzi1997}, and then the next available position (moving downward along the CP field) for the \isi{doubling} \isi{complementizer} in this particular case is the \isi{head of finiteness phrase}. Although on the other hand, given \citeposst{mismas2015} structure of \ili{Slovenian} \isi{left periphery} in \figref{three} above, \ili{Slovenian} topic phrases can appear to the right of a \isi{focused phrase}, in which case \REF{ex:plesnicar:38} above may not be conclusive.

Furthermore, multiple occurrences of complementizers in the CP field are, contrary to what has been observed for \ili{Romance} languages (e.g. \citealt{ledgeway2005, mascarenhas2007, villa2012}), not acceptable with multiple topicalized phrases in \ili{Slovenian}: compare \REF{ex:plesnicar:41} from European \ili{Portuguese}  and \REF{ex:plesnicar:42a} from \ili{Slovenian}. In \ili{Slovenian}, only the \isi{complementizer} that opens and the lower one that closes the \isi{complementizer system} can be realized, as in \REF{ex:plesnicar:42b}, supporting our claim that the position in which \isi{doubling} complementizers are located in \ili{Slovenian} is the \isi{head of finiteness phrase}, as well as again suggesting that the analysis developed for \ili{Romance} cannot be adopted for \ili{Slovenian}.

\begin{exe}
\ex \label{ex:plesnicar:41}
\gll Acho 	que 	amanh{ã} 	que 	a 	Ana 	que 	vai 	conseguir acabar 	o 	trabalho.  \\
  think 	that 	tomorrow 	that 	the 	Ana 	that 	will 	manage {to finish} 	the 	assignment\\
\trans `I think that tomorrow Ana will manage to finish the assignment.'
\exi{}[]{\hfill(\citealt{mascarenhas2007}: 6, (20))}
\end{exe}

\begin{exe}
\ex \begin{xlist}\label{ex:plesnicar:42}
\ex [*] {\label{ex:plesnicar:42a}
\gll Ne	morem	verjet,	 da	ko	pospravlja	stanovanje,	da Andreja	da	pomete		smeti	pod	preprogo.\\
	not	can	believe	 that	when	clean		apartment	that Andreja	that	sweep		dirt	under	rug\\
\trans  Intended: `I cannot believe that Andreja sweeps the dirt under the rug  when she cleans the apartment.'} %spremenila prevod
\ex[]{\label{ex:plesnicar:42b}
\gll Ne morem verjet, 	da 	ko 	pospravlja 	stanovanje, da Andreja pomete 	smeti 	pod preprogo. \\
	not	can	believe	that	when	clean		apartment	that Andreja	sweep		dirt	under	rug\\
\trans  `I cannot believe that Andreja sweeps the dirt under the rug  when she cleans the apartment.'}

\end{xlist}
\end{exe}

\noindent In sum, this section showed that in \ili{Slovenian}, contrary to what has been found for the \ili{Romance} languages, the \isi{doubling} \isi{complementizer} can occur after a \isi{focused phrase}, and it is only the \isi{complementizer} that opens and the one that closes the \isi{complementizer system} that can be realized in sentences with multiple topicalized phrases, both of which separate \ili{Slovenian} \isi{complementizer doubling} from the \isi{doubling} in \ili{Romance} languages. Furthermore, it was also suggested that the position hosting \isi{doubling} complementizers in \ili{Slovenian} is the \isi{head of finiteness phrase}. The main piece of support for this was the restriction on \isi{complementizer} \isi{doubling} set by the \isi{finiteness} system of the clause.

% \subsection{Further issues} \label{s5.1}
When a \isi{focused phrase} is present in the sentence structure, \isi{complementizer} tripling is also possible, with the focused and topicalized phrases freely ordered, as shown in \REF{ex:plesnicar:43} and \REF{ex:plesnicar:44}.

\begin{exe}
\ex \label{ex:plesnicar:43}
\gll Rekel	mu	je,	da	\textsc{tri} \textsc{knjige} da	če	hoče	naredit	izpit, da	mora 	prebrat (in ne dveh).\\
  said	him	\textsc{aux} that	three books	that	if	want	pass	exam that	need	read	(and not two)\\
\trans `He told him that that he should read three books (not two) if he wants to pass the exam.'
\end{exe}

\begin{exe}
\ex \label{ex:plesnicar:44}
\gll Rekel 	mu 	je, 	da 	če 	hoče 	naredit 	izpit, 	da 	\textsc{tri} \textsc{knjige} 	da 	mora 	prebrat 	(in ne dveh).\\
 said 	him	\textsc{aux} that if want pass exam that	three books that need	read (and not two)\\
\trans `He told him that if he wants to pass the exam he should read three books (not two).'
\end{exe}

\noindent As is the case with \isi{complementizer doubling}, \isi{complementizer tripling} constructions also seem to show no significant difference in meaning when the second or the third \isi{complementizer} is present in the structure. In contrast to \isi{complementizer doubling}, \isi{complementizer tripling} poses an additional question about the exact position of the linearly second \isi{complementizer}. Especially given that the focused and topicalized phrases can surface in either order, it may well be that the position of the second \isi{complementizer} of \REF{ex:plesnicar:43} is actually not the same as the position of the second \isi{complementizer} of \REF{ex:plesnicar:44}. What seems clear is that in a structure like \figref{3} above, the second complementizers of both \REF{ex:plesnicar:43} and \REF{ex:plesnicar:44} should sit in one of the available projections between InterP and FinP. Still, for any real claims to be made regarding the tripling phenomenon, a more in-depth investigation will be necessary, but this would go well beyond the scope of this paper and will thus have to wait for future work.

\section{Conclusion}

In this paper we treated \ili{Slovenian} \isi{complementizer doubling} as an inherent property of the \isi{subordinate clause} which can be realized only if there is some additional element in the specifier position of some projection between force phrase and \isi{finiteness phrase}.

We showed that the phrase occurring between the two complementizers of the \isi{embedded clause} must have moved there from a lower structural position, and we claimed that the presence of the \isi{doubling} \isi{complementizer} does not result in any difference in the meaning of the sentence. We suggested that since \isi{complementizer doubling} is possible only in complement clauses, this must be a \isi{syntactic phenomenon} rather than a disfluency-type repetition.

We also argued that the doubly-filled \textsc{comp} sentences of \cite{Bacskai2016} and our \isi{complementizer doubling} constructions are separate phenomena, or ra\-ther, that the \isi{complementizer} in \citeauthor{Bacskai2016}'s doubly-filled \textsc{comp} sentences and our \isi{doubling} \isi{complementizer} are two different elements. The realization of the doubly-filled \textsc{comp} construction requires just one CP projection, whereas this does not suffice for our \isi{complementizer} \isi{doubling construction} which requires at least three structural positions. Furthermore, we showed that \isi{complementizer doubling} is also possible in embedded yes/no questions, which confirms the idea that \ili{Slovenian} \isi{complementizer doubling} is constrained by the requirements of the \isi{matrix predicate}.

We suggested that \ili{Slovenian} \isi{complementizer} \isi{doubling} can be nicely accounted for if we assume a split CP model (\citeauthor{rizzi1997} \citeyear{rizzi1997, rizzi2001}). A comparison between the characteristics of \ili{Slovenian} and \ili{Romance} \isi{complementizer doubling} revealed that the location occupied by our \isi{doubling} \isi{complementizer} cannot be the head of topic projection. We proposed that in \ili{Slovenian}, the \isi{doubling} \isi{complementizer} is hosted by the \isi{head of finiteness phrase}, while the first \isi{complementizer} is the highest element of the \isi{embedded clause} and as such located in \isi{force projection}.




\section*{Abbreviations}

\begin{tabularx}{.45\textwidth}{lX}
\textsc{aux}&auxiliary\\
\textsc{cl}&{clitic}\\
\textsc{decl}&declarative\\
\textsc{fut}&future\\
\end{tabularx}
\begin{tabularx}{.45\textwidth}{lX}
\textsc{int}&interogative\\
\textsc{pres}&present {tense}\\
\textsc{past}&past {tense}\\
\textsc{ptcl}&particle\\
\end{tabularx}

\section*{Acknowledgements}
For guidance and support I am grateful to my mentor doc. dr. Rok Žaucer. I owe many thanks to prof. dr. Franc Marušič and dr. Petra Mišmaš for comments and suggestions. I would also like to thank the reviewers for their time spent on reviewing my manuscript and their comments, which helped me to improve the paper.

\sloppy
\printbibliography[heading=subbibliography,notkeyword=this]
\il{Slovenian|)}
\end{document}
