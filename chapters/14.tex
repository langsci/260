\documentclass[output=paper, newtxmath, colorlinks, citecolor=brown]{langsci/langscibook}
\ChapterDOI{10.5281/zenodo.3764869}
%\bibliography{localbibliography}
%% add all extra packages you need to load to this file  
\usepackage{tabularx} 
\definecolor{lsDOIGray}{cmyk}{0,0,0,0.45}

\usepackage{xassoccnt}
\newcounter{realpage}
\DeclareAssociatedCounters{page}{realpage}
\AtBeginDocument{%
  \stepcounter{realpage}
}

%%%%%%%%%%%%%%%%%%%%%%%%%%%%%%%%%%%%%%%%%%%%%%%%%%%%
%%%           Examples                           %%%
%%%%%%%%%%%%%%%%%%%%%%%%%%%%%%%%%%%%%%%%%%%%%%%%%%%%  
%% if you want the source line of examples to be in italics, uncomment the following line
% \renewcommand{\exfont}{\itshape}
\usepackage{lipsum}
\usepackage{langsci-optional}
\usepackage{./langsci-osl}
\usepackage{langsci-lgr}
\usepackage{langsci-gb4e}
\usepackage{stmaryrd}
\usepackage{pifont} % needed for checkmark \ding{51} and cross \ding{55}
\usepackage[linguistics]{forest}

% ch06
\usepackage[euler]{textgreek}
% ch07
\usepackage{soul}
\usepackage{graphicx}
% ch08, ch14
\usepackage{multicol}
% ch11
\usepackage{fnpct}
% ch13
\usepackage{scrextend}
\usepackage{enumitem}
% ch14
\usepackage{tabto}
\usepackage{multirow}

\usepackage{langsci-cgloss}

%\newcommand{\smiley}{ :) }

% non-italics in examples

\renewcommand{\eachwordone}{\upshape}

% non-italics in examples in footnotes

\renewcommand{\fnexfont}{\footnotesize\upshape}
\renewcommand{\fnglossfont}{\footnotesize\upshape}
\renewcommand{\fntransfont}{\footnotesize\upshape}
\renewcommand{\fnexnrfont}{\fnexfont\upshape}

% chapter03 goncharov

\newcommand{\p}{\textsc{pfv\ }}
\newcommand{\im}{\textsc{ipfv\ }}

\makeatletter
\let\thetitle\@title
\let\theauthor\@author 
\makeatother

\newcommand{\togglepaper}[1][0]{  
  \addbibresource{../localbibliography.bib}  
  \papernote{\scriptsize\normalfont
    \theauthor.
    \thetitle. 
    To appear in: 
    Change Volume Editor \& in localcommands.tex 
    Change volume title in localcommands.tex
    Berlin: Language Science Press. [preliminary page numbering]
  }
  \pagenumbering{roman}
  \setcounter{chapter}{#1}
  \addtocounter{chapter}{-1}
}

\providecommand{\orcid}[1]{}
\IfFileExists{../localcommands.tex}{
  % add all extra packages you need to load to this file  
\usepackage{tabularx} 
\definecolor{lsDOIGray}{cmyk}{0,0,0,0.45}

\usepackage{xassoccnt}
\newcounter{realpage}
\DeclareAssociatedCounters{page}{realpage}
\AtBeginDocument{%
  \stepcounter{realpage}
}

%%%%%%%%%%%%%%%%%%%%%%%%%%%%%%%%%%%%%%%%%%%%%%%%%%%%
%%%           Examples                           %%%
%%%%%%%%%%%%%%%%%%%%%%%%%%%%%%%%%%%%%%%%%%%%%%%%%%%%  
%% if you want the source line of examples to be in italics, uncomment the following line
% \renewcommand{\exfont}{\itshape}
\usepackage{lipsum}
\usepackage{langsci-optional}
\usepackage{./langsci-osl}
\usepackage{langsci-lgr}
\usepackage{langsci-gb4e}
\usepackage{stmaryrd}
\usepackage{pifont} % needed for checkmark \ding{51} and cross \ding{55}
\usepackage[linguistics]{forest}

% ch06
\usepackage[euler]{textgreek}
% ch07
\usepackage{soul}
\usepackage{graphicx}
% ch08, ch14
\usepackage{multicol}
% ch11
\usepackage{fnpct}
% ch13
\usepackage{scrextend}
\usepackage{enumitem}
% ch14
\usepackage{tabto}
\usepackage{multirow}

\usepackage{langsci-cgloss}

  \newcommand{\smiley}{ :) }

% non-italics in examples

\renewcommand{\eachwordone}{\upshape}

% non-italics in examples in footnotes

\renewcommand{\fnexfont}{\footnotesize\upshape}
\renewcommand{\fnglossfont}{\footnotesize\upshape}
\renewcommand{\fntransfont}{\footnotesize\upshape}
\renewcommand{\fnexnrfont}{\fnexfont\upshape}

% chapter03 goncharov

\newcommand{\p}{\textsc{pfv\ }}
\newcommand{\im}{\textsc{ipfv\ }}

\makeatletter
\let\thetitle\@title
\let\theauthor\@author 
\makeatother

\newcommand{\togglepaper}[1][0]{  
  \addbibresource{../localbibliography.bib}  
  \papernote{\scriptsize\normalfont
    \theauthor.
    \thetitle. 
    To appear in: 
    Change Volume Editor \& in localcommands.tex 
    Change volume title in localcommands.tex
    Berlin: Language Science Press. [preliminary page numbering]
  }
  \pagenumbering{roman}
  \setcounter{chapter}{#1}
  \addtocounter{chapter}{-1}
}

\providecommand{\orcid}[1]{}
  %% hyphenation points for line breaks
%% Normally, automatic hyphenation in LaTeX is very good
%% If a word is mis-hyphenated, add it to this file
%%
%% add information to TeX file before \begin{document} with:
%% %% hyphenation points for line breaks
%% Normally, automatic hyphenation in LaTeX is very good
%% If a word is mis-hyphenated, add it to this file
%%
%% add information to TeX file before \begin{document} with:
%% %% hyphenation points for line breaks
%% Normally, automatic hyphenation in LaTeX is very good
%% If a word is mis-hyphenated, add it to this file
%%
%% add information to TeX file before \begin{document} with:
%% \include{localhyphenation}
\hyphenation{
Ro-ma-no-va
Isa-čen-ko
}

\hyphenation{
Ro-ma-no-va
Isa-čen-ko
}

\hyphenation{
Ro-ma-no-va
Isa-čen-ko
}

  \togglepaper[14]%%chapternumber
}{}
%\togglepaper[14]
\title{Several quantifiers are different than others: Polish indefinite numerals}

\author{Marcin Wągiel\affiliation{Masaryk University in Brno}}

\abstract{In this paper, I examine properties of two Polish indefinite quantifiers, namely \textit{ileś} `some, some number’ and \textit{kilka} `several, a few’. I argue that they share morpho-syntactic properties with cardinal numerals rather than with vague quantifiers such as \textit{mało} `little, few' and \textit{dużo} `much, many' and propose that they should be modeled as involving a built-in classifier comprising both a measure function and choice function. The difference between the two indefinites boils down to the type of set the choice function selects a member from and the type of measure function that is employed.

	\keywords{indefinites, numerals, quantifiers, choice functions, classifiers, Polish}
}


\begin{document}
	\maketitle
	\shorttitlerunninghead{Several quantifiers are different than others}
\il{Polish|(}
	\section{Introduction}\label{introduction}

	For some time, properties of different series of \ili{Slavic} indefinites have been successfully explored (e.g., \citealt{blaszczak2001investigation}, \citealt{testelets_bylinina2005sluicing}, \citealt{yanovich2005choice}, \citealt{geist2008specificity}, \citealt{pereltsvaig2008russian}, \citealt{eremina2012semantics}, \citealt{docekal_strachonova2015freedom}, \citealt{richtarcikova2015epistemic}, \citealt{simik2015epistemic}, \citealt{strachonova2016ceska}). However, one particular class of \isi{indefinite} expressions seems to have been somewhat overlooked, namely \isi{indefinite quantifiers} such as those exemplified in \REF{ex:slavic-indefinite-quantifiers}.

	\ea \label{ex:slavic-indefinite-quantifiers} \ea neskol'ko \hfill (\ili{Russian})
	\ex několik \hfill (\ili{Czech})
	\ex nyakolko \hfill (\ili{Bulgarian})
	\ex \gll nekoliko\\
	several/a few\\ \hfill ({Bosnian/Croatian/Serbian})
    \z
    \z

	\noindent Remarkably, a similar gap is discernible in a long and prolific research on quantifiers since certain characteristics of expressions corresponding to \ili{English} \textit{several} remain surprisingly understudied (with the notable exception of \citealt{kayne2007several}). In this paper, I will examine an alternation involving two types of \ili{Polish} \isi{indefinite} quantifiers, such as those seen in \REF{ex:polish-indefinite-quantifiers}. In terms of terminology, I will follow the \ili{Polish} descriptive tradition and refer to such expressions as \textsc{\isi{indefinite} numerals}, a term which I take to be legitimate in view of the data discussed in the subsequent sections.

	\ea \label{ex:polish-indefinite-quantifiers} \ea \gll \text{ileś (tam)}\\
	some/some number\\
	\ex \gll kilka\\
	several/a few\\
    \z
    \z

	\noindent Though the alternation does not hold in every \ili{Slavic} language, it does not seem to be a \ili{Polish} idiosyncrasy, as attested by the prima facie similar contrast between \ili{Russian} \textit{skol'ko-to} and \textit{neskol'ko} `several'. The approach developed here is intended to fit a broader research program dedicated to accounting for semantic properties of distinct types of \ili{Slavic} \isi{numeral} expressions \citep{docekal2012atoms,docekal2013numerals,wagiel2014boys,wagiel2015sums,wagiel-toappear-entities,docekal_wagiel2018event}.  %\citeyear{wagiel-toappear-grammatical},
	Thus, the insights presented here might have wider applicability, at least within \ili{Slavic}.

	The paper is outlined as follows. In \sectref{sec:cardinals-indefinite-numerals-and-vague-quantifiers}, I employ a battery of tests to determine morpho-syntactic and semantic properties of the \ili{Polish} \isi{indefinite} numerals \textit{kilka} and \textit{ileś}. In \sectref{sec:some-intriguing-contrasts}, I discuss additional data concerning the alternation in question including the evidence in favor of specificity. In \sectref{sec:setting-the-stage}, I introduce the basic machinery necessary for the analysis: i.e., choice functions, measure functions, and the intersective theory of \isi{cardinal} numerals. In \sectref{sec:putting-pieces-together}, I develop a morpho-semantic approach to account for the discussed data. Finally, \sectref{sec:conclusion} concludes the article.

	\section{Cardinals, indefinite numerals, and vague quantifiers}\label{sec:cardinals-indefinite-numerals-and-vague-quantifiers}

	\subsection{Polish indefinite series}\label{sec:polish-indefinite-series}

	Similar to other \ili{Slavic} languages, there are several series of \isi{indefinite} expressions in \ili{Polish} and \tabref{table:indefinite-series-in-polish} gives the paradigm for the main ones. Based on morphological evidence, it seems straightforward to assume that \ili{Polish} indefinites constitute derivationally complex expressions which can be decomposed into a wh-element and an \isi{indefinite} suffix. In addition, the indefinites in the \textit{-ś} series can be followed by an optional \isi{pronoun} \textit{tam} `there' which can express either a great level of ignorance, or depreciative attitude (cf. \citealt{bylinina2010depreciative}).

	\begin{table}[h]
		\centering
		\caption{Indefinite series in Polish}
		\label{table:indefinite-series-in-polish}
		\begin{tabularx}{\textwidth}{llXXX}
			\lsptoprule
			\multicolumn{2}{l}{wh-word} & -ś     & -kolwiek     & -bądź      \\ \midrule
			kto     & `who'   & ktoś (tam)   & ktokolwiek   & kto bądź   \\
			co      & `what'   & coś (tam)   & cokolwiek    & co bądź    \\
			gdzie   & `where'  & gdzieś (tam) & gdziekolwiek & gdzie bądź \\
			kiedy   & `when'   & kiedyś (tam) & kiedykolwiek & kiedy bądź \\
			jak     & `how'    & jakoś (tam) & jakkolwiek   & jak bądź   \\
			jaki    & `what/which'   & jakiś (tam) & jakikolwiek  & jaki bądź  \\
			ile     & `how much/many'  & ileś (tam)  & ilekolwiek   & ile bądź   \\ \lspbottomrule
		\end{tabularx}
	\end{table}

	As the last row in \tabref{table:indefinite-series-in-polish} shows, the \ili{Polish} wh-word \textit{ile} `how much/ many' can take the \isi{indefinite} morpheme \textit{-ś} as well as the \isi{free choice item} (\isi{FCI}) markers \textit{-kolwiek} and \textit{bądź}. Unlike other wh-words, it is incompatible with the negative prefix \textit{ni-} (*\textit{nile} vs. \textit{nikt} `no one') and the depreciative \isi{FCI} element \textit{byle} (*\textit{byle ile} vs. \textit{byle kto} `anyone (someone considered unworthy)') but it can occur within grammaticalized expressions such as \textit{bóg wie ile} `God knows how much/many' and \textit{chuj wie ile} `who the fuck knows how much/many'. Despite the fact that the \textit{ile} series is somewhat defective compared to other wh-words, \textit{ileś} is a proper \isi{indefinite} whose meaning could be probably best paraphrased in \ili{English} as \textit{some number} or \textit{some amount}.

	On the other hand, \textit{kilka} seems to be semantically more restricted. According to the intuition of a majority of \ili{Polish} native speakers it refers to a number between 3 and 9.\footnote{Such an intuition is corroborated by the lexical entries in standard dictionaries of the contemporary \ili{Polish} language though perhaps it might be subject to some extent to vagueness or interpersonal variation.} Unlike \textit{ileś}, it does not seem to be derivationally complex. In terms of etymology, it emerged from the obsolete wh-word \textit{koliko} `how much/many' (compare, e.g., \ili{Czech} \textit{kolik} `how much/many' $\sim$ \textit{několik} `some/several') and the cluster \textit{-il-} is arguably related to \textit{ile} (see \citealt{bankowski2000etymologiczny}). However, from a synchronic perspective this relationship is completely opaque and for simplicity I will assume that \textit{kilka} is not a derived form and can only be decomposed into the stem \textit{kilk-} and the following \isi{inflectional} marker.

	I will refrain here from discussing the FCIs \textit{ilekolwiek} and \textit{ile bądź} `any amount/ number' and for the purposes of this paper I will assume that whatever approach accounts for, e.g., the \textit{kto} `who' $\sim$ \textit{ktoś} `someone' $\sim$ \textit{ktokolwiek} `anyone' series (e.g., \citealt{kadmon_landman1993any}, \citealt{aloni2007free}, \citealt{chierchia2013logic}), could also be applied to the \textit{ile} `how much/many' $\sim$ \textit{ileś} `some amount/number' $\sim$ \textit{ilekolwiek} `any amount/number' alternation. Therefore, in the following text I will \isi{focus} exclusively on discussing novel data concerning the distribution as well as morpho-syntactic and semantic properties of \textit{ileś} and \textit{kilka}.

	To begin with, I will assume that two justifiable hypotheses can be formulated with respect to the nature of the analyzed indefinites: (i)~\textit{ileś} and \textit{kilka} are similar to other vague quantifiers or (ii)~to \isi{cardinal} numerals. I will confine my \isi{focus} to testing properties of these expressions in comparison to \textit{pięć} `five' on the one hand and \textit{mało} `few/little' and \textit{dużo} `much/many' as two representatives of a wider class of vague quantifiers (including lexical items such as \textit{sporo} `much/many', \textit{trochę} `some', \textit{niemało} `quite a lot', \textit{niedużo} `not much/many', and \textit{masę} `plenty') on the other. Although due to some lexical idiosyncrasies not every representative of that class has all the discussed properties, e.g., \textit{trochę}, \textit{masę}, and \textit{sporo} are not gradable, the general picture is roughly as presented below.

	\subsection{Inflection}\label{sec:inflection}

	I will start with the observation that in many respects \ili{Polish} \textit{kilka} and \textit{ileś} pattern with higher cardinals (i.e., five and higher) rather than with vague quantifiers such as \textit{mało} and \textit{dużo}. Similar to \textit{pięć}, both \textit{kilka} and \textit{ileś (tam)} agree in gender with a modified NP and display the well-documented virile vs. non-virile alternation (e.g., \citealt{miechowicz-mathiasen2011syntax}). On the other hand, \textit{mało} and \textit{dużo} display no reflex of gender agreement with a modified NP, and thus have virile forms, see \REF{ex:gender-agreement-quantifiers}--\REF{ex:gender-agreement-cardinals}.

	\ea \label{ex:gender-agreement-quantifiers} \ea \gll \minsp{\{} Mało dziewczyn / mało chłopców\} przyszło.\\
	{} few girls.\textsc{nv} {} few boys.\textsc{v} came\\
	\glt `A few \{girls / boys\} came.'
	\ex \gll \minsp{\{} Dużo dziewczyn / dużo chłopców\} przyszło.\\
	{} many girls.\textsc{nv} {} many boys.\textsc{v} came\\
	\glt `Many \{girls / boys\} came.'
    \z
    \z

	\ea \label{ex:gender-agreement-cardinals} \ea \gll \minsp{\{} Pięć dziewczyn / pięciu chłopców\} przyszło.\\
	{} five.\textsc{nv} girls.\textsc{nv} {} five.\textsc{v} boys.\textsc{v} came\\
	\glt `Five \{girls / boys\} came.'
	\ex \gll \minsp{\{} Kilka dziewczyn / kilku chłopców\} przyszło.\\
	{} several.\textsc{nv} girls.\textsc{nv} {} several.\textsc{v} boys.\textsc{v} came\\
	\glt `Several \{girls / boys\} came.'
	\ex \gll \minsp{\{} Ileś dziewczyn / Iluś chłopców\} przyszło.\\
	{} some.\textsc{nv} girls.\textsc{nv} {} some.\textsc{v} boys.\textsc{v} came\\
	\glt `Some \{girls / boys\} came.'
	\z
    \z

	\noindent Another morpho-syntactic similarity between \isi{indefinite} numerals and cardinals is that, unlike \textit{mało} and \textit{dużo}, the indefinites \textit{ileś} and \textit{kilka} do not take a \isi{comparative} and \isi{superlative}, see \REF{ex:comparison-quantifiers}--\REF{ex:comparison-cardinals}.

	\ea \label{ex:comparison-quantifiers} \ea \gll {mało $\sim$} {mniej $\sim$} najmniej\\
	few fewer fewest\\
	\ex \gll {dużo $\sim$} {więcej $\sim$} najwięcej\\
	much more most\\
	\z
    \z

	\ea \label{ex:comparison-cardinals} \ea \gll pięć $\sim$ *\hspace{-2pt} \minsp{\{} pięciej / bardziej pięć\} $\sim$ *\hspace{-2pt} \minsp{\{} najpięciej / najbardziej {pięć\}}\\
	five {} {} {} five.\textsc{cmpr} {} more five {} {} {} five.\textsc{sprl} {} most five\\
	\ex \gll kilka $\sim$ *\hspace{-2pt} \minsp{\{} kilkiej / bardziej {kilka\}} $\sim$ *\hspace{-2pt} \minsp{\{} najkilkiej / najbardziej {kilka\}}\\
	several {} {} {} several.\textsc{cmpr} {} more several {} {} {} several.\textsc{sprl} {} most several\\
	\ex \gll ileś $\sim$ *\hspace{-2pt} \minsp{\{} ilesiej / bardziej {ileś\}} $\sim$ *\hspace{-2pt} \minsp{\{} najilesiej / najbardziej {ileś\}}\\
	some {} {} {} some.\textsc{cmpr} {} more some {} {} {} some.\textsc{sprl} {} most some\\
	\z
    \z

	\noindent In the following sections, I will test the grammaticality of \textit{kilka} and \textit{ileś} in multiple environments in comparison to \isi{cardinal} numerals and the quantifiers \textit{mało} and \textit{dużo}. I will start with different types of modifiers.

	\subsection{Degree and numeral modifiers}\label{sec:degree-and-numeral-modifiers}

	One can distinguish between two types of modifiers that can combine with quantifiers: (i)~degree modifiers such as \textit{very (much)} and (ii)~\isi{numeral} modifiers such as \textit{over (five)}.\footnote{\cite{nouwen2010two} further distinguishes between class A and B \isi{numeral} modifiers. However, for the purpose of this paper a simplified view is entirely sufficient.} Degree modifiers are compatible with quantifiers such as \textit{mało} and \textit{dużo} but cannot combine with \isi{cardinal} numerals. On the other hand, \isi{numeral} modifiers can target cardinals but fail to modify gradable quantifiers. Interestingly, the \isi{indefinite} \isi{numeral} \textit{kilka} behaves exactly like cardinals. The examples in \REF{ex:degree-modifiers-malo}--\REF{ex:numeral-modifiers-cardinals-kilka} illustrate the pattern.
	\vspace{2cm}
	\begin{multicols}{2}

		\ea \label{ex:degree-modifiers-malo} \ea \gll bardzo mało\\
		very few\\
		\ex \gll dość mało\\
		rather few\\
		\ex \gll zbyt mało\\
		too few\\
		\ex \gll tak mało\\
		so few\\
		\ex \gll niemało\\
		not.few\\
		\z
        \z
		\ea \label{ex:numeral-modifiers-malo} \ea[*]{\gll ponad mało\\
		over few\\}
		\ex[*]{\gll najwyżej mało\\
		up.to few\\}
		\ex[*]{\gll około mało\\
		around few\\}
		\ex[*]{\gll co najmniej mało\\
		at least few\\}
		\ex[*]{\gll od mało do stu\\
		from few to 100\\}
		\z
        \z

        \textcolor{white}{.}
        \newline
        \textcolor{white}{.}
        \newline
        \textcolor{white}{.}
        \newline

	\columnbreak

		\ea \label{ex:degree-modifiers-cardinals-kilka} \ea[*]{\gll bardzo \minsp{\{} pięć / {kilka\}}\\
		very {} five {} several\\}
		\ex[*]{\gll dość \minsp{\{} pięć / {kilka\}}\\
		rather {} five {} several\\}
		\ex[*]{\gll zbyt \minsp{\{} pięć / {kilka\}}\\
		too {} five {} several\\}
		\ex[*]{\gll tak \minsp{\{} pięć / {kilka\}}\\
		so {} five {} several\\}
		\ex[*]{\gll niepięć / *\hspace{-2pt} niekilka\\
		not.five {} {} not.several\\}
		\z
        \z

		\ea \label{ex:numeral-modifiers-cardinals-kilka} \ea \gll ponad \minsp{\{} pięć / {kilka\}}\\
		over {} five {} several\\
		\ex \gll najwyżej \minsp{\{} pięć / {kilka\}}\\
		up.to {} five {} several\\
		\ex \gll około \minsp{\{} pięciu / {kilku\}}\\
		around {} five.\textsc{gen} {} several.\textsc{gen}\\
		\ex \gll co najmniej \minsp{\{} pięć / {kilka\}}\\
		at least {} five {} several\\
		\ex \gll od \minsp{\{} pięciu / {kilku\}} do stu\\
		from {} five.\textsc{gen} {} several.\textsc{gen} to 100\\
		\z
        \z

	\end{multicols}

	\noindent Similar to cardinals and \textit{kilka}, the \isi{indefinite} \textit{ileś} is incompatible with degree modifiers, see \REF{ex:degree-modifiers-iles}. Nevertheless, unlike the expressions discussed above it seems to be degraded with most \isi{numeral} modifiers. Notice, however, that despite this fact, the contrast between \REF{ex:numeral-modifiers-iles} and \REF{ex:numeral-modifiers-malo} is still detectable.

	\begin{multicols}{2}

		\ea \label{ex:degree-modifiers-iles}
		\ea[*]{\gll bardzo {ileś (tam)}\\
		very some\\}
		\ex[*]{\gll dość {ileś (tam)}\\
		rather some\\}
		\ex[*]{\gll zbyt {ileś (tam)}\\
		too some\\}
	\columnbreak
	\ex[*]{\gll tak {ileś (tam)}\\
		so some\\}

		\ex[*]{\gll {nieileś (tam)}\\
		not.some\\}
		\z
        \z
		\end{multicols}

\begin{multicols}{2}
		\ea \label{ex:numeral-modifiers-iles} \ea[?]{\gll ponad {ileś (tam)}\\
		over some\\}\label{ex:numeral-modifiers-iles-ponad}
		\ex[?]{\gll najwyżej {ileś (tam)}\\
		up.to some\\}
		\ex[?]{\gll około {iluś (tam)}\\
		around some.\textsc{gen}\\}
	\columnbreak
	\ex[?]{\gll co najmniej {ileś (tam)}\\
		at least some\\}
		\ex[]{\gll od {iluś (tam)} do stu\\
		from some.\textsc{gen} to 100\\}
		\z
        \z

	\end{multicols}

	\noindent I speculate that the reason that the acceptability of \textit{ileś} with \isi{numeral} modifiers is reduced is its high level of indefiniteness. Since such modifiers compare more or less defined values, at least some approximation with respect to the targeted set of numbers is required. Out of the blue \REF{ex:numeral-modifiers-iles-ponad} sounds odd, but if a proper context sets a plausible range of possible values, it becomes perfectly acceptable, as attested in an example from the National Corpus of \ili{Polish} (NCP) provided in \REF{ex:numeral-modifiers-iles-ncp}.

	\ea \gll [\dots] jeśli stan załogi wynosi ponad \text{ileś tam} osób [\dots] powinien być zespół muzyczny [\dots]\\
	{} if state crew equals over some.number people {} it.should be band musical\\
	\glt `[\dots] if a crew amounts to more than some number of people [\dots] there should be a music band arranged [\dots]'\label{ex:numeral-modifiers-iles-ncp}
    \z

\noindent	All in all, the discussed data seem to indicate the distinction between quantifiers \textit{mało} and \textit{dużo} on the one hand and cardinals and the indefinites \textit{kilka} and \textit{ileś} on the other. The next test will involve the (un)grammaticality of quantificational NPs where the \isi{quantifier} is modified by the \isi{adjective} or \isi{possessive} \isi{pronoun}.

	\subsection{Adjectival and pronominal modification}\label{sec:adjectival-and pronominal-modification}

	It has been observed that \ili{Polish} cardinals are compatible with agreeing \isi{adjectival} modifiers such as \textit{dobre} `good' in preposition, see \REF{ex:adjectival-modifiers-agreement-cardinals}, (cf. \citealt{babby1987case} and \citealt{miechowicz-mathiasen2011syntax}).\footnote{An anonymous reviewer wonders whether \textit{dobre} in examples such as \REF{ex:adjectival-modifiers-agreement} is in fact an \isi{adjective} and whether it could be analyzed as an \isi{adverbial} element. The case, gender, and number agreement point to the contrary and, as far as I can tell, there is no evidence for the \isi{adverbial} nature of \textit{dobre} in such examples. Furthermore, swapping the standard adverb \textit{dobrze} `well' for \textit{dobre} results in ungrammaticality.} As indicated by the translations, if the preceding AP employs the agreement strategy, it is the referent of the \isi{numeral} that is modified and not the quantified entities; e.g., in \REF{ex:adjectival-modifiers-agreement-cardinals} it is the number of cookies that is good, not necessarily the cookies themselves. Again, \textit{kilka} and \textit{ileś} pattern with \isi{cardinal} numerals in this respect, whereas \textit{mało} and \textit{dużo} do not allow for \isi{adjectival} modification, as witnessed by the ungrammaticality of \REF{ex:adjectival-modifiers-agreement-malo}.\footnote{It seems that there is a dialectal variation since some \ili{Polish} speakers judge examples such as those in \REF{ex:adjectival-modifiers-agreement} as ungrammatical and accept only APs which agree with the noun to precede the quantificational NP. However, to my knowledge for such speakers the use of the genitival form \textit{dobrych} `good' in \REF{ex:adjectival-modifiers-agreement-malo} is still impossible.}

	\ea \label{ex:adjectival-modifiers-agreement} \ea \gll dobre pięć ciasteczek\label{ex:adjectival-modifiers-agreement-cardinals}\\
	good five cookies.\textsc{gen}\\
	\glt `a good five cookies'
	\ex \gll dobre kilka ciasteczek\\
	good several cookies.\textsc{gen}\\
	\glt `a good several cookies'
	\ex \gll dobre {ileś (tam)} ciasteczek\\
	good some cookies.\textsc{gen}\\
	\glt `a good number of cookies'
	\z
    \z

	\ea \label{ex:adjectival-modifiers-agreement-malo} \ea[*]{\gll dobre mało ciasteczek\\
	good few cookies.\textsc{gen}\\
	\glt Intended: `a good few cookies'}
	\ex[*]{\gll dobre dużo ciasteczek\\
	good many cookies.\textsc{gen}\\
	\glt Intended: `a good many cookies'}
	\z
    \z

\largerpage[2]
	\noindent Similarly, both cardinals and \isi{indefinite} numerals allow for \isi{pronominal} modification employing the agreement strategy (cf. \citealt{miechowicz-mathiasen2011syntax}), while expressions such as \textit{mało} and \textit{dużo} do not, as witnessed by the contrast between \REF{ex:pronominal-modifiers-cardinals} and \REF{ex:pronominal-modifiers-malo}.

	\ea \label{ex:pronominal-modifiers-cardinals} \ea \gll \minsp{\{} te / moje\} pięć ciasteczek\\
	{} these {} my five cookies.\textsc{gen}\\
	\glt `these / my five cookies'
	\ex \gll \minsp{\{} te / moje\} kilka ciasteczek\\
	{} these {} my several cookies.\textsc{gen}\\
	\glt `these / my several cookies'
	\ex \gll \minsp{\{} te / moje\} {ileś (tam)} ciasteczek\\
	{} these {} my some cookies.\textsc{gen}\\
	\glt `these / my cookies some number worth'
	\z
    \z

	\ea \label{ex:pronominal-modifiers-malo} \ea[*]{\gll \minsp{\{} te / moje\} mało ciasteczek\\
	{} these {} my few cookies.\textsc{gen}\\
	\glt Intended: `these / my few cookies'}
	\ex[*]{\gll \minsp{\{} te / moje\} dużo ciasteczek\\
	{} these {} my many cookies.\textsc{gen}\\
	\glt Intended: `these / my many cookies'}
	\z
    \z

	\noindent The (in)compatibility with different types of modifiers appears to be a reliable diagnostic for the classification of quantifiers and it suggests a distinction between cardinals and \textit{kilka} and \textit{ileś} on the one hand and \textit{mało} and \textit{dużo} on the other. Yet another test will explore the acceptability of the expressions in question in contexts involving universal quantification and markers forcing obligatory distributive readings.

	\subsection{Universal quantification and distributivity}\label{sec:universal-quantification-and-distributivity}

\largerpage[2]
	It is a well-known fact that \ili{Slavic} numerals can co-occur with the universal \isi{quantifier} within one phrase (cf. \citealt{corbett1978universals}, \citealt{gvozdanovic1999some}, and \citealt{miechowicz-mathiasen2011syntax}). Examples such as those in \REF{ex:universal-quantifier-cardinals} show that, similar to cardinals, the indefinites \textit{kilka} and \textit{ileś} are also licit in such an environment. However, expressions such as \textit{mało} and \textit{dużo} do not allow for modification by a universal \isi{quantifier}, see \REF{ex:universal-quantifier-malo}.

	\ea \label{ex:universal-quantifier-cardinals} \ea \gll wszystkie pięć ciasteczek\\
	all five cookies.\textsc{gen}\\
	\glt `all the five cookies'
	\ex \gll wszystkie kilka ciasteczek\\
	all several cookies.\textsc{gen}\\
	\glt `all the several cookies'
	\ex \gll wszystkie {ileś (tam)} ciasteczek\\
	all some cookies.\textsc{gen}\\
	\glt `all the cookies (where there are some cookies)'
	\z
    \z

	\ea \label{ex:universal-quantifier-malo} \ea[*]{\gll wszystkie mało ciasteczek\\
	all few cookies.\textsc{gen}\\
	\glt Intended: `all the few cookies'}
	\ex[*]{\gll wszystkie dużo ciasteczek\\
	all many cookies.\textsc{gen}\\
	\glt Intended: `all the many cookies'}
	\z
    \z

	\noindent Another contrast relates to distributivity. As observed by \cite{safir_stowell1988binominal} and discussed by \cite{borer2005name}, \ili{English} binominal \textit{each} does not allow the distributive share expressed by DPs involving the \isi{quantifier} \textit{some}. Similarly, there are a number of restrictions on arguments of the distributive preposition \textit{po} in \ili{Polish} which excludes a collective reading of a sentence in which it occurs (\citealt{przepiorkowski2008generalised}). Interestingly, phrases headed by quantifiers such as \textit{mało} and \textit{dużo} are not acceptable as complements of \textit{po}, see \REF{ex:distributive-po-malo}, unlike \textit{kilka} and \textit{ileś} which again pattern with cardinals, see \REF{ex:distributive-po-cardinals}.

	\ea \label{ex:distributive-po-cardinals} \ea \gll Dałem im po pięć ciasteczek.\\
	I.gave them \textsc{distr} five cookies.\textsc{gen}\\
	\glt `I gave them five cookies each.'
	\ex \gll Dałem im po kilka ciasteczek.\\
	I.gave them \textsc{distr} several cookies.\textsc{gen}\\
	\glt `I gave them several cookies each.'
	\ex \gll Dałem im po {ileś (tam)} ciasteczek.\\
	I.gave them \textsc{distr} some cookies.\textsc{gen}\\
	\glt `I gave some cookies to each of them.'
	\z
    \z

	\ea \label{ex:distributive-po-malo} \ea[*]{\gll Dałem im po mało ciasteczek.\\
	I.gave them \textsc{distr} few cookies.\textsc{gen}\\
	\glt Intended: `I gave few cookies to each of them.'}
	\ex[*]{\gll Dałem im po dużo ciasteczek.\\
	I.gave them \textsc{distr} many cookies.\textsc{gen}\\
	\glt Intended: `I gave many cookies to each of them.'}
	\z
    \z

	\noindent It seems that the contrasts discussed here cannot simply stem from, e.g., distinct ranges of vagueness or other superficial differences between the indefinites \textit{kilka} and \textit{ileś} as compared to \textit{mało} and \textit{dużo}. Rather, the data suggest that a much more essential disparity is involved and the expressions in question should be treated as belonging to two distinct classes.

	\subsection{Uncountable NPs}\label{sec:uncountable-nps}

	So far, we have discussed environments in which cardinals pattern both with \textit{kilka} and \textit{ileś}. However, another division can be drawn based on the interaction with uncountable nominals such as mass nouns and pluralia tantum. While cardinals and \textit{kilka} cannot combine directly with such expressions\footnote{I put aside cases where the mass denotation is shifted to the count domain by means of the universal packager or the universal sorter.} and require either a measure word or a specialized \isi{classifier} suffix, \REF{ex:uncountable-nps-cardinals-kilka}, \textit{ileś} patterns in this respect with quantifiers such as \textit{mało}.\footnote{The use of forms such as \textit{pięcioro} and \textit{kilkoro} with pluralia tantum seems to be fading, especially in younger generations. Some speakers, however, still use such expressions and the plurale tantum noun \textit{drzwi} `door' ranks in 11th place as a collocation candidate for the lemma \textit{kilkoro} in the NCP. For more details concerning different uses of suffixed numerals such as \textit{pięcioro} see \cite{wagiel2014boys,wagiel2015sums}.} In particular, it is compatible both with mass nouns and pluralia tantum as well as measure and \isi{classifier} constructions involving such expressions and cannot take the \isi{classifier} suffix, as presented in \REF{ex:uncountable-nps-malo-iles}.

	\ea \label{ex:uncountable-nps-cardinals-kilka} \ea[*]{\gll \minsp{\{} pięć / {kilka\}} wody\\
	{} five {} several water.\textsc{gen}\\}
	\ex[]{\gll \minsp{\{} pięć / {kilka\}} butelek wody\\
	{} five {} several bottles.\textsc{gen} water.\textsc{gen}\\}
	\ex[*]{\gll \minsp{\{} pięć / {kilka\}} nożyczek\\
	{} five {} several scissors.\textsc{gen}\\}
	\ex[]{\gll \minsp{\{} pięć / {kilka\}} par nożyczek\\
	{} five {} several pairs.\textsc{gen} scissors.\textsc{gen}\\}
	\ex[\%]{\gll \minsp{\{} pięcioro / {kilkoro\}} drzwi\\
	{} five.\textsc{cl} {} several.\textsc{cl} door.\textsc{gen.pl}\\}
	\z
    \z

	\ea \label{ex:uncountable-nps-malo-iles} \ea[]{\gll \minsp{\{} mało /  {ileś (tam)\}} wody\\
	{} little {} some water.\textsc{gen}\\}
	\ex[]{\gll \minsp{\{} mało / {ileś (tam)\}} butelek wody\\
	{} few {} some bottles.\textsc{gen} water.\textsc{gen}\\}
	\ex[]{\gll \minsp{\{} mało / {ileś (tam)\}} nożyczek\\
	{} few {} some scissors.\textsc{gen}\\}
	\ex[]{\gll \minsp{\{} mało / {ileś (tam)\}} par nożyczek\\
	{} few {} some pairs.\textsc{gen} scissors.\textsc{gen}\\}
	\ex[*]{\gll \minsp{\{} małoro / {ilesioro (tam)\}} drzwi\\
	{} few.\textsc{cl} {} some.\textsc{cl} door.\textsc{gen.pl}\\}
	\z
    \z

	\noindent Before we move on to discussing more contrasts regarding \textit{kilka} and \textit{ileś}, let us recapitulate the findings so far.

	\subsection{Data summary}\label{sec:data-summary}

\largerpage[2]
	\tabref{table:morpho-syntactic-properties-of-cardinals-and-indefinite-numerals} summarizes morpho-syntactic and distributional properties of \ili{Polish} \isi{indefinite} numerals as compared to cardinals.

	\begin{table}[h]
		\centering
		\caption{Morpho-syntactic properties of cardinals and indefinite numerals}
		\label{table:morpho-syntactic-properties-of-cardinals-and-indefinite-numerals}
		\begin{tabularx}{0.9\textwidth}{@{}lXXXX@{}}
			\lsptoprule
			\multirow{2}{*}{}          & \multicolumn{1}{l}{\textit{pięć}} & \multicolumn{1}{l}{\textit{kilka}} & \multicolumn{1}{l}{\textit{ileś (tam)}} & \multicolumn{1}{l}{\textit{mało}}  \\
			& \multicolumn{1}{l}{\small{`five'}}      & \multicolumn{1}{l}{\small{`several'}}    & \multicolumn{1}{l}{\small{`some'}}            & \multicolumn{1}{l}{\small{`few/little'}} \\
%& \textit{pięć} & \textit{kilka} & \textit{ileś (tam)} &\textit{mało}  \\
%& \small{`five'} & \small{`several'} & \small{`some'} & \small{`few/little'} \\
\midrule
degree modifiers     & * & * & *  & $\checkmark$  \\
comparison   & * & *  & *  & $\checkmark$   \\
mass nouns   & *  & * & $\checkmark$  & $\checkmark$ \\
pluralia tantum & * & * & $\checkmark$ & $\checkmark$ \\
virile vs. non-virile & $\checkmark$ & $\checkmark$ & $\checkmark$ & * \\
universal \isi{quantifier} & $\checkmark$ & $\checkmark$ & $\checkmark$ & * \\
distributive \textsc{po} & $\checkmark$ & $\checkmark$ & $\checkmark$ & *\\
\isi{adjectival} modifiers  & $\checkmark$ & $\checkmark$ & $\checkmark$ & * \\
\isi{pronominal} modifiers & $\checkmark$ & $\checkmark$ & $\checkmark$ & * \\
\isi{numeral} modifiers & $\checkmark$ & $\checkmark$ & ? & * \\
\lspbottomrule
\end{tabularx}
\end{table}

	\normalsize

	As \tabref{table:morpho-syntactic-properties-of-cardinals-and-indefinite-numerals} shows, three patterns can be distinguished within an axis extending over poles constituted by compatibility with \isi{numeral} modifiers on the one hand and degree modifiers on the other. Based on the battery of tests applied in this section, \isi{cardinal} numerals and the \isi{indefinite} \isi{numeral} \textit{kilka} appear to form a logical class which contrasts with the class of vague quantifiers such as \textit{mało} and \textit{dużo}. On the other hand, the \isi{indefinite} \isi{numeral} \textit{ileś} seems to somewhat fall in between the two categories. Although it shares a number of key properties with cardinals, it is not subject to the distributional constraints concerning direct modification of uncountable expressions.

	I conclude that \textit{kilka} is essentially a \isi{cardinal} in disguise, whereas \textit{ileś} seems to be a \isi{numeral} augmented with some more general semantic features. In the next section, I will provide more data that shed new light on the core of the discussed alternation.

	\section{Some intriguing contrasts}\label{sec:some-intriguing-contrasts}

	\subsection{Predicate position}\label{sec:predicate-position}

	 As illustrated in \REF{ex:predicate-position}, \ili{Polish} cardinals and \isi{indefinite} numerals have yet another property in common, namely they both can appear in \isi{predicate position}.

	\ea \label{ex:predicate-position} \ea \gll Tych dziewczyn było pięć.\\
	these girls were five\\
	\glt `The girls were five in number.'
	\ex \gll Tych dziewczyn było kilka.\\
	these girls were several\\
	\glt `The girls were several in number.'
	\ex \gll Tych dziewczyn było {ileś (tam)}.\\
	these girls were some\\
	\glt `The girls were in some number.'
	\z
    \z

	\noindent At this point, it might be tempting to analyze \textit{kilka} and \textit{ileś} essentially on a par with \textit{pięć}. However, this is not the whole story. In the following sections, I will \isi{focus} on some non-trivial differences between cardinals and \isi{indefinite} numerals on the one hand and \textit{kilka} and \textit{ileś} on the other. By examining this distinction more closely, we can provide a proper semantic account of \ili{Polish} \isi{indefinite} numerals.

	\subsection{Reference to number concepts}\label{sec:reference-to-number-concepts}
	\largerpage[2]
	One could attempt to analyze \isi{indefinite} expressions such as \ili{English} \textit{several} in terms of \isi{existential quantification} over numbers of a certain size. However, it appears that there is a serious problem with the \isi{existential quantification} approach (see \citealt{schwarzschild2002grammar}). In particular, \isi{indefinite} numerals differ from \isi{cardinal} numerals in that they cannot be used to name number concepts and do not fit contexts calling for numerical arguments, see \REF{ex:numerical-arguments}. Furthermore, consider the mathematical statement in \REF{ex:numerical-contexts-math-cardinal}. A natural way to paraphrase it making use of the existential \isi{quantifier} is given in \REF{ex:numerical-contexts-math-cardinal-paraphrase}. Nonetheless, similar statements involving indefinites in \REF{ex:numerical-contexts-math-kilka} and \REF{ex:numerical-contexts-math-iles} are not felicitous despite the fact that their intended meaning can be easily paraphrased in terms of \isi{existential quantification}, as provided in \REF{ex:numerical-contexts-math-kilka-paraphrase} and \REF{ex:numerical-contexts-math-iles-paraphrase} respectively.

	\ea \label{ex:numerical-arguments} \ea \gll liczba \minsp{\{} pięć / *\hspace{-2pt} kilka / *\hspace{-2pt} {ileś\}}\label{ex:numerical-arguments-naming-numbers}\\
	number {} five {} {} several {} {} some\\
	\ex \gll Jaś umie policzyć do \minsp{\{} pięciu / *\hspace{-2pt} kilku / *\hspace{-2pt} {iluś\}}.\label{ex:numerical-arguments-context}\\
	Jaś can count.up to {} five.\textsc{gen} {} {} several.\textsc{gen} {} {} some.\textsc{gen}\\
	\glt `Jaś can count up to five.'
	\z
    \z

	\ea \label{ex:numerical-contexts-math} \ea \gll Cztery plus pięć to mniej niż dziesięć.\label{ex:numerical-contexts-math-cardinal}\\
	four plus five this less than ten\\
	\glt `Four plus five is less than ten.'
	%		\bg. W pokoju było dokładnie \textcolor{red}{pięć} kotów.\\
	%		in room was exactly five cats$_{\textsc{gen}}$\\
	%		`There were exactly five cats in the room.'
	\ex There is a number \textit{n} = 5 such that 4 + \textit{n} < 10.\label{ex:numerical-contexts-math-cardinal-paraphrase}
	\z
    \z

	\ea \ea[\#]{\gll Cztery plus kilka to mniej niż dziesięć.\label{ex:numerical-contexts-math-kilka}\\
	four plus several this less than ten\\
	\glt Intended: `Four plus several is less than ten.'}
	%		\bg. *W pokoju było dokładnie \textcolor{blue}{kilka} kotów.\\
	%		in room was exactly several cats$_{\textsc{gen}}$\\
	\ex[]{There is a number \textit{n} $\geq$ 3 $\wedge$ $\leq$ 9 such that 4 + \textit{n} < 10.}\label{ex:numerical-contexts-math-kilka-paraphrase}
    \z
    \z

	\ea \ea[\#]{\gll Cztery plus {ileś (tam)} to mniej niż dziesięć.\label{ex:numerical-contexts-math-iles}\\
	four plus some this less than ten\\
	\glt Intended: `Four plus some number is less than ten.'}
	%		\bg. *W pokoju było dokładnie \textcolor{blue}{kilka} kotów.\\
	%		in room was exactly several cats$_{\textsc{gen}}$\\
	\ex[]{There is a number \textit{n} such that 4 + \textit{n} < 10.}\label{ex:numerical-contexts-math-iles-paraphrase}
	\z
    \z

	\noindent The facts described above suggest that \ili{Polish} \isi{indefinite} numerals cannot be modeled in terms of \isi{existential quantification} over numbers. The following section will provide \isi{additional evidence} calling for an alternative treatment.

	\subsection{Specific reading}\label{sec:specific-reading}

	To my knowledge, it is a novel observation that \ili{Polish} \isi{indefinite} numerals can have a so-called \isi{specific reading}, i.e., an interpretation corresponding to the widest scope in the sentence (cf. \citealt{fodor_sag1982referential} and \citealt{kratzer1998scope}).\footnote{\cite{fodor_sag1982referential} call it a “referential interpretation". I will stick to the term \textsc{specific} though.} For instance, \REF{ex:indef-scope-kilka} can be interpreted with \textit{każdy} `each' scoping over \textit{kilka}: i.e., for each teacher there is some \isi{indefinite} number of which they know that that many students were called before the dean. Such an interpretation is sometimes referred to as a quantificational reading. However, \REF{ex:indef-scope-kilka} can also mean that in a given context there is a certain number of my students, say five, and each teacher knows that the number of my students that were called before the dean is that number. The same applies to \textit{ileś}, as illustrated in \REF{ex:indef-scope-iles}.

	\ea \gll Każdy nauczyciel wie, że kilku moich studentów wezwano do dziekana.\label{ex:indef-scope-kilka}\\
	each teacher knows that several my students were.called to dean\\
	\glt `Each teacher knows that several students of mine had been called before the dean.'
	\ea each > kilka \hfill quantificational reading\label{ex:indef-scope-kilka-quantificational}
	\ex kilka > each \hfill \isi{specific reading}\label{ex:indef-scope-kilka-specific}
	\z
    \z

    \ea \gll Każdy nauczyciel wie, że {iluś (tam)} moich studentów wezwano do dziekana.\label{ex:indef-scope-iles}\\
	each teacher knows that some my students were.called to dean\\
	\glt `Each teacher knows that some students of mine had been called before the dean.'
	\ea each > ileś \hfill quantificational reading\label{ex:indef-scope-iles-quantificational}
	\ex ileś > each \hfill \isi{specific reading}\label{ex:indef-scope-iles-specific}
	\z
    \z

	\noindent Alongside the ability to escape islands and insensitivity to various operators, the capability to take the widest scope is considered to be one of the diagnostics to detect specific indefinites such as \textit{a certain word} in \REF{ex:specific-indefinite}.

	\ea There is a certain word that I can never remember.\label{ex:specific-indefinite}
    \z

\noindent In a certain way the evidence seems to steer in the opposite directions. On the one hand, \isi{indefinite} numerals appear to be `referential' in the sense that they can indicate a specific, though \isi{indefinite}, number. On the other hand, however, they are infelicitous in contexts calling clearly for numerical arguments such as terms in mathematical equations.

	\subsection{Referential restrictions}\label{sec:referential-restrictions}

	Another contrast concerns referential restrictions that apply to \textit{kilka}. While \textit{ileś} can be used to denote any real (or perhaps even complex) number, \textit{kilka} seems to be restricted to a subset of integers, specifically the set \{3, 4, 5, 6, 7, 8, 9\}.\footnote{Some speakers may include 10 while others may restrict the set even further by excluding 3. I acknowledge that this issue might be subject to some degree to idiolectal variation but for the sake of simplicity I will ignore this fact in the following analysis.} Notice that a constraint regarding natural numbers seems to apply also to cardinals. For instance, in a scenario where there are four and a half apples on the table and it is conspicuous that the half does not count as a whole apple, it is rather odd to utter \REF{ex:cardinals-integers-apples}.\footnote{An example of such a scenario would be a cooking \isi{event} in which one bakes stuffed apples. In such a context half an apple is useless and simply does not count.} In such a context, it is also strange to use \REF{ex:kilka-integers-apples}. However, \REF{ex:iles-real-apples} seems perfectly felicitous. A similar contrast is given in \REF{ex:integers-pi}. Since $\pi$ is an irrational number, it can be associated with  the co-referential \textit{ileś} in the main clause, however using \textit{kilka} in such a sentence is impossible.

	\ea \ea[\#]{\gll Na stole leży {pięć} jabłek.\label{ex:cardinals-integers-apples}\\
	on table lies five apples\\
	\glt Intended: `There are five apples on the table.'}
    \ex[\#]{\gll Na stole leży {kilka} jabłek.\label{ex:kilka-integers-apples}\\
	on table lies several apples\\
	\glt Intended: `There are several apples on the table.'}
    \ex[]{\gll Na stole leży {ileś (tam)} jabłek.\label{ex:iles-real-apples}\\
	on table lies some apples\\
	\glt `There are some apples on the table.'}
	\z
    \z

	\ea \label{ex:integers-pi} \ea[\#]{\gll Pole koła to {kilka} razy r$^2$, a dokładnie $\pi$ razy r$^2$.\label{ex:kilka-integers-pi}\\
	area circle this several times r$^2$ and precisely $\pi$ times r$^2$\\
	\glt Intended: `The area of a circle is several times r$^2$, precisely $\pi$ times r$^2$.'}
    \ex[]{\gll Pole koła to {ileś (tam)} razy r$^2$, a dokładnie $\pi$ razy r$^2$.\label{ex:iles-integers-pi}\\
	area circle this some times r$^2$ and precisely $\pi$ times r$^2$\\
	\glt `The area of a circle is some number times r$^2$, precisely $\pi$ times r$^2$.'}
	\z
    \z

\noindent The data suggest yet another distinction between \isi{indefinite} numerals. Similar to cardinals, \textit{kilka} makes reference to natural numbers whereas \textit{ileś} is not restricted in such a way. Rather, it is apt to denote any number associated with a particular plurality or quantity.

	\subsection{Cardinal suffixes}\label{sec:cardinal-suffixes}

	The final data point to be discussed in this section concerns an interesting fact that unlike, e.g., \ili{English} \textit{several} (\citealt{kayne2007several}), the \ili{Polish} \isi{indefinite} \textit{kilka} can take \isi{cardinal} suffixes, as illustrated in \REF{ex:cardinal-suffixes-kilka}. On the other hand, \textit{ileś} is significantly degraded with \isi{cardinal} suffixes, see \REF{ex:cardinal-suffixes-iles}.\footnote{Although such forms are definitely not part of standard \ili{Polish} and many speakers judge them as ungrammatical, for some speakers they are marginally acceptable. However, the balanced NCP subcorpus which contains more than 240 million tokens returns no hits for the forms \textit{ileśnaście} and \textit{ileśdziesiąt} and six hits for \textit{ileśset}, two of which are from the prose of a linguistically very creative author. Therefore, I will assume that such forms are not well-formed expressions of \ili{Polish}.}

%    \Lsciex.\label{ex:cardinal-suffixes-ncp} \gll Za kolejną {wioską [\dots]} {ileśset} hektarów pszenżyta jeszcze stoi na polu.\\
%behind next village some.hundred hectares triticale still stands on field\\
%\glt `Behind the next village there are still several hundred hectares of triticale in the fields.'

	\ea \label{ex:cardinal-suffixes} \ea \gll {{pięć} $\sim$} {{pięt}naście $\sim$} {{pięć}dziesiąt $\sim$} {pięć}set\\
	five fifteen fifty five.hundred\\
	\ex \gll {{kilka} $\sim$} {{kilka}naście $\sim$} {{kilka}dziesiąt $\sim$} {kilka}set\label{ex:cardinal-suffixes-kilka} \\
	several several.teen several.ty several.hundred\\
	\ex \gll {{ileś} $\sim$} *\hspace{-2pt} {{ileś}naście $\sim$} *\hspace{-2pt} {{ileś}dziesiąt $\sim$} *\hspace{-2pt} {ileś}set\label{ex:cardinal-suffixes-iles} \\
	some {} some.teen {} some.ty {} some.hundred\\
	\z
    \z

	\noindent Interestingly, the interpretation of the suffixed \isi{indefinite} numerals seems to be derived from the meaning of \textit{kilka}. For instance, at least for some speakers \textit{kilkanaście} does not mean a number between 11 and 19 but rather it seems to exclude the values 11 and 12, hence \{13,\dots, 19\}. Similar, it would be awkward to refer to a plurality including approximately twenty members using \textit{kilkadziesiąt}; for a collection of around thirty entities it would be felicitous though. In spite of the fact that such intuitions may not be shared by all native speakers and I suspect some interpersonal variation here, my judgments as well as the judgments of the informants I have consulted are quite clear with respect to this issue and I will assume them to hold in general.

	\subsection{Data summary}\label{sec:data-summary2}

	Although \ili{Polish} \isi{indefinite} numerals pattern with cardinals such as \textit{pięć} `five' rather than with vague quantifiers such as \textit{mało} `few/little' and \textit{dużo} `much/many', there are a number of respects in which they differ. In particular, though both cardinals and \isi{indefinite} numerals can occur in \isi{predicate position} and can have a \isi{specific reading}, \textit{kilka} and \textit{ileś} cannot be used to name numbers, i.e., to refer to abstract concepts, and do not fit clearly numerical contexts. On the other hand, \textit{ileś} differs from cardinals and \textit{kilka} in that it cannot take \isi{cardinal} suffixes and is not restricted to natural numbers: i.e., unlike \textit{kilka} it can be used to talk about any real and possibly even complex number. \tabref{table:semantic-properties-of-cardinals-and-indefinite-numerals} summarizes the similarities and contrasts discussed in this section.

	\begin{table}[h]
		\centering
		\caption{Semantic properties of cardinals and indefinite numerals}
		\label{table:semantic-properties-of-cardinals-and-indefinite-numerals}
		\begin{tabularx}{0.8\textwidth}{@{}lXXX@{}}
			\lsptoprule
			\multirow{2}{*}{}      & \textit{pięć}         & \textit{kilka}        & \textit{ileś (tam)}   \\
			& \small{`five'}       & \small{`several'}    & \small{`some'}       \\ \midrule
			\isi{predicate position}     & $\checkmark$ & $\checkmark$ & $\checkmark$ \\
			\isi{specific reading}       & $\checkmark$ & $\checkmark$ & $\checkmark$ \\
			\isi{cardinal} suffixes      & $\checkmark$ & $\checkmark$ & *            \\
			restricted to integers & $\checkmark$ & $\checkmark$ & *            \\
			names of numbers       & $\checkmark$ & *            & *            \\
			numeric contexts       & $\checkmark$ & *            & *            \\ \lspbottomrule
		\end{tabularx}
	\end{table}

	\newpage
	I conclude that a neat classification developed here calls for a more elaborate analysis of numerical expressions than usually assumed. In particular, a proper treatment of numerical expressions should account for the semantic differences between the class of cardinals and two types of \isi{indefinite} quantifiers, namely \textit{kilka} and \textit{ileś}.

	Before we move on to spelling out the semantics for \isi{indefinite} numerals that will capture the discussed patterns and contrasts, it will be useful to introduce several theoretical tools. In the next section I will sketch a framework within which the proposed analysis will be grounded.

	\section{Setting the stage}\label{sec:setting-the-stage}

	\subsection{Choice functions}\label{sec:choice-functions}


	Following \cite{reinhart1997quantifier} and \citet{kratzer1998scope} as well as subsequent cross-lin\-guis\-tic research on specific indefinites (see \citealt{alonso-ovalle_menendez-benito2003some} for \ili{Spanish} \textit{algún}, \citealt{kratzer_shimoyama2002indeterminate} for \ili{German} \textit{irgendein}, \citealt{yanovich2005choice} for \ili{Russian} \isi{indefinite} series, and \citealt{matthewson1998interpretation} for indefinites in St’át’im\-cets), I model \textit{ileś} and \textit{kilka} as choice functions (\isi{CF}): i.e., operators selecting a member from a set. On the adopted view, \isi{CF} indefinites are not existentially quantified. Instead, the \isi{CF} variable remains free at LF and its value is provided by the context. In particular, I embrace the approach that CFs provide a null \isi{pronominal} element of type $\langle \langle \tau,t\rangle,\tau \rangle$, where $\tau$ is a generalized primitive type, see \REF{ex:cf-generalized} and \REF{ex:cf-individuals} for entities.

%[1]

%	\ea \ea $f_{\langle \langle \tau,t\rangle,\tau\rangle}$ is a \isi{CF} if $P_{\langle \tau,t\rangle}(f(P_{\langle \tau,t\rangle})) = 1$\label{ex:cf-generalized}
%	\ex $f_{\langle \langle e,t\rangle,e\rangle}$ is a \isi{CF} if $P_{\langle e,t\rangle}(f(P_{\langle e,t\rangle})) = 1$\label{ex:cf-individuals}
%	\z
%    \z

%MW: Do you mean like below? Really?
%[2]
%	\ea \ea $f_{\langle \langle \tau,t\rangle,\tau\rangle} \text{ is a \isi{CF} if } P_{\langle \tau,t\rangle}(f(P_{\langle \tau,t\rangle})) = 1$\label{ex:cf-generalized}
% 	\ex $f_{\langle \langle e,t\rangle,e\rangle} \text{ is a \isi{CF} if } P_{\langle e,t\rangle}(f(P_{\langle e,t\rangle})) = 1$\label{ex:cf-individuals}
% 	\z
%     \z

%Radek: I don't quite see a difference between the above [2] and [1]. What I meant is [3]. The problem with [1] (and [2]) is that in order for the definition to work, P, and in fact also f, would have to be constants. But that's not what you intend - you want to define it completely generally (for any f of the given type and any P, not just for a single constant \isi{choice function} and a single constant P.

 \ea \ea For any $f_{\langle \langle \tau,t\rangle,\tau\rangle}$ and any $P_{\langle \tau,t\rangle}$, $f$ is a \isi{CF} if $P(f(P))=1$\label{ex:cf-generalized}
 \ex For any $f_{\langle \langle e,t\rangle,e\rangle}$ and any $P_{\langle e,t\rangle}$, $f$ is a \isi{CF} if $P(f(P))=1$\label{ex:cf-individuals}
 \z\z

\noindent If a \isi{CF} $f$ is applied to a set of, e.g., sleeping individuals, it will yield a specific sleeper \isi{relative} to a particular context. Similar, when applied to a set of natural numbers, it will return a relevant integer. In this way, one can account for the referential flavor of specific indefinites without employing \isi{existential quantification}.

	\subsection{Measure functions}\label{sec:measure-functions}

    \largerpage[2]
	Following \cite{krifka1989nominal}, I model quantification in \isi{numeral} and measure constructions in terms of extensive measure functions (\isi{MF}), i.e., operations that map a plurality of individuals or quantity of substance onto a real number corresponding to the number of individuals or units making up the plurality or quantity. Such MFs are additive, see \REF{ex:mf-additive} and have the Archimedean property, see \REF{ex:mf-archimedean}. In addition, assuming the remainder principle for $\sqcup$ guarantees monotonicity, see \REF{ex:mf-monotonic} (cf. \citealt{schwarzschild2002grammar}).

	\ea \ea $\mu$ is an additive \isi{MF} with respect to $\sqcup$ iff for any $x_e$ and any $y_e$,\\$\neg x \circ y \rightarrow [\mu(x\sqcup y) = \mu(x) + \mu(y)]$\label{ex:mf-additive}
	\ex $\mu$ is an Archimedean \isi{MF} iff for any $x_e$ and any $y_e$,\\$[\mu(x) > 0 \wedge y \sqsubseteq x] \rightarrow \mu(y) > 0$\label{ex:mf-archimedean}
	\ex $\mu$ is a monotonic \isi{MF} with respect to $\sqsubseteq$ iff for any $x_e$ and any $y_e$,\\$x \sqsubset y \rightarrow \mu(x) < \mu(y)$\label{ex:mf-monotonic}
	\z
    \z

%MW: Do you mean like below? Really?

%	\ea \ea $\mu \text{ is an additive \isi{MF} with respect to } \sqcup \text{ iff }$\\$\neg x \circ y \rightarrow [\mu(x\sqcup y) = \mu(x) + \mu(y)]$\label{ex:mf-additive}
%	\ex $\mu \text{ is an Archimedean \isi{MF} iff }$\\$[\mu(x) > 0 \wedge y \sqsubseteq x] \rightarrow \mu(y) > 0$\label{ex:mf-archimedean}
%	\ex $\mu \text{ is a monotonic \isi{MF} with respect to } \sqsubseteq \text{ iff }$\\$x \sqsubset y \rightarrow \mu(x) < \mu(y)$\label{ex:mf-monotonic}
%	\z
%    \z

	\noindent Counting is therefore modeled as a form of measuring. For instance, the \isi{MF} \textsc{liter} returns the integer 3 if there are three liters of an entity in question, see \REF{ex:mf-liter}. Similar, the \isi{MF} \# can be introduced which would yield 3 if a number of individual members of a plurality it is applied to equals 3, see \REF{ex:mf-cardinality}.\footnote{Here I depart from \citeposst{krifka1989nominal} original proposal. In his system, the \textsc{nu} operation (for `natural unit') is postulated which when applied to a property returns a number of natural units associated with that property.} Let us assume that \# is defined in such a way that it takes only a plurality of atomic individuals, i.e., entities that do not have proper parts, and returns a number of atoms making up that plurality. Such a restriction guarantees its incompatibility with mass nouns unless their denotation is shifted to the count domain, e.g., via the universal packager or the universal sorter.

	\ea \ea $\llbracket\text{three liters of juice}\rrbracket=\lambda x[\textsc{juice}(x) \wedge \textsc{liter}(x)=3]$\label{ex:mf-liter}
	\ex $\llbracket\text{three apples}\rrbracket=\lambda x[\textsc{apple}(x) \wedge \#(x)=3]$\label{ex:mf-cardinality}
	\z
    \z

	\noindent Furthermore, to account for the compatibility of \textit{ileś} with both countable and uncountable NPs I will follow \cite{bale_barner2009interpretation} in assuming a generalized context-dependent \isi{MF} $\mu$. Such an approach posits a mechanism of contextual conditioning along the lines defined in \REF{ex:contextual-conditioning-mf}.

    \ea $\mu$ is interpreted as one of the MFs $m_z$ in the series $\langle m_1, m_2, m_3\dots m_n\rangle$ such that the argument for $\mu$ is in the range of $m_z$; furthermore, contextually $m_z$ is preferred to $m_y$ if $z<y$\label{ex:contextual-conditioning-mf}
    \z

%MW: Do you mean like below? Really?

%    \ea $\mu \text{ is interpreted as one of the MFs } m_z \text{ in the series } \langle m_1, m_2, m_3\dots m_n\rangle \text{ such that the argument for } \mu \text{ is in the range of } m_z\text{ ; furthermore, contextually } m_z \text{ is preferred to } m_y \text{ if } z<y$\label{ex:contextual-conditioning-mf}
%    \z

    \noindent A contextually conditioned \isi{MF} can cover the meanings of both pure measure constructions such as \REF{ex:mf-liter} and counting expressions like those in \REF{ex:mf-cardinality}. In particular, $\mu$ is interpreted as an \isi{MF} counting units of, e.g., volume, when combined with a mass term denoting a substance and as an \isi{MF} counting atomic entities when combined with expressions denoting individuated semi-lattices such as count nouns and pluralia tantum.

	\subsection{Cardinals}\label{sec:cardinals}

	\cite{rothstein2013fregean,rothstein2017semantics} distinguishes between several functions of numerals. In a non-\isi{classifier} language such as \ili{English} cardinals can be used as (i)~nominal modifiers, (ii)~predicates, and (iii)~names of concept numbers. When used in attributive and \isi{predicate position} numerals are \isi{cardinal} predicates of the same type as adjectives (\citealt{landman2003predicate}), see \REF{ex:cardinals-predicates}, whereas when used as names of numbers, they refer to abstract objects of a primitive semantic type \textit{n}, see \REF{ex:cardinals-names}. On this view, \isi{cardinal} predicates denote sets of plural entities with a specific cardinality, i.e., $\{ x: \#(x) = n\}$, and have standard intersective semantics.\footnote{Both Landman and Rothstein use the symbol $|\dots|$ instead of $\#$. I have replaced it for the sake of notational uniformity and clarity.} For instance, \textit{three apples} denotes a set of pluralities that are both in the denotation of \textit{apples} and have the property \textit{three}, i.e., a set of triples of apples. Rothstein assumes that \isi{cardinal} properties are basic, whereas their individual correlates, i.e., names of number concepts, are derived and building on Fregean property theory (\citealt{chierchia1985formal}) postulates shifting operations $^\cup$ and $^\cap$ which allow for switching freely between the two.

	\ea \ea $\llbracket\text{three}_{\langle e,t\rangle}\rrbracket=\lambda x[\#(x) = 3]$\label{ex:cardinals-predicates}
	\ex $\llbracket\text{three}_n\rrbracket=3$\label{ex:cardinals-names}
	\z
    \z

\noindent In the system described above, complex numerals such as \textit{twenty-three} are derived by means of a null $+$ operator which works as illustrated in \REF{ex:plus}.

	\ea\label{ex:plus} \ea $\llbracket+\rrbracket=\lambda m\lambda n[m+n]$
	\ex $\llbracket\text{twenty-three}\rrbracket=\lambda m\lambda n[m+n](20)(3)=\lambda n[20+n](3)=20+3$
	\z
    \z

	\noindent With all the ingredients in place, let us now see what they can account for and how they interact. In the following section, I will provide an analysis of the \ili{Polish} \isi{indefinite} numerals \textit{kilka} and \textit{ileś} which captures their similarities with cardinals as well as accounts for the discussed differences.

	\section{Putting the pieces together}\label{sec:putting-pieces-together}

	\subsection{Adaptations and extensions}\label{sec:adaptations-and-extensions}


	Within the patch-work framework adopted here there are several adjustments and developments I will make. First of all, unlike Rothstein, I assume that the use of cardinals as names of numbers is the basic one. In particular, I posit that numerals are complex expressions involving the \isi{numeral} root which is an expression of type $n$, the Numeral head which introduces gender, and optionally the \isi{classifier} element \textsc{card} (for `\isi{cardinal} property') which takes a number and returns a set of atomic individuals whose cardinality equals that number, see \REF{ex:card}. Proper counting is guaranteed by the $\#$ \isi{MF} and \isi{presupposition} of atomicity incorporated into the semantics of \textsc{card}. In other words, cardinals are born as names of numbers (cf. \citealt{scha1981distributive}) and by adding additional structure can be converted to \isi{cardinal} properties at type $\langle e,t\rangle$. I assume that in a language such as \ili{English} or \ili{Polish} \textsc{card} has no overt exponent. However, in \isi{classifier} languages it is introduced by the \isi{classifier} (see \citealt{sudo2016semantic} for a similar proposal).

	\ea $\llbracket\textsc{card}\rrbracket=\lambda n\lambda x\,.\,\cnst{atom}(x)\,[\#(x)=n]$\label{ex:card}
	\z

\noindent Furthermore, I posit yet another \isi{classifier} element, namely \textsc{quant} (for `\isi{quantificational property}') which also shifts number concepts to sets of entities but unlike \textsc{card} it employs the contextually conditioned \isi{MF} $\mu$ which can either measure, e.g., volume or count individuals depending on a context. Such conditioning makes \textsc{quant} compatible with both countable and uncountable NPs.

	\ea $\llbracket\textsc{quant}\rrbracket=\lambda n\lambda x[\mu(x)=n]$\label{ex:quant}
	\z

\noindent Finally, I propose that in \ili{Polish} suffixed numerals there is no covert $+$ operation but rather \isi{cardinal} suffixes are number operators of type $\langle n,n\rangle$ themselves. They take the denotation of the \isi{numeral} root and yield a number enlarged via addition or multiplication, see \REF{ex:cardinal-suffixes-semantics}, which can be then shifted by \textsc{card}. Notice, however, that the \isi{cardinal} suffixes incorporate a special \isi{presupposition} that makes them compatible only with natural numbers. Such a move will explain the behavior of \textit{ileś}, but it is also independently motivated by the fact that \isi{cardinal} suffixes are not compatible with expressions denoting fractions, as shown by the contrast in \REF{ex:cardinal-suffixes-integers}.

	\ea \label{ex:cardinal-suffixes-semantics} \ea $\llbracket \text{-naście}\rrbracket = \lambda n\,.\,\cnst{integer}(n)\,[n+10]$\label{ex:cardinal-suffixes-teen}
\ex $\llbracket \text{-dziesiąt}\rrbracket = \lambda n\,.\,\cnst{integer}(n)\,[n\times 10]$\label{ex:cardinal-suffixes-ty}
\ex $\llbracket \text{-set}\rrbracket = \lambda n\,.\,\cnst{integer}(n)\,[n\times 100]$\label{ex:cardinal-suffixes-hundred}
	\z
    \z

\begin{multicols}{2}
	\ea \label{ex:cardinal-suffixes-integers} \ea[]{\gll dziesięć i pół\\
    ten and half\\}\columnbreak
    \ex[*]{\gll półnaście\\
    half.teen\\}
	\z
    \z
\end{multicols}

\noindent Let us now examine how the proposed semantics accounts for \ili{Polish} cardinals and \isi{indefinite} numerals.

	\subsection{Composition of cardinals}\label{sec:composition-of-cardinals}

I argue that \ili{Polish} \isi{cardinal} numerals are complex expressions. First, let us consider cardinals in numerical contexts such as \REF{ex:numerical-contexts-math-cardinal-paraphrase} where they are used as names of abstract mathematical concepts. In general, I take \isi{numeral} roots to be category-free, as often claimed (e.g., \citealt{halle_marantz1993distributed}). Due to the fact that \ili{Polish} cardinals can be used not only as modifiers and predicates, but also as names of numbers and can be modified by agreeing adjectives, I assume that in a sense they have some nominal-like properties. Therefore, I posit that a gender value is always associated with the Numeral head which forges the \isi{cardinal}. Let us consider the derivation of the non-virile \isi{numeral} \textit{pięć} `five', see \figref{ex:cardinal-name-tree}. The category-free root $\sqrt{\textit{pięć-}}$ is a name of the natural number 5, i.e., an expression of a primitive type $n$. Though the Numeral head has a crucial structural role, i.e., it assigns the [NV] (for `non-virile') gender value and forms the \isi{numeral}, it lacks any particular semantic contribution, and the resulting expression  is therefore simply the name of number 5.\footnote{In the case of the form \textit{pięciu}, the Numeral head assigns the [V] (for ``virile'') value.}

\begin{figure}[h!]
    \centering
    \begin{forest}
    [{NumeralP$_n$ \\ \scriptsize$5$}, align=center, base=top, for tree={parent anchor=south, child anchor=north}
    [{Numeral \\ \scriptsize\textsc{[nv]} \\ -$\varnothing$}, align=center, base=top]
    [{$\sqrt{\textit{pięć-}}_n$ \\ \scriptsize$5$}, align=center, base=top]
    ]
    \end{forest}
    \caption{Derivation of the number name \textit{pięć} `five'}
    \label{ex:cardinal-name-tree}
\end{figure}

However, the structure in \figref{ex:cardinal-name-tree} can be further augmented with the silent node which introduces the \textsc{card} operation, see \figref{ex:cardinal-predicate-tree}. As a result, the number 5 is shifted to the set of atomic individuals whose cardinality equals 5. Such an expression can be used both as a nominal modifier and in \isi{predicate position}.

\begin{figure}
    \centering
    \begin{forest}
   [{NumeralP$_{\langle e,t\rangle}$\\\scriptsize$\lambda x.\cnst{atom}(x)[\#(x)=5]$}, align=center, base=top, for tree={parent anchor=south, child anchor=north}
   [{\textsc{card}$_{\langle n,\langle e,t\rangle\rangle}$\\\scriptsize$\lambda n\lambda x.\cnst{atom}(x)[\#(x)=n]$}, align=center, base=top]
    [{NumeralP$_n$\\\scriptsize$5$}, align=center, base=top
    [{Numeral\\\scriptsize\textsc{[nv]}\\-$\varnothing$}, align=center, base=top]
    [{$\sqrt{\textit{pięć-}}_n$\\\scriptsize$5$}, align=center, base=top]
    ]
    ]
    \end{forest}
    \caption{Derivation of the cardinal predicate \textit{pięć} `five'}
    \label{ex:cardinal-predicate-tree}
\end{figure}
\begin{figure}
    \centering
    \begin{forest}
    [{NumeralP$_{\langle e,t\rangle}$\\\scriptsize$\lambda x.\cnst{atom}(x)[\#(x)=15]$}, align=center, base=top, for tree={parent anchor=south, child anchor=north}
    [{\textsc{card}$_{\langle n,\langle e,t\rangle\rangle}$\\\scriptsize$\lambda n\lambda x.\cnst{atom}(x)[\#(x)=n]$}, align=center, base=top]
    [{NumeralP$_n$\\\scriptsize$15$}, align=center, base=top
    [{Numeral\\\scriptsize\textsc{[nv]}\\\textit{-e}}, align=center, base=top]
    [{\scriptsize$n$\\\scriptsize$15$}, align=center, base=top
    [{\textit{-nast-}$_{\langle n,n\rangle}$\\\scriptsize$\lambda n.\cnst{integer}(n)[n+10]$}, align=center, base=top]
    [{$\sqrt{\textit{pięć-}}_n$\\\scriptsize$5$}, align=center, base=top ]
    ]
    ]
    ]
    \end{forest}
    \caption{Derivation of the cardinal predicate \textit{piętnaście} `fifteen'}
    \label{ex:cardinal-predicate-teen-tree}
\end{figure}

\largerpage[2]
Finally, a derivationally complex \isi{numeral} such as \textit{piętnaście} `fifteen' can be obtained by incorporating the node associated with the \isi{cardinal} suffix in the structure. Specifically, I posit that it is not until the \isi{cardinal} suffix attaches to the root and yields an enlarged number that the Numeral head applies and forms the NumeralP which can serve as an argument for \textsc{card}. The tree in \figref{ex:cardinal-predicate-teen-tree} gives the structure for the non-virile \isi{cardinal} \textit{piętnaście}; the derivation of other suffixed cardinals is analogous.\footnote{Notice that \textit{pięć-} and \textit{pięt-} are allomorphs, similar to the suffixes \textit{-naści-} and \textit{-nast-}, as in the virile form \textit{piętnastu}. I take \textit{-nast-} to be the basic form and assume that it alternates with \textit{-naści-} in contexts preceding \textit{-e}.}
\clearpage



With the proposed mechanism of deriving \ili{Polish} cardinals in place, let us now move to the semantics of \isi{indefinite} numerals. The next section is dedicated to explaining the composition of \textit{kilka} and \textit{ileś}.

	\subsection{Composition of indefinite numerals}\label{sec:composition-of-indefinite-numerals}

\subsubsection{Deriving \textit{kilka}}\label{sec:deriving-kilka}

\noindent I will start with the structure for \textit{kilka} `several', see \figref{ex:kilka-tree}. I presume that the root $\sqrt{\textit{kilk-}}$ involves a built-in \isi{CF} that applies to the restricted set of alternatives, namely the set of natural numbers \{3, 4, 5, 6, 7, 8, 9\}, and yields a specific value in a given context. The root then combines with the Numeral head which assigns the [NV] gender value. However, unlike in the case of cardinals, the Numeral head does have a semantic contribution. In particular, it introduces the \textsc{card} operation which shifts the \isi{indefinite} number to the \isi{cardinal} property. The resulting expression is of type $\langle e,t\rangle$, and thus it is illicit in contexts calling for numeric arguments, as already illustrated in \REF{ex:numerical-contexts-math-kilka}. Furthermore, the fact that the \isi{MF} \# requires atomic denotations explains why \textit{kilka} is incompatible with mass terms.

\begin{figure}[h!]
	\centering
	\begin{forest}
	[{NumeralP$_{\langle e,t\rangle}$\\\scriptsize$\lambda x.\cnst{atom}(x)[\#(x)=f_{\langle\langle n, t \rangle, n \rangle}(\lambda n[\cnst{integer}(n) \wedge n \geq 3 \wedge n \leq 9])]$}, align=center, base=top, for tree={parent anchor=south, child anchor=north}
	[{Numeral\\\scriptsize\textsc{[nv]}\\\textsc{card}$_{\langle n,\langle e,t\rangle\rangle}$\\\scriptsize$\lambda n\lambda x.\cnst{atom}(x)[\#(x)=n]$\\\textit{-a}}, align=center, base=top]
	[{$\sqrt{\textit{kilk-}}_n$\\\scriptsize$f_{\langle\langle n, t \rangle, n \rangle}(\lambda n[\cnst{integer}(n) \wedge n \geq 3 \wedge n \leq 9])$}, align=center, base=top ]
	]
	\end{forest}
	\caption{Derivation of \textit{kilka} `several, a few'}
	\label{ex:kilka-tree}
\end{figure}

The proposed semantics also accounts for the fact that \textit{kilka} can combine with \isi{cardinal} suffixes. Since the number selected by the \isi{CF} $f$ is a natural number, it can serve as an argument for the \isi{cardinal} suffixes, as defined in \REF{ex:cardinal-suffixes-semantics}.

\subsubsection{Deriving \textit{ileś}}\label{sec:deriving-iles}

	As discussed in \sectref{sec:polish-indefinite-series}, the \isi{indefinite} \textit{ileś} `some number' is a complex expression involving a wh-word and the \isi{indefinite} suffix \textit{-ś}. In general,	I assume that wh-elements denote properties. In this case, the wh-root $\sqrt{\textit{il-}}$ denotes a property of type $\langle n,t\rangle$, namely a property of being a real number.\footnote{Arguably, it might be even a complex number. However, since I remain agnostic with respect to the question whether the concept of complex numbers is part of the semantics of natural language, I will stick to reals.} Furthermore, I adopt the view that \isi{indefinite} suffixes in \ili{Slavic} introduce a generalized \isi{CF} of type $\langle\langle \tau,t\rangle,\tau\rangle$, see \REF{ex:indefinite-suffix-gcf}, which can attach to any wh-element to yield an \isi{indefinite} expression (\citealt{yanovich2005choice}).

	\ea \label{ex:indefinite-suffix-gcf} $\lambda P_{\langle \tau, t \rangle}[f_{\langle\langle \tau,t\rangle,\tau\rangle}(P)]$%, where $f$ is a GCF (Generalized Choice Function)
	\z

\noindent I propose that the composition of \textit{ileś} proceeds as in \figref{ex:iles-tree}. The \isi{indefinite} suffix \textit{-ś} combines directly with the wh-root $\sqrt{\textit{il-}}$ so that the \isi{CF} $f$ yields a specific real number \isi{relative} to a particular context.\footnote{The surface order of morphemes in \figref{ex:iles-tree} is derived by (phrasal) movement of the root $\sqrt{\textit{il-}}$ to the left of the two functional heads \textit{-e-} and \textit{-ś}, which remain in the base order. As pointed out by an anonymous
reviewer, this is not a frequent phenomenon, and it goes against
traditional accounts of morpheme order based on head movement (e.g.,
\citealt{baker1988incorporation}), which would lead to a mirror-image order such as *\textit{il-ś-e}. However, the type of movement needed for \figref{ex:iles-tree} has been argued independently to be necessary for various morpheme orders within words as well (e.g.,
\citealt{koopman_szabolcsi2000verbal}, \citealt{julien2002syntactic}; see also \citealt{caha2017suppletion} for discussion).} Similar to \textit{kilka}, the Numeral head not only assigns the gender value, but also introduces the \isi{classifier} element. However, in this case it is not \textsc{card} but \textsc{quant}.

\begin{figure}[h!]
    \centering
    \begin{forest}
    [{NumeralP$_{\langle e,t\rangle}$\\\scriptsize$\lambda x[\mu(x)=f_{\langle\langle n, t \rangle, n \rangle}(\lambda n[\cnst{real}(n)])]$}, align=center, base=top, for tree={parent anchor=south, child anchor=north}
    [{Numeral\\\scriptsize\textsc{[nv]}\\\textsc{quant}$_{\langle n,\langle e,t\rangle\rangle}$\\\scriptsize$\lambda n\lambda x[\mu(x)=n]$\\\textit{-e-}}, align=center, base=top]
    [{\scriptsize$n$\\\scriptsize$f_{\langle\langle n, t \rangle, n \rangle}\lambda n[\cnst{real}(n)]$}, align=center, base=top
    [{indef.suffix$_{\langle\langle \tau,t\rangle,\tau\rangle}$\\\scriptsize$\lambda P[f(P)]$\\\textit{-ś}}, align=center, base=top]
    [{$\sqrt{\textit{il-}}_{\langle n,t\rangle}$\\\scriptsize$\lambda n[\cnst{real}(n)]$}, align=center, base=top ]
    ]
    ]
    \end{forest}
    \caption{Derivation of \textit{ileś} `some, some number'}
    \label{ex:iles-tree}
\end{figure}


    The type of the NumeralP is again $\langle e,t\rangle$ which does not allow \textit{ileś} to refer to number concepts in clearly numeric environments. On the other hand, the contextually conditioned \isi{MF} $\mu$ accounts for the fact that \textit{ileś} is compatible both with count and mass terms. In the first case, it simply returns the number of atomic individuals making up a plurality whereas in the latter it yields the amount of substance. Finally, the fact that the \isi{indefinite} number is not necessarily an integer makes \textit{ileś} incompatible with \isi{cardinal} suffixes.

    The last issue concerns how to ensure that the Numeral head gets the correct semantics in combination with a particular root, i.e., $\sqrt{\textit{kilk-}}$, $\sqrt{\textit{il-}}$, and \isi{cardinal} roots such as $\sqrt{\textit{pięć-}}$. For this purpose, I postulate the interface instructions as provided in \REF{ex:numeral-lf-interpretation}.

	\ea  \textit{Interpretation of the \ili{Polish} Numeral head at LF}\smallskip\label{ex:numeral-lf-interpretation}\\
    \text{Numeral} \tabto{1.5cm}${}\Leftrightarrow \textsc{card}\ /\ [\ \underline{\hspace{1em}}\ [\sqrt{\textit{kilk-}}]\ ]$\\
    \tabto{1.5cm}${}\Leftrightarrow \textsc{quant}\ /\ [\ \underline{\hspace{1em}}\ [\sqrt{\textit{il-}}]\ ]$\vspace{1pt}\\
    \tabto{1.5cm}${}\Leftrightarrow \varnothing\ /\ \text{elsewhere}$
	\z

	\noindent Given the standard elsewhere principle, the application of a specific operation overrides the application of a more general rule, and thus what happens at LF is as follows. The Numeral head is interpreted as \textsc{card} only in case it dominates the root $\sqrt{\textit{kilk-}}$ and as \textsc{quant} if and only if its complement is the root $\sqrt{\textit{il-}}$. In all other cases, i.e., when Numeral combines with the \isi{cardinal} root, it is semantically vacuous. The proposed mechanism guarantees adequate interpretations of the structures postulated for number-denoting cardinals and \isi{indefinite} numerals in \figref{ex:cardinal-name-tree}, \figref{ex:kilka-tree}, and \figref{ex:iles-tree}. Insertion of an additional null \textsc{card} node higher in the tree, see \figref{ex:cardinal-predicate-tree}, gives rise to a \isi{cardinal} predicate which can be used as a nominal modifier and in \isi{predicate position}.

	\section{Conclusion}\label{sec:conclusion}

	In this paper, I presented novel data concerning the distribution as well as semantic properties of the \ili{Polish} \isi{indefinite} quantifiers \textit{kilka} `several, a few' and \textit{ileś (tam)} `some, some number'. Based on a number of tests, I concluded that such indefinites pattern with \isi{cardinal} numerals rather than with vague quantifiers such as \textit{mało} `little, few'. Moreover, I posited that \textit{kilka} and \textit{ileś} should be treated as specific indefinites since they can have a `referential' reading in an embedded clause, i.e., they can scope over a \isi{quantifier} in a matrix clause. Therefore, I proposed that \ili{Polish} \isi{indefinite} numerals essentially share the core choice-functional semantics and argued that they should be analyzed as having a built-in \isi{classifier} involving a \isi{measure function}. The difference between the two results from the fact that \textit{kilka} employs a cardinality function which is compatible only with atomic denotations and yields a value from the set of natural numbers $\{3,\dots,9\}$, whereas \textit{ileś} introduces a contextually conditioned \isi{measure function} which, depending on a context, returns a real number corresponding either to a cardinality of a plurality or to a measure calibrated in relevant units.

	Further research should \isi{focus} on cross-linguistic investigations related to \isi{indefinite} numerals both within \ili{Slavic} and beyond as well as the behavior of FCIs such as \textit{ilekolwiek} `any number' in \ili{Polish}. An open issue concerns the exact nature of the mapping between semantics and morphology in the case of the discussed indefinites from a typological point of view.

	\section*{Abbreviations}

	\begin{tabularx}{.48\textwidth}{lQ}
		BCS&Bosnian/Croatian/Serbian\\
		NCP&National Corpus of Polish\\
		LF&Logical Form\\
		{FCI}&{free choice item}\\
		{CF}&{choice function}\\
		{MF}&{measure function}\\
		\textsc{card}&{cardinal} property\\
		\textsc{quant}&{quantificational property}\\
	\end{tabularx}
	\begin{tabularx}{.48\textwidth}{lQ}
		\textsc{v}&virile\\
		\textsc{nv}&non-virile\\
		\textsc{cmpr}&{comparative}\\
		\textsc{sprl}&{superlative}\\
		\textsc{gen}&genitive\\
		\textsc{distr}&{distributivity marker}\\
		\textsc{cl}&{classifier}\\
		\textsc{pl}&plural\\
	\end{tabularx}

	\section*{Acknowledgements}

	I would like to sincerely thank the audience at the FDSL 12.5 conference as well as two anonymous reviewers for their helpful comments and questions. I am especially grateful to Boban Arsenijević, Pavel Caha, Nina Haslinger, Roumyana Pancheva, Yasu Sudo, Viola Schmitt, and Jacek Witkoś for inspiring discussions of the data and/or the account developed in this paper. All errors are, of course, my own responsibility. I gratefully acknowledge that the research was supported by a Czech Science Foundation (GAČR) grant to the Department of Linguistics and Baltic Languages at the Masaryk University in Brno (GA17-16111S).

\sloppy
\printbibliography[heading=subbibliography,notkeyword=this]
\il{Polish|)}
\end{document}
