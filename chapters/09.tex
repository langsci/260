\documentclass[output=paper,
% colorlinks,citecolor=brown,newtxmath
]{langscibook} 
% \bibliography{localbibliography}

% \input{localpackages.tex}
% \newcommand{\smiley}{ :) }

% non-italics in examples

\renewcommand{\eachwordone}{\upshape}

% non-italics in examples in footnotes

\renewcommand{\fnexfont}{\footnotesize\upshape}
\renewcommand{\fnglossfont}{\footnotesize\upshape}
\renewcommand{\fntransfont}{\footnotesize\upshape}
\renewcommand{\fnexnrfont}{\fnexfont\upshape}

% chapter03 goncharov

\newcommand{\p}{\textsc{pfv\ }}
\newcommand{\im}{\textsc{ipfv\ }}

\makeatletter
\let\thetitle\@title
\let\theauthor\@author 
\makeatother

\newcommand{\togglepaper}[1][0]{  
  \addbibresource{../localbibliography.bib}  
  \papernote{\scriptsize\normalfont
    \theauthor.
    \thetitle. 
    To appear in: 
    Change Volume Editor \& in localcommands.tex 
    Change volume title in localcommands.tex
    Berlin: Language Science Press. [preliminary page numbering]
  }
  \pagenumbering{roman}
  \setcounter{chapter}{#1}
  \addtocounter{chapter}{-1}
}

\providecommand{\orcid}[1]{}

\author{Ilya Naumov\affiliation{National Research University Higher School of Economics}}

\title{Constraining the distribution of the perdurative in Russian}

\abstract{In this paper, I explore the constraints on the distribution of the perdurative prefix \textit{pro-} in Russian. Applying several diagnostics proposed by \cite{tatevosov2009mnozestvennaja, tatevosov2013mnozestvennaja}, I show that the perdurative \textit{pro-} is a ``selectionally restricted" prefix associated with an additional restriction: it can combine with a predicate built on a secondary imperfective stem only under a pluractional interpretation. I argue that this restriction is an instantiation of a more general semantic requirement imposed by the perdurative: it can be formed from a predicate if there are no subevents making up the activity component of this predicate that are in \citeposst{landman1992progressive} ``stage-of relation''.

\keywords{aspect, (a)telicity, subeventual structure, verbal prefixes, perdurative, Russian}}



% \IfFileExists{../localcommands.tex}{%hack to check whether this is being compiled as part of a collection or standalone
%   \input{../localpackages}
%   \newcommand{\smiley}{ :) }

% non-italics in examples

\renewcommand{\eachwordone}{\upshape}

% non-italics in examples in footnotes

\renewcommand{\fnexfont}{\footnotesize\upshape}
\renewcommand{\fnglossfont}{\footnotesize\upshape}
\renewcommand{\fntransfont}{\footnotesize\upshape}
\renewcommand{\fnexnrfont}{\fnexfont\upshape}

% chapter03 goncharov

\newcommand{\p}{\textsc{pfv\ }}
\newcommand{\im}{\textsc{ipfv\ }}

\makeatletter
\let\thetitle\@title
\let\theauthor\@author 
\makeatother

\newcommand{\togglepaper}[1][0]{  
  \addbibresource{../localbibliography.bib}  
  \papernote{\scriptsize\normalfont
    \theauthor.
    \thetitle. 
    To appear in: 
    Change Volume Editor \& in localcommands.tex 
    Change volume title in localcommands.tex
    Berlin: Language Science Press. [preliminary page numbering]
  }
  \pagenumbering{roman}
  \setcounter{chapter}{#1}
  \addtocounter{chapter}{-1}
}

\providecommand{\orcid}[1]{} 
% \togglepaper[9]
% }{}

\begin{document}%
\maketitle

\section{Introduction} \label{sec:naumov:1}
This chapter focuses on the distribution of the perdurative in Russian. So-called perdurative verbs in Russian are verbs marked by the so-called perdurative prefix \textit{pro-}. Perdurative-prefixed verbs convey the meaning of temporal duration. The addition of the prefix to a verb stem makes a co-occurring measure expression obligatory, see \REF{ex:naumov:1} and \REF{ex:naumov:1b}.\footnote{There is no consensus on the syntactic status of measure expressions occuring with perdurative verbs. A common view is that they are adjuncts \citep[a.o.]{borik2006aspect,ramchand2008verb,gehrke2008ps}. Some researchers argue for a quasi-argument status for these phrases \citep{fowler1993argument}. \citet{schoorlemmer1995participial} distinguishes different types of accusative phrases: some of them are treated as arguments, while others as adjuncts. The most recent and at the same time the most radical analysis is presented in \citet{Žaucer2009vp,Žaucer2012syntax}, where all accusative phrases are claimed to be unselected direct objects introduced by the prefix. I will not go into this problem here and refer the interested reader to the cited works.}

\ea \ea \label{ex:naumov:1}
\gll Lena	govori-la	(\hspace{-2pt} ves’	večer)		po	telefonu.\\  
     L.	talk-\textsc{pst} {} all	evening	on	phone\\ 
\glt `Lena talked on the phone for the whole evening.'
\ex\gll Lena pro-govori-la *(\hspace{-2pt} ves' večer)		po telefonu.\\  
     L. \textsc{pro}$_{\textsc{per}}$-talk-\textsc{pst} {} all	evening	on	phone\\ 
\glt `Lena spent the whole evening talking on the phone.'\label{ex:naumov:1b}
\z \z

\noindent While the syntax (\citealt{babko1999zero,ramchand2005time,svenonius2004slavic,romanova2007constructing,gehrke2008ps,Žaucer2009vp,tolskaya2015verbal}, a.o.) and semantics (\citealt{isacenko1960grammaticeskij,bondarko1967,forsyth1970grammar,flier1985scope,zalizniak2000,gehrke2003aspectual,filip2005measures}, a.o.) of the perdurative in Russian has been extensively discussed, the constraints on its distribution, to the best of my knowledge, have not been the subject of a separate research. Nevertheless, data suggest that these constraints are intricate. For example, the perdurative can be combined with some predicates describing atelic processes, \REF{ex:naumov:2} (recall also \REF{ex:naumov:1b}), but not with others, \REF{ex:naumov:3}.

\ea[*]{\label{ex:naumov:2}
\gll Kolja		pro-pisa-l		pismo	vsjo	utro.\\  
     K.		\textsc{pro}$_\textsc{per}$-write-\textsc{pst}	letter	all	morning\\ 
\glt Intended: `Kolja spent the whole morning writing the letter.'}
\z

\ea[]{\label{ex:naumov:3}
\gll Kolja	pro-taska-l		čemodan	neskol’ko	časov.\\  
     K.		\textsc{pro}$_\textsc{per}$-carry-\textsc{pst}	suitcase	several		hours\\ 
\glt `Kolja spent several hours carrying the suitcase.'}
\z

\noindent As \citet{pazelskaya2006} convincingly argue, so-called simplex imperfective stems, like \textit{pisa-} and \textit{taska-}, project an activity event structure. With respect to common diagnostics on subeventual complexity (such as the interpretation of \textit{opjat’} `again' and negation) both \textit{pisat’ pis’mo} `write a letter' and \textit{taskat’ čemodan} `carry a suitcase'
 demonstrate identical properties: they are subeventually simplex denoting events that do not lead to the attainment of the result state. If nothing else is added here, it remains a mystery why the perdurative is bad from the former and fine from the latter.

Moreover, the perdurative cannot be derived from predicates describing some goal-oriented activity that stops before the corresponding result state is attained. An example of such a predicate is \textit{otkryvat' zamok} `open a lock'. The unavailability of the perdurative from this predicate is demonstrated in \REF{ex:naumov:4}.

\ea \textit{Situation:} The lock in the door is broken. After spending half an hour trying to open it, Vasja gave up.\medskip\\ 
\gll *\hspace{-2pt} Vasja		pro-[[otkr]-yva]-l		zamok		polčasa.\\    
     {} V.
     \textsc{pro}$_\textsc{per}$-open-\textsc{yva}-\textsc{pst}	lock		half.an.hour\\ 
\glt Intended: `Vasja spent half an hour opening the lock.' \label{ex:naumov:4}
\z

\noindent Unlike in the examples above, in \REF{ex:naumov:4} the perdurative attaches to the so-called secondary imperfective verb stem containing the so-called secondary imperfective suffix \textit{-yva}. It has been claimed by \citet{altshuler2013,altshuler2014} and \citet{tatevosov2017temporal} that in this type of predicates \textit{-yva} functions as a partitive operator. It extracts (proper) non-final parts (more precisely, stages) from the extension of the original telic event predicate \textit{otrkyt’ zamok} `open a lock'. Why cannot the perdurative successfully operate on this eventuality description?

In principle, the perdurative can be derived from a prediсate based on the secondary imperfective verb stem but only if this predicate has a pluractional interpretation. Consider the examples in \REF{ex:naumov:5}. While the sentence in \REF{ex:naumov:5a} has two possible readings, the sentence in \REF{ex:naumov:5b}, which contains a perdurative-prefixed verb, is only compatible with a scenario in which the agent opened his mouth repeatedly.\footnote{I would like to thank an anonymous reviewer for suggesting scenario №1 for examples in \REF{ex:naumov:5}.} That is, the restriction on the possible interpretation arises at the stage when the perdurative attaches to the stem.

\ea \label{ex:naumov:5} \ea \label{ex:naumov:5a}
\gll Kolja 		otkr-yva-l 		rot 		minutu.\\  
     K. 		open-\textsc{yva}-\textsc{pst} 	mouth 		minute\\ 
\glt `Kolja was opening the mouth for a minute.' \\
\ea[\ding{51}]{Scenario №1: Kolja was opening the mouth wider and wider for a minute until he got cramps in his cheekbones.}
\ex[\ding{51}]{Scenario №2: Kolja was opening the mouth several times for a minute until he realized that he was not producing any sounds.}
\z

\ex{ \label{ex:naumov:5b}
\gll Kolja 	pro-[[otkr]-yva]-l 		rot 		minutu.\\  
     K. \textsc{pro}$_\textsc{per}$-open-\textsc{yva}-\textsc{pst} mouth 		minute\\ 
\glt `Kolja spent a minute opening the mouth.' \\
\ea[\ding{55}]{ Scenario №1: Kolja spent a minute opening the mouth wider and wider until he got cramps in his cheekbones.}
\ex[\ding{51}]{Scenario №2: Kolja spent a minute opening the mouth several times until he realized that he was not producing any sounds.}\z
}
\z\z


\noindent\sloppy Although predicates describing non-culminating events, as in \REF{ex:naumov:4}, and plural events, as in \REF{ex:naumov:5b}, are made up from the same morphological pieces that come in the same order (the secondary imperfective suffix \textit{-yva} attached before the perdurative \textit{pro-}), they still differ with respect to the availability of the perdurative. What is the underlying property distinguishing these classes of predicates that the perdurative is sensitive to?

In the remainder of the paper I will explore restrictions on the distribution of the perdurative in greater detail. In \sectref{sec:naumov:2}, I will show that the distribution of the perdurative is subject to (morpho)syntactic constraints. Applying the diagnostics that were proposed by \citet{tatevosov2009mnozestvennaja,tatevosov2013mnozestvennaja}, I will argue that the perdurative \textit{pro-} is a selectionally restricted prefix. Taking into account the interaction of the perdurative \textit{pro-} with some other affixes, I will claim that it originates in a functional projection between \textit{v}P and AspP. In \sectref{sec:naumov:3}, I will present empirical evidence indicating that the distribution of the perdurative is also subject to semantic constraints. Namely, I will argue that predicates allowing the derivation of the perdurative form a natural class with respect to one semantic property: subevents making up the activity component of these predicates are not in \citeposst{landman1992progressive} ``stage-of relation''. The main results and several open questions are listed in \sectref{sec:naumov:4}.


\section{Determing the position of the prefix} \label{sec:naumov:2}
\subsection{The perdurative \textit{pro-} in the big picture} \label{sec:naumov:2.1}
One of the most important observations concerning prefixes in Russian and, more broadly, in Slavic languages that has been made so far is that they form a heterogeneous class and fall at least into two types: \textsc{lexical prefixes} (LPs) and \textsc{superlexical prefixes} (SLPs), or internal and external in other terminology. For the first time this dichotomy was argued for by \citet{babko1999zero} and subsequently became the subject of intense discussion \citep[see, e.\,g.,][]{svenonius2004slavic, ramchand2005time, romanova2005superlexical, romanova2007constructing, tolskaya2015verbal}. SLPs have several characteristics that distinguish them from lexical ones:
\begin{itemize}
    \item External prefixes merge outside VP.
\item External prefixes are attached over the internal ones.
\item External prefixes do not affect the argument 	structure of the verb stem, or predictably decrease its transitivity.
\item External prefixes are semantically compositional.
\item External prefixes express temporal or quantifying meanings.
\end{itemize}

\noindent Among the works listed above, there is no agreement on the position of the perdurative \textit{pro-} within this classification. Some authors include it in the list of SLPs, while others do not.


\begin{table}[h!]
\caption{The perdurative \textit{pro-} as an SLP}
\small
\label{tab1}
\begin{tabularx}{\textwidth}{cCCCCC}
  \lsptoprule
            \citeauthor{babko1999zero}  &	\citeauthor{ramchand2005time} &	\citeauthor{svenonius2004slavic}  &	\citeauthor{romanova2007constructing}  &	\citeauthor{gehrke2008ps}  &	\citeauthor{tolskaya2015verbal}\\ 
            (\citeyear{babko1999zero}) & (\citeyear{ramchand2005time}) & (\citeyear{svenonius2004slavic}) & (\citeyear{romanova2007constructing}) & (\citeyear{gehrke2008ps}) & (\citeyear{tolskaya2015verbal}) \\
  \midrule
 $+$ &    $+$  &    $-$ & $-$ & $+$ & $-$\\
  \lspbottomrule
 \end{tabularx}
\end{table}  

This binary opposition has been recently refined. Focusing on Russian data, \citet{tatevosov2009mnozestvennaja,tatevosov2013mnozestvennaja} claims that prefixes commonly subsumed under the label SLPs fall, in fact, into at least two distinct groups with respect to constraints that regulate their distribution. Namely, there are prefixes that demonstrate selectional restrictions, \REF{ex:naumov:6a}, and prefixes that demonstrate positional restrictions, \REF{ex:naumov:6b}.\footnote{ The property of being ``formally (im)perfective'' does not imply carrying any aspectual semantics. This is a morphological notion.}


\ea \label{ex:naumov:6} Possible restrictions on the distribution of SLPs:
\ea \label{ex:naumov:6a} Selectional restriction. The possibility of attaching a prefix to a stem can be constrained by the stem's formal (im)perfectivity.
\ex \label{ex:naumov:6b} Positional restriction. The possibility of attaching a prefix to a stem can be constrained by the positional relationship between this prefix and the secondary imperfective suffix \textit{-yva}.
\z \z

\noindent With respect to these two restrictions there emerge \textsc{selectionally restricted} (SR) and \textsc{positionally restricted} (PR) prefixes. If the distribution of a given prefix can be described through \REF{ex:naumov:6a}, this prefix is said to be an SR-prefix. If the distribution of a given prefix can be described through \REF{ex:naumov:6b}, this prefix is said to be a PR-prefix. The list of SR-prefixes includes: delimitative \textit{po-}, cumulative \mbox{\textit{na-},} distributive \textit{pere-}, inchoative \textit{za-}. The list of PR-prefixes includes: completive \mbox{\textit{do-},} repetitive \textit{pere-}, attenuative \textit{pod-}, attenuative \textit{pri-}. The perdurative \textit{pro-} is not considered separately by \citet{tatevosov2009mnozestvennaja, tatevosov2013mnozestvennaja}.

There is one more alternative view on the lexical/superlexical distinction found in  \citet{Žaucer2009vp,Žaucer2012syntax}. The author of these works consistently argues that at least some SLPs merge within the same resultative projection as LPs. The perdurative \textit{pro-} is claimed to be one of these SLPs. It should be pointed out, however, that \citeauthor{Žaucer2009vp}'s (\citeyear{Žaucer2009vp,Žaucer2012syntax}) proposal is based mainly on data from Slovenian. The present work does not set as its goal to revise it. What I aim to do is to try to determine the position of the perdurative \textit{pro-} in Russian. For this I will use diagnostics proposed by \citet{tatevosov2009mnozestvennaja, tatevosov2013mnozestvennaja}.

\subsection{The position of the perdurative \textit{pro-}} \label{2.2}
In this part of the paper, I will follow \cite{tatevosov2009mnozestvennaja,tatevosov2013mnozestvennaja} and assume that SLPs fall into at least two separate classes: SR-prefixes and PR-prefixes. 
The class membership of a given prefix is determined via the restrictions from \REF{ex:naumov:6}. Below I will show that the distribution of the perdurative \textit{pro-} is subject to selectional restrictions and argue that the perdurative \textit{pro-} is an SR-prefix.

First, the perdurative \textit{pro-} selects for formally imperfective stems. Its distribution falls under the generalization in \REF{ex:naumov:7}. 

\ea \label{ex:naumov:7}
The perdurative \textit{pro-} merges with formally imperfective stems.
\z

\noindent It has already been demonstrated in \sectref{sec:naumov:1} that the perdurative \textit{pro-} can combine both with simplex imperfective stems and imperfective stems derived through applying the secondary imperfective suffix \textit{-yva}. Here, I repeat the relevant examples.

    \ea Perdurative \textit{pro-} with a non-derived (non-prefixed) imperfective stem:\label{ex:naumov:8}\smallskip\\
    \gll Lena	pro-govori-la		ves’	večer		po	telefonu. \\
L. \textsc{pro}$_\textsc{per}$-talk-\textsc{pst}	all	evening	on	phone\\
\glt `Lena spent the whole evening talking on the phone.'
\z


    \ea Perdurative \textit{pro-} with a stem imperfectivized by         \textit{-yva:}\label{ex:naumov:9}\smallskip\\
    \gll Kolja 	pro-[[otkr]-yva]-l 		rot 		minutu.\\  
     K. \textsc{pro}$_\textsc{per}$-open-\textsc{yva}-\textsc{pst} mouth 		minute\\ 
    \glt `Kolja spent a minute opening the mouth.'
\ea[\ding{55}]{\ Scenario №1: Kolja spent a minute opening the mouth wider and wider until he got cramps in his cheekbones.}
\ex[\ding{51}]{Scenario №2: Kolja spent a minute opening the mouth several times until he realized that he was not producing any sounds.}
\z\z 

\noindent What is the source of the restriction on possible interpretations in \REF{ex:naumov:9}?

As was shown in \sectref{sec:naumov:1} (recall \REF{ex:naumov:5a}), when the perdurative \textit{pro-} is not attached above  \textit{-yva}, the sentence is also compatible with a progressive interpretation. \citet{tatevosov2015severing} claims that when \textit{-yva} induces a progressive interpretation, it functions as a partitive operator and merges in a position between VP and \textit{v}. One can naturally assume that the second (pluractional) interpretation of the sentence in \REF{ex:naumov:5a} is also induced by the \textit{-yva} suffix and that in this case it functions as a pluractional operator. It has been proposed that pluractional operators apply very low in the syntactic structure, namely, at the level of V (see, e.g, \citealt{lasersohn1995plurality,van2004adverbials}). If this view is correct, the restriction in \REF{ex:naumov:9} could be treated as (morpho)syntactic. Namely, despite no surface difference, it could be stated that \textit{-yva} has the possibility to enter the derivation in two distinct hierarchical positions: within VP and above it. When the suffix induces a progressive interpretation, it merges, as \citet{tatevosov2015severing} proposes, between VP and \textit{v}. In contrast, when this suffix induces a pluractional interpretation, it adjoins to V. The incompatibility of the perdurative \textit{pro-} and the ``partitive'' \textit{-yva} could be explained by claiming that they compete for the same position and, thus, block the derivation. Such a configuration would look like in \REF{ex:naumov:10}. 

\ea \label{ex:naumov:10}
{[\textsubscript{\textit{v}P} {\ldots} [\textsubscript{\textit{v}$'$} \ldots [\textsubscript{F} \textit{pro}\textsubscript{\textsc{per}}- \textit{-yva}\textsubscript{\textsc{part}} {\ldots} [\textsubscript{VP} {\ldots}]]]].}
\z

\noindent The explanation for the compatibility of the perdurative \textit{pro-} and the ``pluractional''
\textit{-yva} would be that the latter occupies a position lower in the tree and does not prevent the former from merging with a stem, \REF{ex:naumov:11}.

\ea \label{ex:naumov:11}
{[\textsubscript{\textit{v}P} {\ldots} [\textsubscript{\textit{v}$'$} {\ldots} [\textsubscript{F} \textit{pro}\textsubscript{\textsc{per}}- {\ldots} [\textsubscript{VP} {\ldots} [\textsubscript{V$'$} \textit{-yva}\textsubscript{\textsc{iter}}  {\ldots}]]]]].}
\z

\noindent While this line of reasoning may be true, I cannot come up with any empirical evidence in favor of it.\footnote{In fact, below I will argue, relying on the fact that the perdurative \textit{pro-} stacks above the repetitive \textit{pere-}, that the perdurative \textit{pro-} merges above \textit{v}P. This, if true, can be taken as an indirect argument against such a reasoning.} Moreover, as \citet{gianina2015pluractionality} show, there are a number of empirical and theoretical challenges for the view that pluractional operators are V-level operators. They argue for high aspect-level pluractionality. I will pursue a different path and try to show in \sectref{3} that the restriction we observe in examples like \REF{ex:naumov:5b} and \REF{ex:naumov:9} occurs due to semantic reasons and can be explained without any specific assumptions about the syntax of pluractional operators.

On the flip side, the perdurative \textit{pro-} does not select for formally perfective stems. Its distribution falls under the generalization in \REF{ex:naumov:12}. 

\ea \label{ex:naumov:12}
The perdurative \textit{pro-} does not merge with formally perfective stems.
\z

    \ea Perdurative \textit{pro-} with a non-derived (non-prefixed) perfective stem:\label{ex:naumov:13}\smallskip\\
    \gll * Vasja		pro-reši-l			zadanije	desjat’		minut.\\
    {} V.		\textsc{pro}$_\textsc{per}$-solve-\textsc{pst}		task		ten		minutes
    \\
    \glt \ \ Intended: `Vasja spent ten minutes solving the task.'
    \z

    \ea Perdurative \textit{pro-} with a perfective stem derived by prefixation:\label{ex:naumov:14}\smallskip\\
    \gll * Maša	pro-[na-[pisa]]-la		pis’mo		dva	časa.\\
    {} M.	\textsc{pro}$_\textsc{per}$-\textsc{na}-write-\textsc{pst}	letter		two	hours\\
    \glt \ \ Intended: `Maša spent two hours writing the letter.'
    \z


\noindent As can be seen from the examples, the perdurative \textit{pro-} cannot combine either with simplex perfective stems, \REF{ex:naumov:13}, or with perfective stems derived by prefixation, \REF{ex:naumov:14}.


The intermediate conclusion that can be drawn at this stage is the following. The distribution of the perdurative \textit{pro-} is constrained by the (im)perfectivity of the stem. Specifically, the perdurative \textit{pro-} merges with formally imperfective stems and does not merge with formally perfective ones. Therefore, the distribution of the perdurative \textit{pro-} is subject to selectional restrictions and hence the perdurative \textit{pro-} is an SR-prefix. Next, I will consider constraints that regulate the distribution of PR-prefixes and argue that (i) the perdurative \textit{pro-} is not subject to these constraints; (ii) the syntactic position in which the perdurative \textit{pro-} merges is above \textit{v}P.

PR-prefixes do not impose restrictions on the (im)perfectivity of the stem with which they combine \citep{tatevosov2009mnozestvennaja,tatevosov2013mnozestvennaja}. As was shown in \REF{ex:naumov:13} and \REF{ex:naumov:14}, this does not hold for the perdurative \textit{pro-} because the perdurative \textit{pro-} requires the stem to which it attaches not to be formally perfective.

The distribution of PR-prefixes falls under the generalization in \REF{ex:naumov:15}.

\ea \label{ex:naumov:15}
PR-prefixes do not merge above the secondary imperfective suffix \textit{-yva}.
\z

\noindent As was shown in \REF{ex:naumov:9}, repeated here as \REF{ex:naumov:16}, this generalization also does not hold for the perdurative \textit{pro-}.
 

 \ea Perdurative \textit{pro-} above the secondary imperfective suffix \textit{-yva:}\label{ex:naumov:16}\smallskip\\
     \gll Kolja	pro-[[otkr]-yva]-l		rot		minutu.\\
K.	\textsc{pro}$_\textsc{per}$-open-\textsc{yva}-\textsc{pst}	mouth		minute\\
     \glt `Kolja spent a minute opening the mouth.'
     \z


\noindent Another configuration, when it attaches under the secondary imperfective, is also present, \REF{ex:naumov:17}.

\ea The perdurative \textit{pro-} under the secondary imperfective suffix \textit{-yva:}\label{ex:naumov:17}\smallskip\\
    \gll Prošloj	zimoj	ja	[\hspace{-2pt} pro-[lež]]-iva-l	na	divane	po	desjat’ časov	v	den’.\\
    Last	winter	I	{} \textsc{pro}$_\textsc{per}$-lay-\textsc{yva}-\textsc{pst}	on	sofa	by	ten hours	in	day \\
    \glt `Last winter I spent ten hours a day laying on the sofa.'
    \z


\noindent Unlike PR-prefixes, SR-prefixes do not have restrictions relative to the position of the secondary imperfective \textit{-yva}. The fact that the perdurative \textit{pro-} can merge both under and above the suffix also unites it with SR-prefixes.


There is one more observation concerning SR- vs. PR-prefixes dichotomy: SR-prefixes attach above PR-prefixes and they cannot merge as adjacent heads, \textit{-yva} must merge between them \citep{tatevosov2009mnozestvennaja,tatevosov2013mnozestvennaja}. As the example in \REF{ex:naumov:18} demonstrates this is true for the perdurative \textit{pro-}. In this example, the perdurative \textit{pro-} attaches to the stem that already contains the repetitive suffix \textit{pere-}, which is assumed to be a PR-prefix.\footnote{Note that \REF{ex:naumov:18} is compatible only with a pluractional interpretation, as is reflected in the translation.}

\ea \label{ex:naumov:18}
\gll Vasja	\minsp{\{} pro-[[pere-[čit]]-yva]-l / \minsp{*} pro-[pere-[čita]]-l	\} etot	abzac polčasa,	no 	ničego		ne	ponja-l. \\
V. {} \textsc{pro}$_\textsc{per}$-\textsc{pere}$_\textsc{rep}$-read-\textsc{yva}-\textsc{pst} {} {} \textsc{pro}$_\textsc{per}$-\textsc{pere}$_\textsc{rep}$-read-\textsc{pst}		{} 	this	paragraph	half.an.hour	but	nothing	not	understand-\textsc{pst} \\
\glt `Vasja spent half an hour reading this paragraph over and over again, but did not understand anything.'
\z

\noindent PR-prefixes have the possibility to enter the derivation in two distinct syntactic positions: between VP and \textit{v}P, and above \textit{v}P \citep{tatevosov2008intermediate}. The position of the prefix affects the interpretation of the predicate. If the prefix merges before \textit{v}P is projected and takes in its scope only the result state, the restitutive reading obtains. In contrast, if the prefix merges after \textit{v}P is projected and takes in its scope the whole event, the repetitive reading obtains. In \REF{ex:naumov:18}, we observe the second possibility: \textit{pere-} enters the derivation above \textit{v}P, and, thus, yields the repetitive reading `Vasja read the paragraph, and that had happened before'. Therefore, as the perdurative \textit{pro-} attaches above \textit{pere-}, it must merge in a position above \textit{v}P. We are open to two different possibilities: (a) the perdurative \textit{pro-} merges in AspP and (b) the perdurative \textit{pro-} merges lower, between \textit{v}P and AspP. Although (a) seems to be more straightforward, below I will speak in favor of (b).\footnote{Note that in principle both of them are consistent with the semantic proposal made in \sectref{3}.}

In formal Slavic literature, there have been various proposals regarding the position in which SLPs merge. They can be divided into two groups with respect to how they treat the hierarchical relationship between prefixes and perfectivity: (i) perfectivity is introduced as high as prefixes; (ii) perfectivity is introduced higher than prefixes.

The first view is the mainstream and presented in a large number of works on the syntax of prefixation. For example, \cite{babko1999zero} treats SL-prefixes as left-adjuncts to Asp and \cite{svenonius2004slavic} treats SL-prefixes as PPs occupying SpecAspP (see also \citealt{ramchand2005time} for a similar proposal). These authors assume that perfectivity is directly encoded in SL-prefixes or introduced sufficiently local to them.

The second view is less popular and presented in \citet{pinon1994} and \citet{filip2000quantization, filip2005measures,filip2008events}. These authors focus on the delimitative prefix \textit{po-} and claim that it is an event modifier with the semantics of a measure adverbial. The result of its application is an event predicate, not a property of times. According to these works, perfectivity is not part of the meaning of the delimitative \textit{po-}. It is introduced by a phonologically silent operator located higher in the syntactic structure.

The crucial thing is that these approaches have different predictions about the possibility of multiple pieces of aspectual morphology within a word form. Specifically, if we assume that the perdurative \textit{pro-} merges in the projection of aspectual operators, we do not expect any other aspectual morphology after its application. Data, however, suggest just the opposite. It has already been shown in \REF{ex:naumov:17}, repeated here as \REF{ex:naumov:19}, that the secondary imperfective suffix \textit{-yva} can attach above the perdurative \textit{pro-}.

\ea \label{ex:naumov:19}
\gll Prošloj	zimoj	ja	[\hspace{-2pt} pro-[lež]]-iva-l	   na	divane	   po	desjat’ časov	v	den’. \\
Last	winter	I   {}	\textsc{pro}$_\textsc{per}$-lay-\textsc{yva}-\textsc{pst}	   on	sofa	   by	ten hours	in	day \\
\glt `Last winter I spent ten hours a day laying on the sofa.'
\z

\noindent If we take aspectual morphology to be merged in the projection of aspectual operators, then in the case when there are multiple pieces of this morphology within a word form, as in \REF{ex:naumov:19}, we have to postulate several adjacent AspP projections. However, there exists a well-established ban on consecutive identical projections. \citet{de2018negation} have recently proposed the following formulation of this constraint:

\ea[*] {\label{ex:naumov:20}
$\langle$X, X$\rangle$ \\
The functional sequence must not contain two immediately consecutive identical projections.}
\z

\noindent That is, the syntactic structure with several adjacent AspP projections would contradict the constraint on admissible functional sequence and would be undesirable from a theoretical point of view. This is the reason why I opt for (b): the perdurative \textit{pro-} merges in a functional projection between \textit{v}P and AspP.\footnote{A reviewer poses the following question: ``Which projection is positioned there? Are there more projections present? In fact, it seems to me that there could be only different types of aspectual projections like in \citet{cinque1999}''. While I have no basis to make more concrete claims regarding the nature of this projection, I assume that an approach in the spirit of \citet{cinque1999} can be potentially undertaken. For example, \citet{markova2011} proposes that some prefixes in Bulgarian are derived in dedicated aspectual projections a la \citet{cinque1999}.}

Summarizing all that has been discussed so far, I make the following conclusions:

\begin{itemize}
    \item The distribution of the perdurative \textit{pro-} is subject to constraints that regulate the distribution of SR-prefixes.
\item The syntactic position in which the perdurative \textit{pro-} merges is between \textit{v}P and AspP.
\item The perdurative \textit{pro-} felicitously combines with predicates based on secondary imperfective verb stems only when they receive a pluractional interpretation.
\end{itemize}

\noindent In the next section, I will try to argue for a unified semantic constraint that regulates the distribution of the perdurative in Russian. The proposal will be based on the observation about the specific behaviour of pluractional predicates with respect to \citeposst{landman1992progressive} stage-of relation.

\section{A semantic constraint on the distribution of the perdurative} \label{sec:naumov:3}
\subsection{Theoretical background} \label{sec:naumov:3.1}

I adopt a neo-Davidsonian version of event semantics where verbs are represented as one-place predicates over sets of eventualities. I assume that the denotation of \textit{v}P is a predicate of events which is mapped to predicates of times at Asp (see, e.g., \citealt{wolfgang1994time} and much further literature). In addition, adhering to the common view in predicate decomposition, I presuppose that while accomplishment predicates are subeventually complex and consist of, at least, two separate subevents—activity (or process subevent) and result state (or become subevent)—that are connected by a finite set of causal relations (\citealt{dowty1979word}; \citealt{rothstein2004structuring}; a.o.), activity predicates have simple structures consisting of a single activity subevent. I assume, following \citet{pazelskaya2006}, that predicates based on ``simplex imperfective'' verb stems are associated with an activity event structure. Following \citet{altshuler2013,altshuler2014} and \citet{gronn2003semantics, gronn2013}, I take both simplex imperfectives and secondary imperfectives as denoting (not necessarily proper) parts (or stages) of complete eventualities. 

\subsection{Previous studies} \label{sec:naumov:3.2}
To the best of my knowledge, the semantics and the distribution of the perdurative in Russian have not been the subject of a separate study so far. Traditional Slavic aspectology is rather consistent in giving a more or less uniform treatment of the meaning of the perdurative. For example, \cite[243--244]{isacenko1960grammaticeskij} proposes that perdurative verbs denote ``the completion of a process that lasted for a specific period of time''. \cite[16]{bondarko1967} postulate a ``long-term Aktionsart''; its nucleus includes ``verbs with the prefix \textit{pro-} denoting an action that covers a specific period of time''. \cite[23]{forsyth1970grammar} notes that verbs marked by the perdurative \textit{pro-}, as opposed to verbs marked by the delimitative \textit{po-}, ``suggest a longer period of time'' during which an action is performed. A ``long-term'' or ``perdurative'' Aktionsart is postulated by \cite[112--113]{zalizniak2000}. The authors claim that this Aktionsart includes ``verbs denoting an action that took place within a closed specific period of time''. In formal literature, we find similar opinions on the semantics of the perdurative. For example, \cite[26]{gehrke2003aspectual} indicates that the perdurative ``refers to an unexpectedly long duration of a situation, where it always has to be made explicit that this duration is specific''. \cite[32]{filip2005measures} notes that the perdurative \textit{pro-} ``indicates a relatively long temporal extent of the event (with connotations of wasted time, boredom, and the like).'' 

Although the inference that the eventuality from the extension of the verbal predicate marked by the perdurative prefix \textit{pro-} lasts long is indeed frequently present, it is, in fact, optional and can be canceled by some linguistic means. Consider, e.g., the sentence in \REF{ex:naumov:21}, where the perdurative verb is followed by the adverb \textit{vsego liš’} `just, only'.


\ea \label{ex:naumov:21}
\gll  V	očeredi		Kolja	pro-stoja-l		vsego	liš’	sutki.\\
in	queue		K.	\textsc{pro}$_\textsc{per}$-stay-\textsc{pst} total	only	day \\
\glt `Kolja stood in the queue only for a day.'
\z

\noindent Here, the use of the adverb explicitly indicates that an event of staying described by the perdurative verb \textit{pro-stojat’} `stay' did not last long relative to expectations. The fact that the inference of long duration arises in the presence of the perdurative \textit{pro-} and can be absent in the appropriate context, suggests that this inference is an implicature. The generation of implicatures is generally taken to be due to the non-use of a non-weaker alternative: the choice of one alternative implies the negation of the other. The only candidate for the position of the alternative item for the perdurative \textit{pro-} is the so-called delimitative prefix \textit{po-}, which also conveys the meaning of temporal duration. Indeed, the perdurative and the delimitative are usually considered together and sometimes assumed to impose similar restrictions on the predicates with which they combine (see, e.g., the references above). Constraints on the distribution of the delimitative have been recently explored by \citet{tatevosov2017temporal}. In the next subsection, I will give an outline of the theory developed in these works and take it as a starting point in identifying constraints on the distribution of the perdurative. 

\subsection{The delimitative and unique temporal arrangement} \label{sec:naumov:3.3}

\citet{tatevosov2017temporal} claims that in Russian the derivation of non-culminating accom\-plish\-ments—predicates that appear in perfective clauses and describe some goal-oriented activity that stops before the corresponding result state is attained—proceeds in two steps: the secondary imperfective suffix \textit{-yva} first merges with a formally perfective stem and then the delimitative prefix \textit{po-} attaches to the resulting complex.\footnote{\citet{tatevosov2017temporal} proposes that non-culminating accomplishments exist in Russian, and the delimitative \textit{po-} is a means to derive them. There are, however, alternative views on this problem. For example, \citet{martin2017non} claims that Russian does not possess non-culminating accomplishments, and the delimitative-prefixed verbs do not exemplify this class of predicates. I believe that this theoretical debate is orthogonal to the current purposes.} Each step is subject to certain semantic restrictions. The ones that we are interested in here are those that emerge when the delimitative comes into play. \citet{tatevosov2017temporal} proposes that the delimitative can be derived from an event predicate iff the activity component of this predicate does not contain in its extension subevents that are temporally ordered in a unique way. The notion of unique temporal orderedness is formalized through the property of \textsc{unique temporal arrangement} (UTA). An informal definition of UTA is given in \REF{ex:naumov:22}.

\eanoraggedright \textit{Definition:} Unique temporal arrangement\smallskip\\
Whenever an event $e$ falls under $P$, there is exactly one way for $e$ to start, there is exactly one way for $e$ to finish, and for any non-final part of $e$ there is exactly one follow up.\label{ex:naumov:22}
\z

\noindent This property is best understood by looking at concrete examples. Let us compare predicates \textit{zapivat’ tabletku} `wash down a pill' and \textit{zapolnjat’ anketu} `fill in a form', which are both based on secondary imperfective verb stems and within our assumption denote non-final parts of complete eventualities. The predicate \textit{zapivat’ tabletku} `wash down a pill' does not allow the formation of the delimitative, while \textit{zapolnjat’ anketu} `fill in a form' does. These predicates differ with respect to the UTA property. The activity component of the predicate \textit{zapivat’ tabletku} `wash down a pill' consists of very specific subevents that have to come in a very specific order to represent an activity of washing down a pill. In contrast, the predicate \textit{zapolnjat’ anketu} `fill in a form' denotes events in which subevents making up the activity of filling in a form do not have to come in any specific order. Even if some of them are skipped or occur more than once, their sum still represents an activity of filling in a form. According to \citet{tatevosov2017temporal}, the delimitative \textit{po-} is sensitive to this difference. Crucially, the delimitative \textit{po-} can also combine with predicates based on simplex imperfective verb stems, like \textit{govorit’ po telefonu} `talk on a phone', which are associated with an acitivity event structure and also do not possess UTA. Putting these facts together, \citet{tatevosov2017temporal} concludes that the complement of the delimitative, whether it is derived or non-derived, must be an Activity.

The important fact for the purposes of this work is that, despite superficial similarity, the perdurative and the delimitative impose different requirements on predicates with which they combine. As was shown in \sectref{sec:naumov:1}, the perdurative, unlike the delimitative, can be derived from some activity predicates but cannot be derived from others. Consider the contrast between \REF{ex:naumov:23} and \REF{ex:naumov:24}.\footnote{It must be pointed out that the verb \textit{taskat’} in \REF{ex:naumov:23b} and \REF{ex:naumov:24b} is an ``indeterminate'' motion verb. Indeterminate motion verbs are used to describe undirected motion, motion back and forth, motion that is not associated with any particular path being covered during the unfolding of an event. Neither the delimitative nor the perdurative can be derived from a predicate when it describes motion in a single direction.}

\ea \label{ex:naumov:23} \ea [*]{\label{ex:naumov:23a}
\gll  Kolja		pro-pisa-l		pismo		neskol’ko	minut.\\
K. \textsc{pro}$_\textsc{per}$-write-\textsc{pst}   letter		several		minutes    \\      
\glt Intended: `Kolja spent several minutes writing the letter'.}
\ex[]{\label{ex:naumov:23b}
\gll Kolja	pro-taska-l		čemodan	neskol’ko	časov. \\
K. \textsc{pro}$_\textsc{per}$-carry-\textsc{pst} suitcase	several		hours \\
\glt `Kolja spent several hours carrying the suitcase.'}
\z \z

\ea \label{ex:naumov:24} \ea \label{ex:naumov:24a}
\gll  Kolja	po-pisa-l		pismo		neskol’ko	minut.\\
K. \textsc{po}$_\textsc{del}$-write-\textsc{pst}   letter		several		minutes    \\      
\glt `Kolja spent several minutes writing the letter'.
\ex \label{ex:naumov:24b}
\gll Kolja	po-taska-l		čemodan	neskol’ko	časov. \\
K. \textsc{po}$_\textsc{del}$-carry-\textsc{pst} suitcase	several		hours \\
\glt `Kolja spent several hours carrying the suitcase.'
\z \z

\noindent Since the boundary separating predicates that allow the perdurative from predicates that do not allow it runs inside the class of activities, more should be said about their internal structure. A more specific question that I would like to address is: are there any properties that differentiate predicates like \textit{pisat’ pismo} `write a letter’ from predicates like \textit{taskat’ čemodan} `carry a suitcase’? The next subsection is devoted to seeking an answer to this question.

\subsection{Theories of atelicity} \label{sec:naumov:3.4}

In the literature on aspectual composition, it is consistently argued that activity predicates are atelic. What lies under this notion varies across different approaches. In their influential work, carried out within the framework of temporal semantics, \citet{bennett1978toward} propose that atelic predicates must satisfy the \textsc{subinterval property}.

\ea \label{ex:naumov:25}
\textit{Definition:} Subinterval property\smallskip\\
$\cnst{sub}(P) \leftrightarrow \forall i [\cnst{at}(P, i) \shortrightarrow \forall j[j \subset i \shortrightarrow \cnst{at}(P, j)]]$ \\
\textit{P} possesses the subinterval property iff, if \textit{P} is true at $i$, it is true at every subinterval of $i$.
\z

\noindent This definition, however, is too strong and gives rise to the well-known ``minimal-parts problem'' \citep{dowty1979word}. The problem is that it is not the case that all activity predicates are true at every subinterval during which the events from their extension take place. Dynamic eventualities never hold at points, they take time to establish themselves. Therefore, it is reasonable to judge whether a given predicate possesses the subinterval property only relative to an interval that is ``sufficiently large" for the event from its extension to unfold.

Now that we have equipped ourselves with the notion of subinterval property, we can return to the question formulated at the end of the previous subsection and consider the predicates \textit{pisat’ pismo} `write a letter’ and \textit{taskat’ čemodan} `carry a suitcase’ in detail. These predicates are based on underived imperfective stems and, within the assumption introduced in \sectref{sec:naumov:3.1}, denote non-final parts of complete events and belong to the class of activities. The subinterval property is not what distinguishes them: both \textit{pisat’ pismo} `write a letter’ and \textit{taskat’ čemodan} `carry a suitcase’ possess this property. If a nonfinal part of an event of writing a letter holds at an interval $i$, it also holds at every sufficiently large subinterval of $i$. Similarly, if a non-final part of an event of carrying a suitcase holds at an interval $i$, it also holds at every sufficiently large subinterval of $i$. The perdurative, however, can be derived only from the latter but not from the former. 

In the mereological approach to aspectual composition \citep{krifka1989nominal,krifka1989nominal, krifka1992nominal,krifka1998origins}, atelic predicates are defined as being cumulative. A predicate is \textsc{cumulative} if it satisfies the properties of \textsc{additivity}, \REF{ex:naumov:26}, and \textsc{divisivity}, \REF{ex:naumov:27}.

\ea \label{ex:naumov:26}
\textit{Definition:} Additivity\smallskip\\
$\forall P[\cnst{cum}(P) \leftrightarrow \forall x \forall y[P(x) \wedge P(y) \shortrightarrow P(x \oplus y)]]$ \\
$P$ is additive iff whenever it applies to the entities $x$ and $y$, it also applies to the sum $x \oplus y$
\z

\ea \label{ex:naumov:27}
\textit{Definition:} Divisivity\smallskip\\
$\forall P[\cnst{div}(P) \leftrightarrow \forall x \forall x' [P(x) \wedge x' \subset x \shortrightarrow P(x')]]$ \\
$P$ is divisive iff whenever $P$ applies to $x$, then it must also apply to any $x'$ that is properly included in $x$.
\z

\noindent The notion of cumulativity also cannot help us to distinguish \textit{pisat’ pismo} `write a letter’ from \textit{taskat’ čemodan} `carry a suitcase’. Both these predicates are additive and divisive down to minimal parts.\footnote{Due to space limitations, I leave the verification of this statement to the reader.}

\citet{landman2010incremental,landman2012felicity} claim that neither the subinterval property nor the property of cumulativity is adequate for distinguishing activities from other classes of predicates. They argue that activity predicates are lexically constrained as being incrementally homogeneous.

\textsc{Incremental homogeneity} is based on two essential components: \textsc{cross-temporal identity} and \textsc{event onsets}. Cross-temporal identity is a semantic primitive that is used to compare events with different running times. Events with different running times count as the same event if they are in an equivalence relation of cross-temporal identity, \REF{es28}.

\eanoraggedright \label{es28}
\textit{Definition:} Equivalence relation of cross-temporal identity\smallskip\\
$e_1$ is cross-temporally identical to $e_2$, $e_1 \sim e_2$ iff $e_1$ and $e_2$ count as one and the same event, i.e. for counting purposes $e_1$ and $e_2$ count as one event.
\z

\noindent The second component is the notion of onset of an event. For $e$, the onset of $e$ is the smallest initial part of $e$ that is big enough to count both as $e$ and as cross-temporally identical to $e$, \REF{ex:naumov:29}.

\ea \label{ex:naumov:29}
\textit{Definition:} Onset of an event\smallskip\\
Let $e$ be an eventuality of verb type $V$. \\
The onset of $e$, relative to $V$, $O(e,V)$ is the smallest eventuality of type $V$ such that: $O(e,V) \sim e$ and $\tau(O(e,V) \subseteq_{\cnst{in}} e$.
\z

\noindent Taking these, the notion of incremental homogeneity is defined, \REF{ex:naumov:30}.
\ea \label{ex:naumov:30}
\textit{Definition:} Incremental homogeneity\smallskip\\
Let $\alpha$ be a VP with event type $\alpha$ and verbal event type $V_\alpha$.

Let $e\in V_\alpha$ and $e \in \alpha$. \\
$e$ is incrementally homogeneous w.r.t. $\alpha$ and $V_\alpha$ iff for every interval $i$: if $\tau(O(e,V)) \subseteq_{\cnst{in}} i \subseteq_{\cnst{in}} \tau(e)$ then there is an eventuality $e'$ of event type $\alpha$ such that: $e' \sim e$ and $\tau(e') = i$.
\z

\noindent For an event $e \in \alpha$ to be incrementally homogeneous, the onset of $e$ must count both as event type V$_\alpha$ and as event type $\alpha$. If $\alpha$ does not hold at the onset of $e$, neither $e$ is incrementally homogeneous nor the predicate that is true of $e$. This is exactly what happens to predicates like \textit{pisat’ pismo} `write a letter’. Let the predicate \textit{pisat’ pismo} `write a letter’ be an event type $\alpha$, with verbal event type V$_\alpha$, which is \textit{pisat’} `write’.

\ea \label{ex:naumov:31} \ea
 $\alpha = \lambda e. \textsc{write}(e) \wedge \cnst{theme}(\textsc{letter})(e)$ \\
    Event type of writing letter events
\ex $V_\alpha = \lambda e. \textsc{write}(e)$ \\
    Event type of writing events
\z \z

\noindent Suppose that $e$ is a non-final part of an event of writing a letter. According to the definition, the onset of $e$ is its most initial proper part that counts as a writing activity. It is not the case, however, that this onset must count as a non-final part of an event of writing a letter. An event of someone diligently tracing out the first letter of the letter undoubtedly counts as writing but very unlikely counts as writing a letter. Thus, the predicate \textit{pisat’ pismo} `write a letter’ is not incrementally homogeneous, and the perdurative cannot be derived from it.

In contrast, \textit{taskat’ čemodan} `carry a suitcase’ is an incrementally homogeneous predicate. Suppose that $e$ is a non-final part of an event of carrying a suitcase. According to the definition, the onset of $e$ is its most initial proper part that counts as a carrying activity. What is crucial is that this onset counts also as a non-final part of an event of carrying a suitcase. An event of someone taking a suitcase and making a few steps in different directions counts both as an event of carrying and as an event of carrying a suitcase. Thus, the predicate\textit{ taskat’ čemodan} `carry a suitcase’ is incrementally homogeneous, and the perdurative can be derived from it.

Can we stop here and say that the restriction on incremental homogeneity is what regulates the distribution of the perdurative? The answer is negative. There are incrementally homogeneous predicates that are marginal with the perdurative. These are the predicates in which, as \citet[106]{tatevosov2009event} claim, ``the activity that up to its final point does not contribute to the development of the become subevent at all''. An example of such a predicate is \textit{otkryvat’ zamok} `open a lock’. This is an atelic predicate obtained from its telic counterpart \textit{otkryt’ zamok} `open a lock’ through applying the secondary imperfective suffix \textit{-yva}. Within our assumption, this predicate denotes non-final parts of a complete event of opening a lock. These non-final parts do not induce any change of state. The result state is brought about by the very final subevent. This subevent, however, is not in the denotation of the secondary imperfective \textit{otkryvat’ zamok} `open a lock’. 

Consider the example in \REF{ex:naumov:32}. In the provided situation, the sentence is judged as ungrammatical.

\ea \label{ex:naumov:32} \textit{Situation:} The lock in the door is broken. After spending half an hour trying to open it, Vasja gave up.\smallskip\\ 
\gll * Vasja		pro-[[otkr]-yva]-l		zamok		polčasa.\\  
    {} V.		\textsc{pro}$_\textsc{per}$-open-\textsc{yva}-\textsc{pst}	lock		half.an.hour\\ 
\glt \ \ Intended: `Vasja spent half an hour opening the lock.'
\z

\noindent It is easy to show that this predicate is indeed incrementally homogeneous. Let $e$ be a non-final part of a complete event of opening a lock. According to the definition of onset given above, the onset of $e$ is its most initial proper part that counts as an opening activity. Crucially, this onset counts as a non-final part of an event of opening a lock, too. The very first manipulation with a lock aimed at unlocking it counts both as opening and as opening a lock. Thus, the predicate \textit{otkryvat’ zamok} `open a lock’ is incrementally homogeneous. However, the perdurative is blocked.

In the next subsection, I will try to identify the property that distinguishes predicates that allow the perdurative from predicates that do not and try to formulate a unified semantic constraint that regulates the distribution of the perdurative \textit{pro-} in Russian.

\subsection{The perdurative and Arrangement by Stages} \label{sec:naumov:3.5}

I start this subsection from the observation, previously presented in \sectref{sec:naumov:1}, that the perdurative can be derived from predicates based on the secondary imperfective verb stems only if they receive a pluractional interpretation, \REF{ex:naumov:33}.

\ea{ \label{ex:naumov:33}
\gll Kolja 	pro-[[otkr]-yva]-l 		rot 		minutu.\\  
     K. \textsc{pro}$_\textsc{per}$-open-\textsc{yva}-\textsc{pst} mouth 		minute\\ 
\glt `Kolja spent a minute opening the mouth.'
\ea[\ding{55}]{\ Scenario №1: Kolja spent a minute opening the mouth wider and wider until he got cramps in his cheekbones.}
\ex[\ding{51}]{Scenario №2: Kolja spent a minute opening the mouth several times until he realized that he was not producing any sounds.}\z}
\z 

\noindent The above contrast suggests that the distinguishing property of pluractional predicates might serve as a clue to our understanding what the constraint on the distribution of the perdurative is. There have been proposed various analyses of event pluractionality in formal semantic literature (\citealt{lasersohn1995plurality,van2004adverbials,tovena2010pluractionality,henderson2012ways}, a.o.). Space prevents me from reviewing them in detail here. To put it briefly, the basic idea shared by these works is that a pluractional marker takes an underlying predicate and ensures that there is a multiplicity of atomic events of the same type. Importantly, these atomic events are self-sufficient, they do not stand in any special temporal (like UTA) or causal relation to each other. I capture this property by utilizing \citeposst{landman1992progressive} stage-of relation. 

The notion of a \textsc{stage} has been proposed by \citet{landman1992progressive} to define the semantics of the progressive in English. He claims that the progressive is a function from a set of events denoted by VP to a set of stages of those events. A progressive sentence is true if a VP-event stage develops into an event of the same kind denoted by VP. Stage-of relation is a partial ordering of the set of events. For an event $e$ to be a stage of another event $e′$, $e$ must share enough characteristics with $e′$ and must develop into $e′$ in some possible world that is near enough to the world of evaluation. 

Pluractional predicates are organized in a special way with respect to the stage-of relation: although every atomic event from the extension of the pluractional predicate is a stage of the plural event denoted by this predicate, no atomic event is a stage of any other atomic event. I claim that other predicates allowing the derivation of the perdurative are organized exactly as pluractional predicates with respect to the stage-of relation. I propose that for a predicate to be able to form the perdurative, subevents making up the activity component of this predicate must not be arranged by the stage-of relation. Below I will argue that if a predicate blocks the formation of the perdurative, it possesses the property of \textsc{arrangement by stages} (AbS). A formal definition of AbS is given in \REF{ex:naumov:34}.

\ea \label{ex:naumov:34}
Let $e$ be a partial eventuality from the extension of an event predicate P, and \\
Let $e'$ and $e''$ be stages of $e$ such that
\ea $e' \subseteq e$, and
\ex $e'' \subseteq e$, and\smallskip\\
P($e$) possesses AbS iff \\
$\forall e' \forall e''\,[\,e' \ll_{\cnst{t}} e'' \rightarrow e'$ is a stage of $e'']$\smallskip\\
where $\ll_{\cnst{t}}$ is a temporal precedence relation on events.
\z \z

\noindent \REF{ex:naumov:34} says that an event predicate P is an AbS predicate iff for all contextually salient subevents $e'$ and $e''$ in its denotation such that both $e'$ and $e''$ are stages of $e$, if $e'$ temporally precedes $e''$, then $e'$ is a stage of $e''$. AbS is stronger than incremental homogeneity. I claim that this property is what distinguishes predicates that allow the perdurative from predicates that do not. 

Let us consider the predicate \textit{otkryvat’ zamok} `open a lock' again. Imagine a scenario in which one has a bunch of numbered keys but does not know which one opens the lock. Suppose that a stage of a partial eventuality $e$ from the extension of the predicate \textit{otkryvat’ zamok} `open a lock' is an event of using key number one. Let it be $e'$. Suppose that another stage of $e$ is an event of using key number two. Let it be $e''$. In addition, assume that $e'$ temporally precedes $e''$. Clearly, both $e'$ and $e''$ are stages of the bigger event $e$. The crucial fact is that $e'$ is a stage of $e''$, too. According to the original definition proposed in \citet{landman1992progressive}, ``an event is a stage of another event if the second can be regarded as a more developed version of the first, that is, if we can point at it and say, ``It's the same event in a further stage of development.'' \citep[23]{landman1992progressive}. Although \citet{landman1992progressive} does not provide an explanation for what it is to be ``a more developed version'', intuitively, a more developed event is nothing more than a next step a rational agent takes to achieve the desired goal. Returning to the discussed scenario, if the agent loses the hope of opening the lock with the first two keys, it is very unlikely that she uses key number one again. In case she picks it and performs the same kind of activity as before (without changing the turning direction of the key, the applied force, etc.), this can hardly count as a more developed version of the previous event, which is a sum of using key number one and using key number two, because this activity does not bring the agent closer to her goal—have the lock opened. For an event to fall under the denotation of \textit{otkryvat’ zamok} `open a lock', this event has to consist of subevents such that each subsequent subevent is a more developed version of the previous one. That is, \textit{otkryvat’ zamok} `open a lock' is an AbS predicate. This is the reason why the perdurative is not allowed from it.

As was shown in the previous sections, predicates like \textit{govorit’ po telefonu} `talk on the phone' and \textit{taskat’ čemodan} `carry a suitcase' allow the formation of the perdurative. My proposal correctly accounts for this fact. Let us consider the predicate \textit{taskat’ čemodan} `carry a suitcase'. Indeterminate verbs of motion in Russian can express a wide range of meanings \citep{forsyth1970grammar}. In episodic contexts, the perdurative is licensed when a predicate has the so-called multiple directions reading. This reading arises in a scenario with a single event of motion ``in various unspecified directions'' (\citealt{forsyth1970grammar}: 321). It is easy to show that when \textit{taskat’ čemodan} `carry a suitcase' receives the multiple direction reading, it does not possess AbS. Suppose that a stage of a partial eventuality $e$ from the extension of the predicate \textit{taskat’ čemodan} `carry a suitcase' is an event of moving in one particular direction. Let it be $e'$. Suppose that another stage of $e$ is an event of moving in the other direction. Let it be $e''$. Assume that $e'$ temporally precedes $e''$. Both $e'$ and $e''$ are stages of the bigger event $e$: an event of moving in multiple directions is a more developed version of an event of moving in a single direction. However, an event of moving in one direction can hardly count as a more developed version of an event of moving in the other direction. Moreover, after the agent tried to go in two different directions, she can return to the starting point and take the first direction again. This action will still be in the extension of the predicate \textit{taskat’ čemodan} `carry a suitcase'. For an event to fall under its denotation, it does not need to consist of subevents such that every subsequent subevent is a more developed version of the previous one. The same is true for the predicate \textit{govorit’ po telefonu} `talk on a phone'. 

The perdurative also cannot be derived from predicates associated with an incremental relation between the activity and change of state components. Predicates of this type denote events such that for every activity subevent there is a change of state that it induces \citep{rothstein2004structuring}. An instance of such a predicate is \textit{pisat’ pismo} `write a letter'. When one writes a letter, every part of the writing activity corresponds to some part of the process of being written. The perdurative cannot be derived from this predicate, \REF{ex:naumov:35}.

\ea[*] {\label{ex:naumov:35}
\gll Kolja		pro-pisa-l		pismo	vsjo	utro.\\  
     K.		\textsc{pro}$_\textsc{per}$-write-\textsc{pst}	letter	all	morning\\ 
\glt Intended: `Kolja spent the whole morning writing the letter.'}
\z

\noindent My proposal predicts it because this predicate possesses AbS: an activity of writing a letter consists of subevents such that every subsequent subevent is a more developed version of the previous one. Let me show why. Suppose that a stage of a partial eventuality $e$ from the extension of the predicate \textit{pisat’ pismo} `write a letter' is an event of writing an address. Let it be $e'$. Suppose that another stage of $e$ is an event of writing a salutation. Let it be $e''$. Assume that $e'$ temporally precedes $e''$. Since each of these subevents makes a contribution to the state of being written, at the moment when the agent writes the salutation the letter is already finished to the extent of the address. The subevent of writing the salutation extends the degree to which the letter is finished. Therefore, it counts as a more developed version of the subevent of writing the address. After the salutation is done, the agent can return to the address-part of the letter and, e.g., correct the name of the street or re-write it anew if she realizes that she has made a mistake, and this action will count as a more developed subevent and will be in the extension of the predicate \textit{pisat’ pismo} `write a letter'. What she is very unlikely to do is to perform the same kind of activity and write the address again without any changes. The action of writing the address again will not count as a more developed subevent and will not be in the extension of the predicate \textit{pisat’ pismo} `write a letter'.

\section{Conclusion} \label{sec:naumov:4}

In this study, I have analyzed (morpho)syntactic and semantic constraints on the derivation of the perdurative in Russian. Considering the restrictions that the perdurative prefix \textit{pro-} demonstrates when combined with a number of other affixes, I have argued that this prefix is a selectionally restricted prefix that merges in the functional domain between \textit{v}P and AspP. I have shown that the perdurative \textit{pro-} selects for grammatically imperfective stems and combines with predicates based on secondary imperfective verb stems only when they receive a pluractional interpretation. I have demonstrated that the distribution of the perdurative is also constrained semantically. Having started from the observation that the perdurative can be derived from predicates associated with an activity event structure and having discussed the existing proposals for what it is to be an activity predicate, I concluded that none of them is able to account for its distribution. I have argued that predicates allowing the formation of the perdurative do not possess the AbS property. In other words, activity subevents from their extension do not stand in \citeposst{landman1992progressive} stage-of relation to each other.

Several questions remain. How does the analysis developed here account for stative predicates like \textit{žit’ v Moskve} `live in Moscow', which allow the perdurative? In principle, the proposed analysis predicts the availability of the perdurative from statives because statives do not have stages \citep{landman1992progressive}, and hence there is simply nothing that can be arranged by the stage-of relation. This idea, if correct, needs to be spelled out in more detail. If the perdurative is not responsible for introducing perfectivity, what is its semantic contribution? A potential line to follow, as I see it, is to make use of the notion of maximality, recently much discussed in the literature (see \citealt{altshuler2014,filip2017semantics}, a.o.). The perdurative can be treated as an event modifier extracting a maximal stage (or part) of a partial eventuality. The elaboration of this hypothesis is left for future research.

\section*{Abbreviations}
\begin{tabularx}{.45\textwidth}{lX}
AbS & arrangement by stages\\
\textsc{del}&delimitative\\
LP&lexical prefixes\\
\textsc{per}&perdurative\\
PR & positionally restricted\\
&\\
\end{tabularx}
\begin{tabularx}{.45\textwidth}{lX}
\textsc{pst}&past\\
\textsc{rep}&repetitive\\
SLP&super lexical prefixes\\
SR & selectionally restricted\\
UTA& unique temporal arrangement\\
\end{tabularx}

\section*{Acknowledgments}
The present paper has benefited a lot from comments and questions that I received from two anonymous reviewers as well as from the audience at FDSL 12.5. Many thanks to all these people. I am especially grateful to Sergei Tatevosov whose support and whose own work have influenced this project enormously. This work would have not been completed without Lena Pasalskaya's expertise in \LaTeX. Needless to say that all errors and shortcomings are my own responsibility.



\sloppy
\printbibliography[heading=subbibliography,notkeyword=this]

\end{document}
