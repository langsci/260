\documentclass[output=paper,colorlinks,citecolor=brown,newtxmath]{langsci/langscibook}
\ChapterDOI{10.5281/zenodo.3764857}
%\bibliography{localbibliography}
%% add all extra packages you need to load to this file  
\usepackage{tabularx} 
\definecolor{lsDOIGray}{cmyk}{0,0,0,0.45}

\usepackage{xassoccnt}
\newcounter{realpage}
\DeclareAssociatedCounters{page}{realpage}
\AtBeginDocument{%
  \stepcounter{realpage}
}

%%%%%%%%%%%%%%%%%%%%%%%%%%%%%%%%%%%%%%%%%%%%%%%%%%%%
%%%           Examples                           %%%
%%%%%%%%%%%%%%%%%%%%%%%%%%%%%%%%%%%%%%%%%%%%%%%%%%%%  
%% if you want the source line of examples to be in italics, uncomment the following line
% \renewcommand{\exfont}{\itshape}
\usepackage{lipsum}
\usepackage{langsci-optional}
\usepackage{./langsci-osl}
\usepackage{langsci-lgr}
\usepackage{langsci-gb4e}
\usepackage{stmaryrd}
\usepackage{pifont} % needed for checkmark \ding{51} and cross \ding{55}
\usepackage[linguistics]{forest}

% ch06
\usepackage[euler]{textgreek}
% ch07
\usepackage{soul}
\usepackage{graphicx}
% ch08, ch14
\usepackage{multicol}
% ch11
\usepackage{fnpct}
% ch13
\usepackage{scrextend}
\usepackage{enumitem}
% ch14
\usepackage{tabto}
\usepackage{multirow}

\usepackage{langsci-cgloss}

%\newcommand{\smiley}{ :) }

% non-italics in examples

\renewcommand{\eachwordone}{\upshape}

% non-italics in examples in footnotes

\renewcommand{\fnexfont}{\footnotesize\upshape}
\renewcommand{\fnglossfont}{\footnotesize\upshape}
\renewcommand{\fntransfont}{\footnotesize\upshape}
\renewcommand{\fnexnrfont}{\fnexfont\upshape}

% chapter03 goncharov

\newcommand{\p}{\textsc{pfv\ }}
\newcommand{\im}{\textsc{ipfv\ }}

\makeatletter
\let\thetitle\@title
\let\theauthor\@author 
\makeatother

\newcommand{\togglepaper}[1][0]{  
  \addbibresource{../localbibliography.bib}  
  \papernote{\scriptsize\normalfont
    \theauthor.
    \thetitle. 
    To appear in: 
    Change Volume Editor \& in localcommands.tex 
    Change volume title in localcommands.tex
    Berlin: Language Science Press. [preliminary page numbering]
  }
  \pagenumbering{roman}
  \setcounter{chapter}{#1}
  \addtocounter{chapter}{-1}
}

\providecommand{\orcid}[1]{}
\IfFileExists{../localcommands.tex}{
  % add all extra packages you need to load to this file  
\usepackage{tabularx} 
\definecolor{lsDOIGray}{cmyk}{0,0,0,0.45}

\usepackage{xassoccnt}
\newcounter{realpage}
\DeclareAssociatedCounters{page}{realpage}
\AtBeginDocument{%
  \stepcounter{realpage}
}

%%%%%%%%%%%%%%%%%%%%%%%%%%%%%%%%%%%%%%%%%%%%%%%%%%%%
%%%           Examples                           %%%
%%%%%%%%%%%%%%%%%%%%%%%%%%%%%%%%%%%%%%%%%%%%%%%%%%%%  
%% if you want the source line of examples to be in italics, uncomment the following line
% \renewcommand{\exfont}{\itshape}
\usepackage{lipsum}
\usepackage{langsci-optional}
\usepackage{./langsci-osl}
\usepackage{langsci-lgr}
\usepackage{langsci-gb4e}
\usepackage{stmaryrd}
\usepackage{pifont} % needed for checkmark \ding{51} and cross \ding{55}
\usepackage[linguistics]{forest}

% ch06
\usepackage[euler]{textgreek}
% ch07
\usepackage{soul}
\usepackage{graphicx}
% ch08, ch14
\usepackage{multicol}
% ch11
\usepackage{fnpct}
% ch13
\usepackage{scrextend}
\usepackage{enumitem}
% ch14
\usepackage{tabto}
\usepackage{multirow}

\usepackage{langsci-cgloss}

  \newcommand{\smiley}{ :) }

% non-italics in examples

\renewcommand{\eachwordone}{\upshape}

% non-italics in examples in footnotes

\renewcommand{\fnexfont}{\footnotesize\upshape}
\renewcommand{\fnglossfont}{\footnotesize\upshape}
\renewcommand{\fntransfont}{\footnotesize\upshape}
\renewcommand{\fnexnrfont}{\fnexfont\upshape}

% chapter03 goncharov

\newcommand{\p}{\textsc{pfv\ }}
\newcommand{\im}{\textsc{ipfv\ }}

\makeatletter
\let\thetitle\@title
\let\theauthor\@author 
\makeatother

\newcommand{\togglepaper}[1][0]{  
  \addbibresource{../localbibliography.bib}  
  \papernote{\scriptsize\normalfont
    \theauthor.
    \thetitle. 
    To appear in: 
    Change Volume Editor \& in localcommands.tex 
    Change volume title in localcommands.tex
    Berlin: Language Science Press. [preliminary page numbering]
  }
  \pagenumbering{roman}
  \setcounter{chapter}{#1}
  \addtocounter{chapter}{-1}
}

\providecommand{\orcid}[1]{}
  %% hyphenation points for line breaks
%% Normally, automatic hyphenation in LaTeX is very good
%% If a word is mis-hyphenated, add it to this file
%%
%% add information to TeX file before \begin{document} with:
%% %% hyphenation points for line breaks
%% Normally, automatic hyphenation in LaTeX is very good
%% If a word is mis-hyphenated, add it to this file
%%
%% add information to TeX file before \begin{document} with:
%% %% hyphenation points for line breaks
%% Normally, automatic hyphenation in LaTeX is very good
%% If a word is mis-hyphenated, add it to this file
%%
%% add information to TeX file before \begin{document} with:
%% \include{localhyphenation}
\hyphenation{
Ro-ma-no-va
Isa-čen-ko
}

\hyphenation{
Ro-ma-no-va
Isa-čen-ko
}

\hyphenation{
Ro-ma-no-va
Isa-čen-ko
}

  \togglepaper[8]%%chapternumber
}{}
%\togglepaper[8]


%\usepackage{qtree}

% \usepackage{tikz}

% \usepackage{forest}
\forestset{sn edges/.style={
for tree={
parent anchor=south, child anchor=north,
calign=fixed edge angles,
calign primary angle=-60, calign secondary angle=60,
align=center, base=t},
delay={
where content={}{shape=coordinate,
for parent={for children={anchor=north}}}{}}
}}
\forestset{
	roof/.style={edge path'={%
	(.parent first)--(!u.children)--(.parent last)--cycle
	}
	},
	}

\title{How to introduce instrumental agents: Evidence from binding in Russian event nominal phrases}

\author{ Takuya Miyauchi\affiliation{Tokyo University of Foreign Studies and\\Japan Society for the Promotion of Science}}

\abstract{The aim of this paper is to argue that instrumental agents in Russian are introduced by a silent P through binding phenomena by instrumental agents in event nominal phrases. Two assumptions are adopted in this paper: one is the absence of the DP layer in Russian based on binding phenomena and the other is a particular structure of event nominal phrases. I show that the appropriateness of proposing a silent P is supported by Generalized Case Realization Requirement in Russian and that the silent P is a lexical preposition, not a functional one due to its ability to bind objects out of PP.

\keywords{Russian, event nominal phrases, DP--NP, instrumental agents}
}



\begin{document}%
\maketitle
\shorttitlerunninghead{How to introduce instrumental agents}
\il{Russian|(}
\section{Introduction}
In this paper, I claim that instrumental agents in \ili{Russian} are introduced by a silent P (ø), through binding phenomena by instrumental agents in \isi{event nominal} phrases. Two assumptions are adopted here: one is the structure of \isi{event nominal phrase}s proposed by \cite{MiyauchiIto2016} and \cite{Miyauchi2017b}, and the other is that \ili{Russian} nominal phrases are not DP but NP. I discuss that \textsc{Generalized Case Realization Requirement} (\isi{GCRR}) in \ili{Serbo-Croatian} \citep{Horvath2014} can apply (at least partially) to \ili{Russian} and show that setting a silent P is appropriate via this discussion. Finally, I point out a possibility to bind objects out of PP and demonstrate that the silent P is a lexical preposition, not a functional one.


The rest of this paper is organized as follows. In \sectref{sec:BIND}, I outline discussion on the structure of nominal phrases through  \ili{Russian} binding phenomena. \sectref{sec:EVN} offers a syntactic account on a restriction of θ-roles of genitive nouns in \isi{event} nominal phrases with some assumptions, based on \cite{MiyauchiIto2016} and \cite{Miyauchi2017b}. In \sectref{sec:INS}, I propose the syntactic structure of \isi{event} nominal phrases containing instrumental agents and \sectref{sec:GCRR} shows the validity of \isi{GCRR} in \ili{Russian}. \sectref{sec:PP} points out that there are binding phenomena out of PP in \ili{Russian} and the proposed structure is modified. Finally, \sectref{sec:CON} concludes the paper.

\section{Russian binding phenomena and the structure of nominal phrases}\label{sec:BIND}

The structure of \ili{Slavic} nominal phrases without overt articles is controversial in terms of whether they have DP in their structure or not. Some researchers insist on the presence of DP even in \ili{Slavic} (\citealt{Progovac1998,Rappaport2002,Rutkowski2002,Basic2004,FranksPereltsvaig2004,Pereltsvaig2007a,RutokowskiMaliszewska2007,Laterza2016}, etc.), while others maintain that nominal phrases in \ili{Slavic} are NPs (\citealt{Zlatic1998,Trenkic2004,Boskovic2005,Boskovic2007dpnp,Boskovic2009,Petrovic2011,despic2013}, etc.).

In this paper, I investigate instrumental agents in \isi{event} nominal phrases in \ili{Russian} from the standpoint of the no-DP analysis.\footnote{Note that the NP/DP debate itself is beyond the scope of this paper. Please see the references cited above for arguments for and against the DP projection in \ili{Slavic}.}
In this section, I outline the discussion of the structure of nominal phrases through \ili{Russian} binding phenomena, which gives support to the no-DP analysis, based on \citeposst{despic2013} paradigm.\footnote{The content of this section is based on \citet{Miyauchi2016}. Please see \citet{Miyauchi2016} for more details.}
\citet{despic2013} argues that binding phenomena and \citeposst{kayne1994} antisymmetry approach provide a key to examine the existence or absence of the DP projection. He concludes that there is no DP in \ili{Serbo-Croatian} and that the D-like elements are adjuncts.

Following \citet{despic2013}, I adopt \citeposst{kayne1994} definition of c-command given in \REF{def}.\footnote{I use this definition of c-command henceforth.}

\begin{exe}
\ex\label{def}
X c-commands Y iff X and Y are categories, X excludes Y and every category that 	dominates X dominates Y.
\hfill \citep[16]{kayne1994}
\end{exe}

\noindent
The definition of exclusion is as follows \REF{exc}:

\begin{exe}
\ex\label{exc} α excludes β if no segment of α dominates β.
\hfill \citep[9]{Chomsky1986barriers}
\end{exe}

\noindent
The \ili{Russian} sentences \REF{binr}, \REF{binr2}, and \REF{binr3} are ungrammatical with co-reference between possessors and pronouns, while \REF{binrpos}, \REF{binrpos2}, and \REF{binrpos3} are grammatical. There is a clear contrast between \REF{binr}, \REF{binr2}, \REF{binr3} and \REF{binrpos}, \REF{binrpos2}, \REF{binrpos3}.

\begin{exe}\ex\label{bin}
\begin{xlist}
\ex[*]{ \label{binr}
\gll Kolin$_i$ poslednij fil'm sil'no ego$_i$ razočaroval. \\
	Kolya's latest film really him disappointed\\
\trans Intended: `Kolya$_i$'s latest film really disappointed him$_i$.'}
\ex[]{ \label{binrpos}
\gll Poslednij fil'm Koli$_i$ sil'no ego$_i$ razočaroval. \\
	latest film Kolya.{\GEN} really him disappointed\\
\trans `The latest film of Kolya$_i$ really disappointed him$_i$.'}
\end{xlist}
\end{exe}

\begin{exe}\ex\label{bind2}
\begin{xlist}
\ex[*]{ \label{binr2}
\gll Vanin$_i$ papugaj ukusil ego$_i$ včera.\\
     Vanya's parrot bit him yesterday\\
\trans Intended: `Vanya$_i$'s parrot bit him$_i$ yesterday.'}
\ex[]{ \label{binrpos2}
\gll Papugaj Vanin$_i$ ukusil ego$_i$ včera.\\
      parrot Vanya.{\GEN} bit him yesterday\\
\trans `The parrot of Vanya$_i$ bit him$_i$ yesterday.'}
%\trans `Vanya$_i$'s parrot bit him$_i$ yesterday.'
\end{xlist}
\end{exe}
\begin{exe}\ex\label{bind3}
\begin{xlist}
\ex[*]{ \label{binr3}
\gll Sašin$_i$  mjač včera udaril ego$_i$ po golove. \\
     Sasha's ball yesterday hit him in head\\
\trans Intended: `Sasha$_i$'s ball hit him$_i$ in the head yesterday.'}
\ex[]{ \label{binrpos3}
\gll Mjač Saši$_i$ včera udaril ego$_i$ po golove. \\
     ball Sasha.{\GEN} yesterday hit him in head\\
\trans `The ball of Sasha$_i$ hit him$_i$ in the head yesterday.'}
%\trans `Sasha$_i$'s ball hit him$_i$ in the head yesterday.'
\end{xlist}
\end{exe}

\noindent
Let us see \REF{bin} as a representative case.
Following \citet{despic2013}, I argue that \REF{binr} is ungrammatical because the \isi{possessor} \textit{Kolin} `Kolya's' binds the co-indexed \isi{pronoun} \textit{ego} 'him,' which results in Condition B violation.
According to the reasoning in \citet{despic2013}, this suggests that \ili{Russian} nominal phrases lack the DP layer.\footnote{An anonymous reviewer argues that the \ili{Serbo-Croatian} data in \citet{despic2013} corresponding to the examples \REF{bin}--\REF{bind3} are not ungrammatical, which suggest \citeposst{despic2013} conclusion about the presence/absence of the DP layer in Serbo-\ili{Croatian} is questionable, but as I am only concerned with \ili{Russian} here, I do not have much to add. I am only employing the reasoning and the structure of the argument developed in \cite{despic2013}. My argument for the lack of DP in \ili{Russian} is thus valid regardless of the quality of \citeposst{despic2013} Serbo-\ili{Croatian} data.} \figref{bindtreerus} shows the structure of \REF{binr}.\footnote{The object \textit{ego} `him' in  \REF{binr} is scrambled and the word order of this sentence becomes SOV. For the sake of simplicity, however, scrambling is ignored in \figref{bindtreerus}. I take the basic word order in \ili{Russian} to be SVO, following \citet{Isachenko1966}.}
Note that under \citeposst{kayne1994} theory, specifiers are taken to be adjoined phrases. Consequently, specifiers are not distinguished from adjuncts and the bar-level notation does not make sense.\footnote{I do not use the bar-level notation. The conventional X$'$ (X-bar/X-prime) is written as XP or is omitted in the trees in this paper.}

\begin{figure}[h]
\caption{The structure of \REF{binr}}
\label{bindtreerus}
\begin{forest}
  sn edges [ CP [ C ]
                [ TP$_1$ [ NP$_1$ [ Kolya$_i$'s ]
                                  [ NP$_2$ [ AP [ latest, roof] ]
                                           [ N [film ] ] ] ]
                         [ TP$_2$ [ T ]
                                  [ VP [ really ]
                                       [ [ V [ disappointed ] ]
                                         [ NP [ him$_i$, roof] ] ] ] ] ] ]
\end{forest}
\end{figure}

In \figref{bindtreerus}, the first category dominating `Kolya's' is CP, which also dominates the object NP `him.'\footnote{Note that NP$_1$ and TP$_1$ are segments not categories.} Therefore, the \isi{possessor} `Kolya's' c-commands the object `him', violating Condition B.
Accordingly, \REF{binr} is ungrammatical.
If there was an additional DP layer in the nominal phrase, as illustrated in \figref{rusd}, the \isi{possessor} `Kolya's' would not c-command the \isi{pronoun} and there would thus be no Condition B violation. \REF{binr} should be grammatical, contrary to fact.

\begin{figure}[h]
\caption{The structure of the subject of \REF{binr} with a DP layer}
\label{rusd}
\begin{forest}
  sn edges [ DP [ D ]
                [ NP$_1$ [ Kolya$_i$'s ]
                         [ NP$_2$ [ AP [ latest, roof] ]
                                  [ N [ film ] ] ] ] ]
\end{forest}
\end{figure}

In \figref{rusd}, `Kolya's' is dominated by the category DP, which does not dominate any other node outside of the nonimal phrase. That is to say, the \isi{possessor} `Kolya's', does not c-command the object NP. For this reason, with the DP layer, there would be no violation of Condition B in sentences like \REF{binr} and these sentences should be grammatical. Thus, it is concluded that the ungrammaticality of \REF{binr} shows that there is no DP layer in \ili{Russian}.

\newpage
\largerpage
How can we capture the grammaticality of \REF{binrpos}?
Generally, the genitive \isi{possessor} NP is supposed to be located in the complement of the (head) NP (\citealt[38]{franks1995}; \citealt[214]{bailyn2012}; \citealt[84]{Mitrenina2012}).\footnote{To be precise, \citet{bailyn2012} does not propose this structure.	According to \citet{bailyn2012}, adnominal genitives occupy the complement of QP in \REF{fni}:

			\ea\label{fni}
			\relax[$_\textrm{NP}$ N [$_\textrm{QP}$ Q [\textsubscript{NP\textsubscript{\textsc{gen}}}\ \ ] ] ]
			\hfill (\citealt[214]{bailyn2012}; slightly modified by the author)
			\z

			\noindent
			\citet[214]{bailyn2012} proposes that Q assigns the \isi{genitive case} to the sister NP (there is a case where Q is covert). However, these differences in the positions of genitive NP	have no effect on the argument of this paper since genitive \isi{possessor} NP is located lower than possessee NP.}
\figref{bindtreeGEN} represents the structure of \REF{binrpos}.

\begin{figure}[h]
\caption{The structure of \REF{binrpos}}
\label{bindtreeGEN}
\begin{forest}
  sn edges [ CP [ C ]
                [ TP$_1$ [ NP$_1$ [ AP [ latest, roof] ]
                                  [ NP$_2$ [ N [film ] ]
                                           [ NP\textsubscript{\textsc{gen}} [ Kolya.\textsc{gen}$_i$, roof] ] ] ]
                         [ TP$_2$ [ T ]
                                  [ VP [ really ]
                                       [ [ V [ disappointed ] ]
                                         [ NP [ him$_i$, roof] ] ] ] ]
                ] ]
\end{forest}
\end{figure}


In this case, the categories dominating NP$_{\textsc{gen}}$ are NP$_1$ and CP. NP$_{\textsc{gen}}$ does not c-command the object NP because the subject NP$_1$ does not dominate the object NP. Thus, there is no Condition B violation and sentences like \REF{binrpos} are grammatical.

The contrast in grammaticality between prenominal \isi{possessive} constructions \REF{binr}, \REF{binr2}, \REF{binr3} and postnominal ones \REF{binrpos}, \REF{binrpos2}, \REF{binrpos3} supports the argument that the DP is not projected in \ili{Russian}.
For the rest of the paper, I adopt the no-DP analysis of \ili{Russian} nominal phrases.


\section[Russian event nominal phrases and their structures]{{Russian} {event} nominal phrases and their structures\label{sec:EVN}}

\subsection{Russian event nominals}

An ``\isi{event} nominal'' denotes an event or process and inherits \isi{argument structure} of its base \isi{verb} (\citealt{Grimshaw1990} in general, \citealt{Schoorlemmer1998}, \citealt{Pazelskaya2007} for \ili{Russian}).\footnote{The content of this section is mostly based on \cite{MiyauchiIto2016} and \cite{Miyauchi2017b}. Please see \cite{MiyauchiIto2016} and \cite{Miyauchi2017b} for more details.}

It can be followed by a genitive complement in \ili{Russian}.

\begin{exe}\ex \label{type}%Three types of \isi{event} nominals
\begin{xlist}
\ex \label{type1}
 {Type 1:} \ding{51} \isi{external argument} / \ding{55}\ \isi{internal argument}\\\samepage
%%$^{\checkmark}${ext}.arg. / *{int}.arg. → Type1
\gll	udar \minsp{\{} {mužčiny} / \minsp{*} {stola}\}\\
		hit {} man.{\GEN} {} {} table.{\GEN}\\
\trans `the hit of \{the man / the table\}'

\largerpage
\ex \label{type2}
 {Type 2:} \ding{51} \isi{external argument} / \ding{51} \isi{internal argument}\\
\gll	ispolnenie \minsp{\{} {Šaljapina} / {arii}\}\\
		performance {} Chaliapin.{\GEN} {} aria.{\GEN}\\
\trans `the performanc of \{Chaliapin / the aria\}'\samepage

%\newpage
\ex \label{type3}
 {Type 3:} \ding{55}\ \isi{external argument} / \ding{51} \isi{internal argument}\\
\gll	razrušenie \minsp{\{*} {vraga} / {goroda}\}\\
		destruction {} enemy.{\GEN} {} city.{\GEN}\\
\trans `the destruction of \{the enemy / the city\}'
\end{xlist}
\end{exe}

\newpage
\noindent This kind of restriction of genitive complements' θ-roles  is thought to result from argument structures of \isi{event} nominals \citep{Pazelskaya2007}.
Therefore these phenomena have been dealt with as lexical problems.
\cite{MiyauchiIto2016} and \cite{Miyauchi2017b} tried to provide more principled explanation for these phenomena as syntactic problems based on \isi{phase theory}.

I adopt the framework of \textsc{Distributed Morphology} \citep{HalleMarantz1993}, in which $\surd$ (root) moves to a categorizer (\textit{n, v, a}, etc.) to determine its category.
The contention of \cite{MiyauchiIto2016} and \cite{Miyauchi2017b} is that type 1 nominals and type 2/3 nominals differ structurally.
I adopt the structure in \figref{type23tr} for all \isi{event} nominals and explain below how the two types differ.

\begin{figure}[H]
\caption{The structure of the event nominal phrases}
\label{type23tr}
\begin{forest}
  sn edges [ (\isi{VoiceP}) [ (\isi{Voice}) [ -\textit{nie}/-\textit{tie} ] ]
                      [ XP [ X ]
                           [ \textit{n}P [ \textit{n} ]
                                       [ $\surd$P [ $\surd$ ]
                                                  [ (\textsc{int-arg}) ] ] ] ] ]
\end{forest}
\end{figure}

\largerpage[2]
\noindent
Unlike in type 2/3, which project the entire structure, \isi{VoiceP} is not projected in type 1.
This structural difference is supported by
the absence of a verbal nominalizer \textit{-nie/-tie} in type 1 nominals.\footnote{This argument assumes that the nominalizer occupies the head of \isi{VoiceP}. Support for this assumption comes from the fact that the nominalizer is morphologically complex and seems to include the \isi{passive participle} morpheme \textit{-n-/-t-} \citep{Babby1997}.}
I suppose that $\surd$ directly takes an \isi{internal argument} following \citet{Harley2009a}.
Moreover, a \isi{functional head}, X, licenses genitive Case through Agree.\footnote{This X is a counterpart of Num(ber) in \citet{Carstens2001}, which is claimed to be a licenser of Case. \citet{bailyn2012}, on the other hand, argues that the \isi{genitive case} assigner in \ili{Russian} is Q. The true identity of X lies outside the scope of this paper.}

I assume \citeposst{Chomsky2000} \isi{phase theory} and the proposal that \textit{n}P is a phase (\citealt{Carstens2001, Arad2003, hicks2009}, etc.). It then follows from \textsc{Phase Impenetrability Condition}, shown in \REF{PIC}, that internal arguments cannot be genitive in type 1 nominals.


\begin{exe}
\ex\label{PIC}  {Phase Impenetrability Condition (\isi{PIC})}\\
In phase α with head H, the domain of H is not accessible to
operations outside α, only H and its edge are accessible to such operations.\\
%{} \hfill \citep[108]{Chomsky2000}
\end{exe}
\vspace{-2em}
\hfill \citep[108]{Chomsky2000}
\vspace{1em}

\noindent In type 2/3 nominals, the head of $\surd$P moves to the nominalizer \textit{-nie/-tie}, the head of \isi{VoiceP}, in order to derive its form.
Therefore, the \textit{n}P phase slides up to \isi{VoiceP} due to \textsc{phase-sliding} \citep{Gallego2010}.\footnote{According to phase-sliding, when a \isi{verb} head-moves from \textit{v}$^*$ to T, the phasehood of \textit{v}$^*$ also moves to T. I can generalize this proposal: when X, a phase, head-moves to Y,  the phasehood of X also moves to Y. Thus, phase-sliding can be applied to \isi{event} nominals. In this example, $\surd$ moves to \isi{VoiceP} stopping at \textit{n} and X, picking them up along the way because of the head movement constraint \citep{Travis1984, Matushansky2006}. Since \textit{n} is a phase head, phase-sliding also occurs.} The size of the new phase is shown with a box in \figref{type23trre}.

\begin{figure}[h]
\caption{The structure of event nominal phrases with phase-sliding}
\label{type23trre}
%\Tree [.\isi{VoiceP} [.\isi{Voice} {-\textit{nie} / -\textit{tie}} ] [.XP [.X ] [.\textit{n}P [.\textit{n} ] [.$\surd$P [.$\surd$ ] \textsc{int-arg} ] ] ]!{\qframesubtree} ]
%[_{DP} D [_{VoiceP} \fbox{[_{XP} [_{nP} [_{\surd P} Patient $\surd$ ] \textit{n} ] X ] \isi{Voice} ] ]}
\begin{forest}
  sn edges [ \isi{VoiceP} [ \isi{Voice} [ -\textit{nie}/-\textit{tie} ] ]
                    [ XP, tikz={\node [draw,fit to=tree]{};}
                         [ X ]
                         [ \textit{n}P [ \textit{n} ]
                                     [ $\surd$P [ $\surd$ ]
                                                [ (\textsc{int-arg}) ] ] ] ] ]
\end{forest}
\end{figure}

Thus, phase-sliding makes it possible that X Agrees with an \isi{internal argument}, not violating \isi{PIC}. Consequently, internal arguments are allowed to have genitive Case.


\subsection{Genitive external arguments}

To avoid wrong prediction, I suppose two external θ-roles: \isi{possessor} (\textsc{Poss}) and agent (\textsc{Ag}).
In type 1/2, \textsc{Poss} is merged in the specifier of \textit{n}P (\citealt{Carstens2000, Carstens2001, Adger2003}, etc.).
By contrast, \textsc{Ag} in the type 3 is adjoined to \isi{VoiceP} like \textit{by} phrases in \ili{English} \citep{Bruening2013}.\footnote{In fact, a distinction between specifiers and adjuncts cannot be made under \citeposst{kayne1994} theory. What is significant here, however, is the structural difference between \textsc{Poss} and \textsc{Ag}. That is, \textsc{Poss} is located in the \textit{n}P domain, while \textsc{Ag} is in the \isi{VoiceP} domain.}
\textsc{Poss} is c-commanded by the probe, X and thus
X can Agree with it as illustrated in \figref{external}.
Consequently, external arguments in type 1/2 can be genitive at the postnominal position.
On the other hand, \textsc{Ag} is not c-commanded by the probe X  and hence X cannot Agree with it as schematized in \figref{external}.
This is why external arguments in type 3 cannot appear in \isi{genitive case} postnominally.\footnote{In addition to these structures of \isi{event} nominal phrases,
			\citet{Miyauchi2017b} suggests
			that there are two types on \textit{n}P through semantic analyses. Please see \citet[section 5]{Miyauchi2017b} for more details.}

\begin{figure}[h]
\caption{The structure of event nominal phrases with external θ-roles}
\label{external}
%\Tree [.{(\isi{VoiceP})} [ [.{(\isi{Voice})} {-\textit{nie}/-\textit{tie}} ] [.XP [.X ] [.\textit{n}P [ [.\textit{n} ]!\qsetw{.5in} \qroof{{\ \ \ }$\surd${\ \ \ }}.$\surd$P ]!\qsetw{1in} \textsc{Poss} ]!\qsetw{1in} ] ]!\qsetw{1.5in} \textsc{Ag} ]
%[.{(\isi{VoiceP})} [ [.{(\isi{Voice})} {-\textit{nie}/-\textit{tie}} ] [.XP [.X ] [.\textit{n}P [ [.\textit{n} ] [.$\surd$P [.$\surd$ ]!\qsetw{0.5in} {(\textsc{int-arg})} ] ]!\qsetw{1in} \textsc{Poss} ]!\qsetw{1in} ] ]!\qsetw{1.5in} \textsc{Ag} ]
\begin{forest}
  sn edges [ (\isi{VoiceP}) [ [ (\isi{Voice}) [ -\textit{nie}/-\textit{tie} ] ]
                        [ XP [ X ]
                             [ \textit{n}P [ [ \textit{n} ]
                                           [ $\surd$P [ {\ \ $\surd$\ \ }, roof] ] ]
                                         [ \textsc{Poss} ] ] ] ]
                       [ \textsc{Ag} ]  ]
\end{forest}
\end{figure}

\section{Instrumental agents introduced by silent P}\label{sec:INS}

\subsection{Russian event nominals with instrumental agents}


Double genitives are basically banned in \ili{Russian} as shown in examples \REF{ispGG}, \REF{razGG}, where type 2 and type 3 nominals are used.
When both internal and external arguments are expressed in the same phrase, the former is assigned the \isi{genitive case} and the latter is assigned the \isi{instrumental case} as indicated in \REF{ispGI}, \REF{razGI}.

\begin{exe}
\ex\label{ispG}  {Type 2}
\begin{xlist}
\ex[*]{\label{ispGG}
\gll ispolnenie arii Šaljapina\\
	performance aria.{\GEN} Chaliapin.{\GEN}\\
\trans Intended: `the performance of aria by Chaliapin'} \hfill\citep[162]{Ljutikova2016}
\ex[]{\label{ispGI}
\gll ispolnenie arii Šaljapinym\\
	performance aria.{\GEN} Chaliapin.{\INS}\\
\trans `the performance of aria by Chaliapin' \hfill\citep[64]{Gerasimova2016}}
\end{xlist}
\end{exe}

\begin{exe}
\ex\label{razG}  {Type 3}
\begin{xlist}
\ex[*]{\label{razGG}
\gll razrušenie goroda vraga\\
	destruction city.{\GEN} enemy.{\GEN}\\
\trans Intended: `the destruction of the city by the enemy' }

%\newpage
\ex[]{\label{razGI}
\gll razrušenie goroda vragom\\
	destruction city.{\GEN} enemy.{\INS}\\\samepage
\trans `the destruction of the city by the enemy' }
\end{xlist}
\end{exe}

\noindent In \ili{Serbo-Croatian}, if the agent nominal is a complex phrase, the double genitive construction is perfectly acceptable as shown in \REF{dgsc}.\footnote{
			This was kindly pointed out by an anonymous reviewer.}

\begin{exe}
\ex\label{dgsc} %Serbo-\ili{Croatian}\vspace{-1ex}
\gll snimak požara Emira Kusturice\\
	record fire.{\GEN} Emir.{\GEN} Kusturica.{\GEN}\\
\trans `the shot of the fire by Emir Kustiruca'
\end{exe}

\noindent
However, in \ili{Russian}, even if agents are complex, double genitives are not permitted as indicated in \REF{dgr}.\footnote{
			Although the micro-variation between \ili{Serbo-Croatian} \REF{dgsc} and \ili{Russian} \REF{dgr} is significant, this paper cannot address this contrast since it focuses on only \ili{Russian}.}

\begin{exe}
\ex\label{dgr} %\ili{Russian}\samepage
\begin{xlist}
\ex[*]{
\gll s'emka požara Ivana Andreeviča\\
	shot fire.{\GEN}	Ivan.{\GEN} Andreevich.{\GEN}\\
\trans Intended: `the shot of fire by Ivan Andreevich'}
\ex[*]{
\gll ispolnenie arii izvestnogo	 pevca Šaljapina\\
	performance aria.{\GEN} famous.{\GEN} singer.{\GEN} Chaliapin.{\GEN}\\
\trans Intended: `the performance of aria by the famous singer, Chaliapin' }
\end{xlist}
\end{exe}

\noindent
Of course, as is the case with \REF{ispG}, \REF{razG}, the phrases are acceptable when the complex agents are instrumental as show in \REF{gicom}.

\begin{exe}
\ex\label{gicom} %\ili{Russian}
\begin{xlist}
\ex
\gll s'emka požara Ivanom Andreevičem\\
	shot fire.{\GEN}	Ivan.{\INS} Andreevich.{\INS}\\
\trans  `the shot of fire by Ivan Andreevich'
\ex
\gll ispolnenie arii izvestnym	 pevcom Šaljapinym\\
	performance aria.{\GEN} famous.{\INS} singer.{\INS} Chaliapin.{\INS}\\
\trans `the performance of aria by the famous singer, Chaliapin'
\end{xlist}
\end{exe}

\noindent Instrumental phrases can appear as agents of \textsc{complex \isi{event} nominals} (\isi{CEN}), which have argument structures.
\isi{CEN} obligatorily take internal arguments and overtly express them (\citealt{Revzin1973, Schoorlemmer1998}, etc.).
That is, an instrumental agent cannot appear without an \isi{internal argument} unless it is elided as illustrated in \REF{II}--\REF{DI}.\footnote{
			$\Delta$ in \REF{DI} expresses a deleted \isi{internal argument}.			}

\begin{exe}
\ex\label{II}
\begin{xlist}
\ex[*]{\label{ispI}
\gll ispolnenie Šaljapinym\\
	performance Chaliapin.{\INS}\\
\trans Intended: `the performance by Chaliapin' \hfill\citep[90]{Revzin1973}}
\ex[*]{\label{razI}
\gll razrušenie vragom\\
	destruction enemy.{\INS}\\
\trans Intended: `the destruction by the enemy' }
\end{xlist}
\end{exe}

\begin{exe}
\ex\label{DI}
\begin{xlist}
\ex\label{ispdI}
\gll ispolnenie $\Delta$ Šaljapinym \\  %%$\partial$
	performance {} Chaliapin.{\INS}\\
\trans `the performance by Chaliapin of ...'

\ex\label{razdI}
\gll razrušenie $\Delta$ vragom\\
	destruction {} enemy.{\INS}\\
\trans `the destruction by the enemy of ...'
\end{xlist}
\end{exe}

\noindent
The type 2 and 3 nominals have \isi{VoiceP} as mentioned in \sectref{sec:EVN}.
If I assume that instrumental \isi{agentive} phrases are located at a domain related with \isi{VoiceP}, it is natural that the type 1 nominals cannot have  instrumental \isi{agentive} phrases because of the absence of \isi{VoiceP}.
This is reflected in the ungrammaticality of \REF{ud}.

\newpage
\begin{exe}
\ex\label{ud}  {Type 1}
\begin{xlist}
\ex[*]{\label{udGI}
\gll	udar {stola} mužčinoj\\
		hit table.{\GEN} man.{\INS}\\
\trans Intended: `the hit the table by the man'}

\ex[*]{\label{udI}
\gll udar mužčinoj\\
	hit man.{\INS}\\
\trans Intended: `the hit by the man'}
\end{xlist}
\end{exe}


\subsection{Binding in Russian event nominals and instrumental agents as VoiceP specifiers/adjuncts}

Contrast in binding similar to the one presented in \REF{bin}--\REF{bind3} is observed also with instrumental agents in \isi{event} nominal phrases, as shown in \REF{binins} and \REF{binins2}.\footnote{
			An anonymous reviewer pointed out that in \ili{Serbo-Croatian} the postnominal doubled genitive can co-refer a personal \isi{pronoun} as shown in \REF{fnii} below.

\ea[]{\label{fnii}
\gll Snimak požara \minsp{[} Emira Kusturice]$_i$ napravio je od njega$_i$ reportersku zvezdu.\\
      record fire.{\GEN} {} Emir.{\GEN} Kusturica.{\GEN} made {\AUX} from him reporter star\\
\glt `The shot of the fire by [Emir Kusturica]$_i$ made him$_i$  star reporter.'}
\z

            \noindent Unlike \ili{Serbo-Croatian}, \ili{Russian} bans doubled genitives as indicated in \REF{dgr}, so I have nothing further to say on this topic.}

\begin{exe}
\ex\label{binins}
\begin{xlist}
\ex[*]{\label{contPOSS}
\gll  {Ivanovo}$_i$  narušenie  {pravil}  ogorčaet  ego$_i$.\\
	Ivan's violation rules.{\GEN} distresses him\\
\trans Intended: `Ivan$_i$'s violation of the rules distresses him$_i$.'}
\ex[]{\label{contINS}
\gll  Narušenie  {pravil}  {Ivanom}$_i$  ogorčaet  ego$_i$.\\
%	violation rule.{\GEN} I.{\INS} distresses him\\
	violation rules.{\GEN} Ivan.{\INS} distresses him\\
\trans `The violation of the rules by Ivan$_i$ distresses him$_i$.'}
\end{xlist}
\end{exe}

\begin{exe}
\ex\label{binins2}
\begin{xlist}
\ex[*]{\label{contPOSS2}
\gll Ivanovo$_i$  ubijstvo  Viti gluboko opečalilo ego$_i$.\\
	Ivan's murder Vitya.{\GEN} deeply saddened him\\
\trans Intended: `Ivan$_i$'s murder of Vitya deeply saddened him$_i$.'}
%	{} Ivan's murder Vitya-GEN distresses him\\
%\trans `The Ivan$_i$'s murder of Vitya distresses him$_i$'
\ex[]{\label{contINS2}
\gll  Ubijstvo  Viti  Ivanom$_i$ gluboko opečalilo ego$_i$.\\
	murder Vitya.{\GEN} Ivan.{\INS} deeply saddened him\\
\trans `The murder of Vitya by Ivan$_i$ deeply saddened him$_i$.'}
%\gll  Ubijstvo  Viti$_i$  Ivanom_j  ogorčaet  ego_{i/j}.\\
%	murder Vitya-GEN Ivan-INS distresses him\\
%\trans `The murder of Vitya by Ivan$_i$ distresses him$_i$'
\end{xlist}
\end{exe}

\noindent
\REF{contPOSS} and \REF{contPOSS2} are ungrammatical, while \REF{contINS} and \REF{contINS2} are grammatical.

\newpage
Applying the structure of \isi{event} nominals in \figref{external} to \REF{contPOSS} and \REF{contINS}, the structure of \REF{contPOSS} and \REF{contINS} is illustrated in \figref{bindtreePOSS} and \figref{bindtreeINS}.
%\footnote{
%			Note that in \figref{bindtreeINS}, the nodes below XP$_2$ are omitted as they repeat the structure in \figref{bindtreePOSS}. }
I assume that the instrumental agents are specifiers/adjuncts to \isi{VoiceP} as the genitive external arguments in type 3 nominals.\footnote{This position can explain the fact that type 1 \isi{event} nominals cannot have instrumental agents as shown in \REF{ud}.}


\begin{figure}[h]
\caption{The structure of \REF{contPOSS}}
\label{bindtreePOSS}
\begin{forest}
  sn edges [ \isi{VoiceP}$_1$ [ Ivan$_i$'s ]
                        [ \isi{VoiceP}$_2$ [ \isi{Voice} [ violation ] ]
                                     [ XP [ X ]
                                          [ \textit{n}P [ \textit{n} ]
                                                      [ $\surd$P [ $\surd$ ]
                                                                 [ rules.{\GEN} ] ] ] ] ] ]
\end{forest}
\end{figure}

\begin{figure}[h]
\caption{The structure of \REF{contINS}}
\label{bindtreeINS}
%\Tree [.\isi{VoiceP}$_1$ [.\isi{VoiceP}$_2$ [.\isi{Voice} narušenie ] [.XP [.X  ] [.\textit{n}P \textit{n} [.$\surd$P [.$\surd$ ]!\qsetw{0.7in} {pravil} ] ] ] ]!\qsetw{1.2in} {Ivanom}$_i$ ]
\begin{forest}
  sn edges [ \isi{VoiceP}$_1$ [ \isi{VoiceP}$_2$ [ \isi{Voice} [ violation ] ]
                                     %[ {XP\\{...}}
                                     [ XP [ X ]
                                          [ \textit{n}P [ \textit{n} ]
                                                      [ $\surd$P [ $\surd$ ]
                                                                 [ rules.{\GEN} ] ] ]
                                     ]
                         ]
                        [ Ivan.{\INS}$_i$ ] ]
\end{forest}
\end{figure}



As mentioned above, \citet{Schoorlemmer1998} pointed out that only \isi{CEN} can have the instrument \isi{agentive} phrase. That is, X Agrees not with the specifier/adjunct but with the \isi{internal argument} since X does not c-command the specifier/adjunct but only the \isi{internal argument}, even under \citeposst{kayne1994} definition of c-com\-mand. Thus, the specifier/adjunct cannot be genitive.

In \figref{bindtreePOSS} and \figref{bindtreeINS}, both the instrumental agents `Ivan.{\INS}' and the \isi{possessive} \isi{adjective} `Ivan's' do c-command the objects \textit{ego} `him' since the \isi{VoiceP}$_1$ and \isi{VoiceP}$_2$ are segments.
In other words, this structure correctly predicts the ungrammaticality of \REF{contPOSS}, but
it also incorrectly predicts \REF{contINS} to be ungrammatical. Thus, the contrast in \REF{binins} cannot be captured with \figref{bindtreePOSS} and \figref{bindtreeINS}. Something needs to be added to address the observed contrast in grammaticality.

I assume that instrumental agents are introduced as PP with a silent P head as in \figref{ins2}. With this extra layer of structure there is no Condition B violation since the PP blocks the instrumental agent's c-commanding the object.

\begin{figure}[h]
\caption{The modified version of the structure of \REF{contINS}}
\label{ins2}
\begin{forest}
  sn edges [ \isi{VoiceP}$_1$ [ \isi{VoiceP}$_2$ [ \isi{Voice} [ violation ] ]
                                     [ XP [ X ]
                                          [ \textit{n}P [ \textit{n} ]
                                                      [ $\surd$P [ $\surd$ ]
                                                                 [ rules.{\GEN} ] ] ] ] ]
                        [ PP [ P [ φ ] ]
                                 [ Ivan.{\INS}$_i$ ] ] ]
\end{forest}
\end{figure}

\section{Generalized case realization requirement}\label{sec:GCRR}

To capture the contrast in \REF{binins}--\REF{binins2}, I assumed that instrumental agents are introduces by the silent P.
However, it is undesirable to utilize an abstract element with no evidence.
Thus, I need some evidence except the contrasts in \REF{binins}--\REF{binins2}.
In this section, I demonstrate that assuming the silent P in instrumental agent phrases is a consequence of \isi{GCRR}, proposed by \citet{Horvath2014}.


\subsection{GCRR in Serbo-Croatian}


\citet{Horvath2014} addressed the distribution of indeclinable nouns in Serbo-Cro\-a\-tian.
As shown in \REF{SCind},
the indeclinable name \textit{Miki} is ungrammatical although the declinable name \textit{Larisa} is grammatical in the oblique environment.

\begin{exe}
\ex\label{SCind}
\begin{xlist}
\ex
\gll	Divim se \minsp{\{} Larisi / \minsp{*} Miki\}.\\
		admire.1{\SG} {\REFL} {} Larisa.{\DAT} {} {} Miki\\
\trans	`I admire Larisa/Miki.'
%\newpage
\ex
\gll	Ponosim se \minsp{\{} Larisom / \minsp{*} Miki\}.\\
		be.proud.1{\SG} {\REFL} {} Larisa.{\INS} {} {} Miki\\
\trans	`I am proud of Larisa/Miki.'
%\newpage
\ex
\gll	Oduševjena sam \minsp{\{} Larisom / \minsp{*} Miki\}.\\
		impressed.{\PTCP}.\textsc{f} {\AUX}.1{\SG} {} Larisa.{\INS} {} {} Miki\\\samepage
\trans	`I am impressed by Larisa/Miki.'
\hfill\citep[121]{Horvath2014}
\end{xlist}
\end{exe}

\noindent
The indeclinable name is grammatical with the declinable \isi{possessive} \textit{moj} `my' or \isi{adjective} \textit{lep} `beautiful', but it is ungrammatical without them or with the indeclinable \isi{adjective} \textit{braon} `brown', as in \REF{SCmy}.


\begin{exe}
\ex\label{SCmy}
\begin{xlist}
\ex
\gll	Divim se *(\hspace{-2pt} mojoj) Miki.\\
		admire.1{\SG} {\REFL} {} my.{\DAT.\SG} Miki\\
\trans	`I admire (my) Miki.'

\ex
\gll	Oduševjena sam *(\hspace{-2pt} mojom) Miki.\\
		impressed.{\SG}.\textsc{f} {\AUX}.1{\SG} {} my.{\INS}.{\SG} Miki\\
\trans	`I am impressed by (my) Miki.'

\ex
\gll	Divim se \minsp{\{*} braon / lepoj\} Miki.\\
		admire.1{\SG} {\REFL} {} brown {} beautiful.{\DAT}.{\SG} Miki\\
\trans	`I admire \{brunette/beautiful\} Miki.'
\hfill\citep[121]{Horvath2014}
\end{xlist}
\end{exe}

\noindent
In addition, the indeclinable name is grammatical with P even without the declinable \isi{possessive} or \isi{adjective} as illustrated in \REF{SCprep}.


\begin{exe}
\ex\label{SCprep}
\begin{xlist}
\ex
\gll	On je trčao prema (\hspace{-2pt} lepoj) Miki.\\
		he {\AUX}.3{\SG} run.{\PTCP}.{\SG} towards {} beautiful.{\DAT.\SG} Miki\\
\trans	`He ran towards (beautiful) Miki.'

\ex
\gll	Dolazim sa (\hspace{-2pt} mojom) Miki.\\
		come.{1\SG} with {} my.{\INS} Miki\\
\trans	`I am coming with (my) Miki.'

\ex
\gll	Razgovarali smo o (\hspace{-2pt} mojoj) Miki.\\
		talk.{\PTCP.\PL} {\AUX.1\PL} about {} my.{\LOC} Miki\\
\trans	`I talked about (my) Miki.'
\hfill\citep[122--123]{Horvath2014}
\end{xlist}
\end{exe}

\noindent Accepting \citeposst{Pesetsky2013} theory of Case, \cite{Horvath2014} generalized these complicated phenomena of   \ili{Serbo-Croatian} indeclinable nouns in the form of \isi{GCRR} shown in \REF{GCRR}.


\begin{exe}\ex \textit{Generalized Case Realization Requirement (\isi{GCRR})} \label{GCRR}\\
Oblique cases must be overtly realized by some element of the
\textsc{assignment domain} (where assignment domain consists of the assigning head and the assignee -- its noun phrase complement).
\hfill\citep[125]{Horvath2014}
\end{exe}

\noindent
According to \isi{GCRR}, the sentences with the indeclinable name \textit{Miki} in \REF{SCind} are ungrammatical since no element in the assignment domain overtly realizes oblique cases.
As for \REF{SCmy}, sentences are grammatical even with the indeclinable name \textit{Miki} if the declinable \isi{possessive} \textit{moj} `my' or \isi{adjective} \textit{lep} `beautiful' overtly realizes oblique cases.
However it is ungrammatical without them or with the indeclinable \isi{adjective} \textit{braon} `brown' because of the absence of overt realization of oblique cases.
In the case of \REF{SCprep}, each P \textit{prema} `towards,' \textit{sa} `with,' \textit{o} `about' manifests oblique cases and thus
 the sentences are grammatical even if there is no declinable \isi{possessive} or \isi{adjective}.


\subsection{GCRR in Russian}

%There are some data supporting \isi{GCRR} also in \ili{Russian}.
There are also examples supporting the application of \isi{GCRR} in \ili{Russian}.\footnote{
            I consulted four \ili{Russian} native speakers in their twenties for acceptability judgments on \REF{gcrrRR}. All speakers found \REF{gcrrRRb} unacceptable, but there was variation among speakers on \REF{gcrrRRa} and \REF{gcrrRRc}; three consider them acceptable (but unnatural) and the other considers them unacceptable. The speakers accepting \REF{gcrrRRa} also accept \REF{gcrrRRc} and \textit{vice versa}. What is important here is that some speakers accept \REF{gcrrRRa}, \REF{gcrrRRc} and that a clear difference in acceptability can be found between  \REF{gcrrRRa}/\REF{gcrrRRc} and \REF{gcrrRRb}.
            }

\begin{exe}
\ex \label{gcrrRR}%Supporting evidence for \isi{GCRR} in \ili{Russian}
%null P licensing the instrumental Case:
\begin{xlist}
\ex[?]{\label{gcrrRRa}
\gll Professor priexal v Moskvu s okolo pjati studentov.\\
	professor arrived to Moscow with about five.{\GEN} students.{\GEN}\\
\trans `Professor arrived at Moscow with about five students.'}
\ex[*]{\label{gcrrRRb}
%Covert P: * (neither P nor its Comp realizes {\INS})
\gll Professor rukovodit okolo pjati studentov.\\
	professor supervise about five.{\GEN} students.{\GEN}\\
\trans Intended: `Professor supervises about five students.'}
\ex[?]{\label{gcrrRRc}
\gll Professor rukovodit okolo pjat'ju studentami.\\
	professor supervise about five.{\INS} students.{\INS}\\
\trans `Professor supervises about five students.'}
\end{xlist}
\end{exe}

\noindent
Sentence \REF{gcrrRRa} is grammatical with P requiring the \isi{instrumental case} \textit{s} `with'. In this case, there is an overt P and it realizes the \isi{instrumental case} in the assignment domain.
However, \REF{gcrrRRb}, without manifestations of the \isi{instrumental case}, is ungrammatical. This is because neither P nor its complement overtly realizes the \isi{instrumental case}.
In addition, \REF{gcrrRRc} is grammatical since the complement \textit{pjat' studentov} `five students' is declined to bear the \isi{instrumental case}, although the preposition \textit{okolo} `about'
requires its complement in genitive Case.
The (un)grammaticality of these sentences are predicted by \isi{GCRR}, which means that \isi{GCRR} is valid not only in Serbo-\ili{Croatian} but also in \ili{Russian}.


\citet{Pesetsky2013} points out examples like \REF{gcrrRR}.
As shown in \REF{gcrrRRPese}, without manifestations of the \isi{instrumental case}, the sentences show low acceptability and with P requiring the {instrumental case} or with an instrumental \isi{adjective}, the sentences are grammatical.


\begin{exe}
\ex \label{gcrrRRPese}
\begin{xlist}
\ex[]{
\gll Ja čital \textit{Po kom zvonit kolokol}.\\
	I read \textit{For Whom the Bell Tolls}\\
\trans `I read \textit{For Whom the Bell Tolls}.'}
\ex[??]{
\gll Ja vosxiščajus' \textit{Po kom zvonit kolokol}. \\
	I admire \textit{For Whom the Bell Tolls}\\
\trans `I admire \textit{For Whom the Bell Tolls}.'}
\ex[?]{
\gll Ja vosxiščajus' zamečatel'nym \textit{Po kom zvonit kolokol}.\\
	I admire marvelous.{\INS} \textit{For Whom the Bell Tolls}\\
\trans `I admire the marvelous \textit{For Whom the Bell Tolls.}'}
%\newpage
\ex[]{
\gll Pomnju, kak diko rydala nad \textit{Po kom zvonit kolokol}.\\
	remember.1 how wildly cried.{\SG}.\textsc{f} over \textit{For Whom the Bell Tolls}\\
\trans `I remember how wildly I cried over \textit{For Whom the Bell Tolls.}'}
\end{xlist}\hfill\citep[132]{Pesetsky2013}
\end{exe}

\noindent
In addition, if \isi{GCRR} is active, it is predicted that \REF{ivan} becomes ungrammatical when the declinable name \textit{Ivan} is replaced with an indeclinable name, as in that case there is no manifestation of the \isi{instrumental case}.
This is confirmed in \REF{shmit}, where the indeclinable name \textit{Šmidt} is used.

\begin{exe}
\ex\label{insIVSH}
\begin{xlist}
\ex[]{\label{ivan}
\gll  Narušenie  {pravil}  {Ivanom}$_i$  ogorčaet  ego$_i$.\\
	violation rule.{\GEN} Ivan.{\INS} distresses him\\
\trans `The violation of the rules by Ivan$_i$ distresses him$_i$.'}
\ex[*]{\label{shmit}
\gll  Narušenie  {pravil}  {Šmidt}$_i$  ogorčaet  ego$_i$.\\
	 violation rule.{\GEN} Schmidt distresses him\\
\trans Intended: `The violation of the rules by Schmidt$_i$ distresses him$_i$.'}
\end{xlist}
\end{exe}

\noindent
Thus there are reasons to assume the existance of a silent P, which introduces instrumental agent as proposed in \figref{ins2} since \isi{GCRR} is (at least roughly) valid in \ili{Russian} as mentioned above.
%In other words, there is no problem with assuming the existence of the silent P introducing instrumental agents since complements of P realizes the instrumental case as \isi{GCRR} indicates.


\section{Binding out of PP}\label{sec:PP}

\subsection{Functional and lexical prepositions}

\citet{YadroffFranks2001} proposed a distinction between \textsc{functional prepositions} (functinal P) such as \textit{u} `at', \textit{k} `toward',  \textit{bez} `without' and \textsc{lexical prepositions} (lexical P) such as \textit{okolo} `around', \textit{blagodarja} `thanks to', \textit{otnositel'no} `with respect to' in various grammatical respects as illustrated in  \tabref{FCP}.

%\newpage
\begin{table}
  \centering
  \caption{Functional and lexical prepositions \citep[70]{YadroffFranks2001}}
    \begin{tabularx}{\textwidth}{p{1em}p{13em}p{1em}p{13em}}\lsptoprule
    \multicolumn{2}{p{15em}}{Functional prepositions } & \multicolumn{2}{p{15em}}{Lexical prepositions} \\\midrule\addlinespace
    \multicolumn{2}{p{14em}}{\textit{Phonology}} & \multicolumn{1}{r}{} & \multicolumn{1}{r}{} \\
    A.    & Unstressed  & A.    & Stressed \\
    B.    & Monosyllabic & B.    & Polysyllabic \\\addlinespace
    \multicolumn{2}{p{14em}}{\textit{Morphology}} & \multicolumn{1}{r}{} & \multicolumn{1}{r}{} \\
    C.    & Monomorphemic & C.    & Often polymorphemic or compound \\
    D.    & Prothetic \textit{n}- before 3rd-person pronouns & D.    & No prothetic \textit{n}- \\\addlinespace
    \multicolumn{2}{p{14em}}{\textit{Syntax}} & \multicolumn{1}{r}{} & \multicolumn{1}{r}{} \\
    E.    & Object is obligatory & E.    & Object may be optional \\
    F.    & Approximative inversion yields N before P  & F.    & Approximative inversion yields P before N \\
    G.    & Negative particle \textit{ni} does not intervene & G.    & \textit{Ni} intervenes \\
    H.    & May be doubled in colloquial language & H.    & Cannot be doubled \\
    I.    & May be lexically selected & I.    & Cannot be lexically selected \\
    J.    & Allow binding out of PP & J.    & Block binding out of PP \\
    K.    & No intercalating particles & K.    & Intercalating particles \\
    L.    & May govern multiple cases & L.    & Govern one specific case \\\addlinespace
    \multicolumn{2}{p{14em}}{\textit{Semantics}} & \multicolumn{1}{r}{} & \multicolumn{1}{r}{} \\
    M.    & Meaning abstract (hence polysemous) & M.    & Meaning concrete (therefore fixed) \\\lspbottomrule
    \end{tabularx}
  \label{FCP}
\end{table}
%\vspace{-.8em}
%\hfill\citep[70]{YadroffFranks2001}
%\vspace{1em}

To capture the various differences between functional P and lexical P,
\citet{YadroffFranks2001} assume two different syntactic structures for  each type of P.  \figref{trFP} and \figref{trLP} give their structure of the two PPs in \REF{PP}.

\begin{multicols}{2}
\begin{exe}\ex \label{PP}
\begin{xlist}
\ex\label{Pk}
\gll  k ženščinam\\
	towards women.{\DAT}\\
\trans `towards women'
\ex\label{Pb}
\gll  blagodarja ženščinam\\
	thanks.to women.{\DAT}\\\samepage
\trans `thanks to women'
\end{xlist}
\end{exe}
\end{multicols}

\begin{figure}[t]
\caption{The structure of \REF{Pk} \citep[77]{YadroffFranks2001}}
%PP with functional P \textit{k} \citep[77]{YadroffFranks2001}}
\label{trFP}
%\Tree [.FP [.F k\\{[Target]}\\{[Dative]} ] \qroof{ženščinam}.NP ]
\begin{forest}
  sn edges [ FP [ F [ {towards\\\relax[Target\relax]\\\relax[Dative\relax]} ] ]
                [ NP [ women.{\DAT}, roof] ] ]
\end{forest}
\end{figure}

\begin{figure}[t]
\caption{The structure of \REF{Pb} \citep[78]{YadroffFranks2001}}
%PP with lexical P \textit{blagodarja} \citep[78]{YadroffFranks2001}}
\label{trLP}
%\Tree [.XP [.X blagodarja\\{[Complex θ-role]} ] [.FP [.F ø\\{[Beneficiary]}\\{[Dative]}\\{[+def]} ]!\qsetw{0.6in} \qroof{ženščinam}.NP ]!\qsetw{1in} ]
\begin{forest}
  sn edges [ XP [ X [ {thanks.to\\\relax[Complex θ-role\relax]} ] ]
                [ FP [ F [ {ø\\\relax[Beneficiary\relax]\\\relax[Dative\relax]\\\relax[+def\relax]} ] ]
                     [ NP [ women.{\DAT}, roof] ] ] ]
\end{forest}
\end{figure}

\noindent
``X'' is used to indicate that here, \textit{blagodarja} `thanks to' is a bleached lexical item, which lacks a functional structure.\footnote{
			Please see \citet{YadroffFranks2001} for more detail.
			}


\subsection{Binding possibility out of lexical PP}

I showed that a silent P is required to capture the contrasts in \REF{binins}--\REF{binins2}
and that assuming a silent P is valid as shown in the data regarding \isi{GCRR} in \ili{Russian} in \sectref{sec:GCRR}.

\largerpage[-4]
However, binding out of PP is not necessarily blocked as, for example, \citet{YadroffFranks2001} and \cite{Bailyn2010} point out.
With regard to binding possibility (J in \tabref{FCP}), a functional P allows binding out of PP as shown in \REF{ppbinda} and a lexical P blocks binding out of PP as in \REF{ppbindb}.\footnote{
			\citet{YadroffFranks2001} point out that there is also a similar contrast in \ili{English}.
			\ea
            \ea[]{John spoke to [Bill and Mary]$_i$ about each other$_i$'s birthdays.}
			\ex[*]{John spoke about [Bill and Mary]$_i$ in each other$_i$'s houses.}
\hfill \citep[74]{YadroffFranks2001}
			\z\z

			\noindent
			\citet{Bailyn2010} also points out data similar to \REF{ppbind}.
			Binding out of PP \textit{u Petrovyx} `at the Petrovs' is allowed.

			\ea\label{ppbind2}
			\gll	U Petrovyx$_i$ byla svoja$_i$ komnata.\\
					at the.Petrovs was self's.{\NOM} room.{\NOM}\\
			\glt `The Petrovs had their own room.'
			\hfill\citep[14]{Bailyn2010}
			\zlast}

\begin{exe}\ex \label{ppbind}
\begin{xlist}
\ex \label{ppbinda}
\gll	U ėtogo čeloveka$_i$ vsegda est' svoi$_i$ original'nye idei.\\
		at this person always be self's original idea\\\samepage
\trans `In that person's head there are always his own original ideas.'

\ex\label{ppbindb}
\gll	* Okolo ėtogo čeloveka$_i$ vsegda est' svoi$_i$ original'nye idei.\\
		{} around this person always be self's original idea\\
\trans Intended: `Around that person there are always his own original ideas.'
\end{xlist}\samepage\hfill\citep[74]{YadroffFranks2001}
\end{exe}

\noindent Given the grammaticality of \REF{ppbinda}, the silent P (ø) is a lexical P.
Therefore, the structure of \REF{contINS}, shown in \figref{ins2}, should be modified to \figref{trinsXP}.\footnote{
			There are two X(P)s in this tree.
			However, note that it is not guaranteed that they are the same projection.
			The ``X'' whose sister is \textit{n}P represents a genitive assigner as described in
			\cite{MiyauchiIto2016} and \cite{Miyauchi2017b}.
			The ``X'' whose sister is FP is used to indicate that the silent P is a bleached lexical item \citep{YadroffFranks2001}.
			}

\begin{figure}
\caption{The structure of \REF{contINS}}
\label{trinsXP}
\begin{forest}
  sn edges [ \isi{VoiceP}$_1$ [ \isi{VoiceP}$_2$ [ \isi{Voice} [ violation ] ]
                                     [ XP [ X ]
                                          [ \textit{n}P [ \textit{n} ]
                                                      [ $\surd$P [ $\surd$ ]
                                                                 [ rules.{\GEN} ] ] ] ] ]
                        [ XP [ X [ ø ] ]
                             [ FP [ F [ ø ] ]
                                  [ Ivan.{\INS}$_i$ ] ] ] ]
\end{forest}
\end{figure}


\section{Conclusion}\label{sec:CON}

In this paper, I have argued that instrumental agents in \ili{Russian} are introduced by a silent (lexical) P. As I have shown, this PP layer blocks binding of instrumental agents outside the \isi{event} nominal, which is otherwise possible for agents introduced as possessors. My analysis which assumes \ili{Russian} \isi{event} nominals (or noun phrases more generally) lack the DP layer also offers (at least partial) support to the idea that \isi{GCRR} is active in \ili{Russian}.



\section*{Abbreviations}

\begin{tabularx}{.5\textwidth}{lX}
1 & first person \\
3 & third person\\
\textsc{acc} & {accusative} \\
\textsc{Ag} & agent (θ-role) \\
\textsc{aux} & auxiliary \\
{CEN} & complex {event} nominals \\
\textsc{dat} & {dative} \\
\textsc{f} & feminine \\
{GCRR} & generalized case \\
 &realization requirement \\
\textsc{gen} & genitive  \\
\end{tabularx}
\begin{tabularx}{.45\textwidth}{lX}
\textsc{inf} & {infinitive} \\
\textsc{int-arg} & {internal argument} \\
\textsc{ins} & instrumental \\
\textsc{loc} & locative  \\
{PIC} & phase impenetrability\\
&condition \\
\textsc{pl} & plural \\
\textsc{Poss} & {possessor} (θ-role) \\
\textsc{ptcp} & {participle} \\
\textsc{refl}&{reflexive} \\
\textsc{sg} & singular \\
\end{tabularx}



\section*{Acknowledgements}
	I am grateful to the FDSL 12.5 participants for their valuable suggestions. In addition, I would like to thank anonymous reviewers for insightful comments. I also appreciate my informants reviewing the Russian data. The research reported here was supported by JSPS Grants-in-Aid for Scientific Research (\#17J07534, \#19K23073, PI: Takuya Miyauchi). All misunderstandings and errors remain my own.

\sloppy
\printbibliography[heading=subbibliography,notkeyword=this]
\il{Russian|)}
\end{document}
